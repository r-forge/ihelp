%
%  Part of the iHELP project at http://ihelp.r-forge.r-project.org
%
%  Copyright (C) 2013- The iHELP Working Group 
%                                in the Korean R Translation Team
%
%  This program is free software; you can redistribute it and/or modify
%  it under the terms of the GNU General Public License as published by
%  the Free Software Foundation; either version 2 of the License, or
%  (at your option) any later version.
%
%  This program is distributed in the hope that it will be useful,
%  but WITHOUT ANY WARRANTY; without even the implied warranty of
%  MERCHANTABILITY or FITNESS FOR A PARTICULAR PURPOSE.  See the
%  GNU General Public License for more details.
%
%  A copy of the GNU General Public License is available at
%  http://www.r-project.org/Licenses/
%

% Author: Chel Hee Lee
% Created on 2013-JUN-27
% Modified on 2013-JUN-27
% 
% Description:
% GNU R Study Guide 에서 필요한 제출형식을 Knitr를 이용하는 방법
% 
% 이 문서는 많은 사람들에게 중요한 문서 작성함에 있어 가이드라인이 되기 때문에 많이 손 보고 다양한 내용을 넣어 주어야 합니다.  iHELP 워킹 그룹멤버중에서 본 문서를 발전시키고자 하시는 분은 망설이지 말아주세요. 


\documentclass{article}\usepackage{graphicx, color}
%% maxwidth is the original width if it is less than linewidth
%% otherwise use linewidth (to make sure the graphics do not exceed the margin)
\makeatletter
\def\maxwidth{ %
  \ifdim\Gin@nat@width>\linewidth
    \linewidth
  \else
    \Gin@nat@width
  \fi
}
\makeatother

\definecolor{fgcolor}{rgb}{0.2, 0.2, 0.2}
\newcommand{\hlnumber}[1]{\textcolor[rgb]{0,0,0}{#1}}%
\newcommand{\hlfunctioncall}[1]{\textcolor[rgb]{0.501960784313725,0,0.329411764705882}{\textbf{#1}}}%
\newcommand{\hlstring}[1]{\textcolor[rgb]{0.6,0.6,1}{#1}}%
\newcommand{\hlkeyword}[1]{\textcolor[rgb]{0,0,0}{\textbf{#1}}}%
\newcommand{\hlargument}[1]{\textcolor[rgb]{0.690196078431373,0.250980392156863,0.0196078431372549}{#1}}%
\newcommand{\hlcomment}[1]{\textcolor[rgb]{0.180392156862745,0.6,0.341176470588235}{#1}}%
\newcommand{\hlroxygencomment}[1]{\textcolor[rgb]{0.43921568627451,0.47843137254902,0.701960784313725}{#1}}%
\newcommand{\hlformalargs}[1]{\textcolor[rgb]{0.690196078431373,0.250980392156863,0.0196078431372549}{#1}}%
\newcommand{\hleqformalargs}[1]{\textcolor[rgb]{0.690196078431373,0.250980392156863,0.0196078431372549}{#1}}%
\newcommand{\hlassignement}[1]{\textcolor[rgb]{0,0,0}{\textbf{#1}}}%
\newcommand{\hlpackage}[1]{\textcolor[rgb]{0.588235294117647,0.709803921568627,0.145098039215686}{#1}}%
\newcommand{\hlslot}[1]{\textit{#1}}%
\newcommand{\hlsymbol}[1]{\textcolor[rgb]{0,0,0}{#1}}%
\newcommand{\hlprompt}[1]{\textcolor[rgb]{0.2,0.2,0.2}{#1}}%

\usepackage{framed}
\makeatletter
\newenvironment{kframe}{%
 \def\at@end@of@kframe{}%
 \ifinner\ifhmode%
  \def\at@end@of@kframe{\end{minipage}}%
  \begin{minipage}{\columnwidth}%
 \fi\fi%
 \def\FrameCommand##1{\hskip\@totalleftmargin \hskip-\fboxsep
 \colorbox{shadecolor}{##1}\hskip-\fboxsep
     % There is no \\@totalrightmargin, so:
     \hskip-\linewidth \hskip-\@totalleftmargin \hskip\columnwidth}%
 \MakeFramed {\advance\hsize-\width
   \@totalleftmargin\z@ \linewidth\hsize
   \@setminipage}}%
 {\par\unskip\endMakeFramed%
 \at@end@of@kframe}
\makeatother

\definecolor{shadecolor}{rgb}{.97, .97, .97}
\definecolor{messagecolor}{rgb}{0, 0, 0}
\definecolor{warningcolor}{rgb}{1, 0, 1}
\definecolor{errorcolor}{rgb}{1, 0, 0}
\newenvironment{knitrout}{}{} % an empty environment to be redefined in TeX

\usepackage{alltt}

\usepackage{../mystyle}

\title{문서작성요령}
\author{R-iHELP Working Group \\ Chel Hee Lee}
\date{\today}
\IfFileExists{upquote.sty}{\usepackage{upquote}}{}

\begin{document}

\maketitle

\texttt{iHELP GNU R Study Guide}에 올려질 문서를 작성하여 배포하고자 하시는 분들을 위한 가장 단순한 형식을 정리한 문서입니다. 
본 작업지침서의 문서 역시 개발 문서 중 하나에 포함되므로, 본 문서를 더 발전 시켜나가는데 망설이지 마시길 당부드립니다.

\section{시작 전 작업환경 확인}

Ubuntu 13.04, texlive, R, 

\begin{itemize}
  \item 설치 중 아래와 같은 에러를 겪을 수 있습니다. 

    \begin{Schunk}
      \begin{Sinput}
In install.packages(``RCurl'', dependencies = TRUE) :
  installation of package 'RCurl' had non-zero exit status
      \end{Sinput}
    \end{Schunk}

    대부분의 경우는 아래와 같은 방법으로 해결이 가능합니다. 
    \begin{verbatim}
sudo apt-get install libcurl4-openssl-dev
    \end{verbatim}
    
\end{itemize}


\section{실행방법}

% \begin{Verbatim}[fontfamily=NanumMyungjoBold]
% \begin{Verbatim}[fontfamily=monospace]
% \begin{Verbatim}[fontfamily=NanumGothic]
% \begin{Verbatim}[fontfamily=helvetica]
% [fontfamily=Courier, fontshape=it, fontseries=b]
\begin{Verbatim}[fontfamily=rm]
$ Rscript -e "library(knitr); knit('inst.Rnw');"
$ pdflatex inst.tex
한글 폰트 테스트 
\end{Verbatim}

\begin{knitrout}
\definecolor{shadecolor}{rgb}{0.969, 0.969, 0.969}\color{fgcolor}\begin{kframe}
\begin{alltt}
\hlfunctioncall{library}(knitr)
\hlfunctioncall{knit2pdf}(\hlstring{"inst.Rnw"}, compiler = \hlstring{"pdflatex"})
\end{alltt}
\end{kframe}
\end{knitrout}



\section{문서 작성법}

R 코드청크를 $<<>>=$ 와 $@$ 사이에 넣어주어야 합니다. 
\begin{knitrout}
\definecolor{shadecolor}{rgb}{0.969, 0.969, 0.969}\color{fgcolor}\begin{kframe}
\begin{alltt}
\hlcomment{# 한글폰트 테스트}
.Machine$double.xmin
\end{alltt}
\begin{verbatim}
## [1] 2.225e-308
\end{verbatim}
\end{kframe}
\end{knitrout}


\begin{knitrout}
\definecolor{shadecolor}{rgb}{0.969, 0.969, 0.969}\color{fgcolor}\begin{kframe}
\begin{alltt}
\hlfunctioncall{sessionInfo}()
\end{alltt}
\begin{verbatim}
## R version 2.15.2 (2012-10-26)
## Platform: x86_64-pc-linux-gnu (64-bit)
## 
## locale:
##  [1] LC_CTYPE=en_US.UTF-8       LC_NUMERIC=C              
##  [3] LC_TIME=en_US.UTF-8        LC_COLLATE=en_US.UTF-8    
##  [5] LC_MONETARY=en_US.UTF-8    LC_MESSAGES=en_US.UTF-8   
##  [7] LC_PAPER=C                 LC_NAME=C                 
##  [9] LC_ADDRESS=C               LC_TELEPHONE=C            
## [11] LC_MEASUREMENT=en_US.UTF-8 LC_IDENTIFICATION=C       
## 
## attached base packages:
## [1] stats     graphics  grDevices utils     datasets  methods   base     
## 
## other attached packages:
## [1] knitr_1.2
## 
## loaded via a namespace (and not attached):
## [1] digest_0.6.2   evaluate_0.4.3 formatR_0.8    stringr_0.6.2 
## [5] tcltk_2.15.2   tools_2.15.2
\end{verbatim}
\end{kframe}
\end{knitrout}



% <<fig=TRUE>>=
% plot(rnorm(100))
% @ 


\section{유용한 링크}

\section{Reference}


\end{document}


  

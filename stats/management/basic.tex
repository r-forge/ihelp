
\documentclass{article}

\usepackage{verbatim}
\usepackage{Sweave}
\usepackage{amsmath, amssymb, amsthm}
\usepackage{kotex}

\begin{document}

이 문서는 R의 기본적인 사용법을 다루고 있습니다. 보다 자세한 내용은 R-iHelp 프로젝트 홈페이지의 [We Share] @href{http://ihelp.r-forge.r-project.org/shared.html} 섹션을 참고하시기 바랍니다.
또한 이 문서는 통계적 분석에서 R의 기술적인 활용을 다루고 있으므로 통계이론에 대해서는 다루고 있지 않습니다.

통계적 검정은 사고의 수준과 깊은 관련이 있습니다.

하나의 알고 싶은 대상이 있다고 가정합니다. 물론 이것이 실체가 있는지의 여부는 문제가 되지 않습니다. 그 이유는 통계학의 방법은 그 대상을 그대로 다루는 것이 아니라 대상의 속성을 규정하고 그 규정된 속성을 대신할 수 있는 지표를 개발하여 이 지표를 대상의 대용으로 삼기 때문입니다. 예를 들어, 어떤 한 상대가 자신의 이상형에 가까운지  알아보고자 한다고 생각해 보십시오. 물론 "난 느낌이 통하는 상대가 좋아"라고 말할 수도 있지만  통계적인 방법은 이성의 특징을 반영할 수 있는 속성를 먼저 규정합니다. 즉, 성별(비이성애자가 있을 수도 있으니까요.. 흠..), 나이차(연상인지 연하인지를 표현 할 수 있게 한다면 더 좋겠죠), 키, 몸무게, 눈크기, 얼굴폭, 얼굴길이, 가슴둘레, 엉덩이둘레, ... , 종교, 부모님 생존 여부, 저축금액, 대출금액, 피부가 하얀 정도, 성격은... 뭐.. 일단 넘어갑니다... 

% 아래건 주석 처럼하고 싶은데...

% 성격을 굳이 다루고자 한다면, 먼저 성격에 대해 생각해 봅니다. 여러분이 다른 사람의 성격을 판단할 때 고려하는 
% 요소는 무엇일까요?  이런 식인 것입니다. 
% 1. 그 사람은 다른 사람에게 친절하다.
% 2. 그 사람은 다른 사람에 대한 배려가 많다. 
% 3. 그 사람은 다른 사람과의 약속을 반드시 지키는 편이다.
% 4. 그 사람은 ...
% 등등의 항목을 1~5 점 또는 1~7점으로 평가합니다. 그리고 이렇게 평가한 점수의 평균을 일표본 평균 검정(t-검정
% 이나 F-검정)을 한 결과를 바탕으로 성격이 좋다 나쁘다를 말할 수도 있습니다. 

이제 이런 것들에 대한 평가가 이루어지고 나면 상대방이 내 이상형에 가까운 정도를 확인합니다. 확인하는  방법은 - 최소한 통계적으로는 - 일표본 평균 검정을 하는 것입니다. 즉 위에서 나열한 이상형 평가 속성에 대한 평가 점수의 평균을 바탕으로 이 평균값이 얼마나 높은가를 확률적으로 평가하는 것입니다. 일정한 확률분포를 가정한 뒤에 이 확률분포에서 속성 평가 점수의 평균이 확률분포에서 발생할 가능성이 얼마나 되나를 확인하는 것입니다. 점수가 높아서 그러한 점수가 나타날 확률이 매우 적다면 그 상대방은 내 이상형에 가까울 수 있다고 판단하는 것입니다.
그런데 여기까지에서는 상대방이 한 명인 점에 주의하십시오. 만약 여러분이 지난 주에 소개팅을 세 번을 하였거나 결혼정보업체에 등록을 해서 그 업체에서 세 명의 사람을 소개 받았다고 가정해 보십시오. 그러면 그 세 명의 상대방 중 한 명을 선택하여 집중하여야 할 것입니다(한 명 이상의 교제 상대를 유지하는 방법에 대해서는 다루지 않겠습니다). 


% 경영학에서 주로 데이터를 1차데이터(소프트 데이터)와 2차데이터(하드데이터)로 구분합니다. 1차데이터는 설문이나 실험 등을 통하여 획득한 데이터를 말하고 2차데이터는 주가지수, 물가수준, 환율 등의 데이터를 말합니다.


% 내장되어 있는 데이터 집합을 보기 위해서 @code{data()}를 입력합니다.

% \begin{Soutput}
% \begin{Schunk}
% > data()
% \end{Soutput}
% \end{Schunk}
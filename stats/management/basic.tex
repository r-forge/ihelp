
\documentclass{article}

\usepackage{verbatim}
\usepackage{Sweave}
\usepackage{amsmath, amssymb, amsthm}
\usepackage{kotex}

\begin{document}

이 문서는 R의 기본적인 사용법을 다루고 있습니다. 보다 자세한 내용은 R-iHelp 프로젝트 홈페이지의 [We Share] @href{http://ihelp.r-forge.r-project.org/shared.html} 섹션을 참고하시기 바랍니다.
또한 이 문서는 통계적 분석에서 R의 기술적인 활용을 다루고 있으므로 통계이론에 대해서는 다루고 있지 않습니다.

통계적 검정은 사고의 수준과 깊은 관련이 있습니다.

하나의 알고 싶은 대상이 있다고 가정합니다. 물론 이것이 실체가 있는지의 여부는 문제가 되지 않습니다. 그 이유는 통계학의 방법은 그 대상을 그대로 다루는 것이 아니라 대상의 속성을 규정하고 그 규정된 속성을 대신할 수 있는 지표를 개발하여 이 지표를 대상의 대용으로 삼기 때문입니다. 예를 들어, 어떤 사람의 이상형을 알아보고자 한다고 생각해 보십시오. 물론 그 사람의 얘기를 들을 수도 있지만 통계적인 방법은 이성의 특징을 반영할 수 있는 속성를 먼저 규정합니다. 즉, 성별(비이성애자가 있을 수도 있으니까요.. 흠..), 나이차(연상인지 연하인지를 표현할 수 있게 한다면 더 좋겠죠), 키, 몸무게, 눈크기, 얼굴폭, 얼굴길이, 가슴둘레, 엉덩이둘레, ... , 종교, 부모님 생존 여부, 저축금액, 대출금액, 


% 경영학에서 주로 데이터를 1차데이터(소프트 데이터)와 2차데이터(하드데이터)로 구분합니다. 1차데이터는 설문이나 실험 등을 통하여 획득한 데이터를 말하고 2차데이터는 주가지수, 물가수준, 환율 등의 데이터를 말합니다.


% 내장되어 있는 데이터 집합을 보기 위해서 @code{data()}를 입력합니다.

% \begin{Soutput}
% \begin{Schunk}
% > data()
% \end{Soutput}
% \end{Schunk}

\documentclass{article}

\usepackage{verbatim}
\usepackage{Sweave}
\usepackage{amsmath, amssymb, amsthm}
\usepackage{kotex}

\begin{document}

이 문서는 R의 기본적인 사용법을 다루고 있습니다. 보다 자세한 내용은 R-iHelp 프로젝트 홈페이지의 [We Share] @href{http://ihelp.r-forge.r-project.org/shared.html} 섹션을 참고하시기 바랍니다.

% 경영학에서 주로 데이터를 1차데이터(소프트 데이터)와 2차데이터(하드데이터)로 구분합니다. 1차데이터는 설문이나 실험 등을 통하여 획득한 데이터를 말하고 2차데이터는 주가지수, 물가수준, 환율 등의 데이터를 말합니다.


% 내장되어 있는 데이터 집합을 보기 위해서 @code{data()}를 입력합니다.

% \begin{Soutput}
% \begin{Schunk}
% > data()
% \end{Soutput}
% \end{Schunk}
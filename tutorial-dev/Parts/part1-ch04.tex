%  File tutorial-dev/Parts/part1-ch02.tex
%  Part of the iHELP project at http://ihelp.r-forge.r-project.org
%
%  Copyright (C) 2013- The iHELP Working Group 
%                                in the Korean R Translation Team
%
%  This program is free software; you can redistribute it and/or modify
%  it under the terms of the GNU General Public License as published by
%  the Free Software Foundation; either version 2 of the License, or
%  (at your option) any later version.
%
%  This program is distributed in the hope that it will be useful,
%  but WITHOUT ANY WARRANTY; without even the implied warranty of
%  MERCHANTABILITY or FITNESS FOR A PARTICULAR PURPOSE.  See the
%  GNU General Public License for more details.
%
%  A copy of the GNU General Public License is available at
%  http://www.r-project.org/Licenses/
%

\chapter{시간과 문자열}

데이터의 값은 숫자만 있는 것이 아니라 이름 또는 주소와 같은 문자열이 존재하거나, 생년월일, 입원일 또는 퇴원일과 같은 날짜 데이터가 있습니다.
R에서는 이러한 문자열과 날짜를 처리하는데 있어 유용한 기능들을 제공하고 있습니다.

\section{문자열 다루기}



\section{날짜와 시간을 다루기}

날짜와 시간을 다루는데는 크게 두가지로 나눌 수 있습니다.
하나는 2013년 05월 16일과 같이 시간정보 없이 날짜만을 다루는 Date 클래스, 그리고 다른 하나는 2013년 05월 16일 오후 06시 28분 33초 와 같이 날짜와 시간을 동시에 다루는 POSIXlt와 POSIXct 라는 클래스들입니다.
POSIXlt와 POSIXct 모두 시간을 초단위까지 다루지만, POSIXlt는 1970년 1월 1일 00시 00분 00초를 기준으로 현재의 날짜와 시간을 초단위로 기억해 두는 반면 POSIXct는 `sec', `min', `hour', `mday', `mon', `year', 'wday' `yday', `isdst'라는 아홉개의 구성요소를 가진 리스트로 되어 있습니다. 
여기에서 `mday'는 한 달중에 몇번째 날짜인지를 나타내고, `year'는 1990년 이래로 몇 번째 해인가에 대한 정보이며, `wday'는 1주일중 몇 번째 날짜인지 아려줍니다. 
그리고 `yday'는 1년중 몇번째 날인지에 대한 정보가 있으며, `isdst'는 일광시간절약제에 대한 정보를 담고 있습니다.
날짜와 시간에 대해서는 Ripley, B. D. 와 Hornik, K. (2001)가 작성한  \href{http://www.r-project.org/doc/Rnews/Rnews_2001-2.pdf}{Date-time classes} 이라는 내용을 읽어보시길 바랍니다. 
% ?POSIXct 의 예제 살펴보면 좋음 


% 베리굿 노트: http://www.pitt.edu/~njc23/
% http://cm.bell-labs.com/cm/ms/departments/sia/Sbook/
% http://developer.r-project.org/methodDefinition.html
% http://www.biostat.jhsph.edu/~rpeng/teaching.html
% http://onertipaday.blogspot.ca/
% http://web.udl.es/Biomath/Bioestadistica/R/Manuals/r_in_a_nutshell.pdf
% http://web.udl.es/Biomath/Bioestadistica/R/Manuals/

% \item \texttt{do.call()} 함수를 사용하는 법에 대해서..
% http://cran.r-project.org/web/packages/rockchalk/vignettes/Rchaeology.pdf
% http://www.stat.berkeley.edu/classes/s133/all2011.pdf  (다운로드 해두었음 Desktop/tmpRsrc/all2011.pdf)
% http://www.stat.berkeley.edu/classes/s133/resources.html 
	

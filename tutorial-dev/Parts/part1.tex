% 
% 이런 느낌은 특히 최소한 내가 알기로는 경영학과 경제학 분야(거의 확실함) 및 사회학, 심리학 분야가 특히 심하다. 
% 그리고 이들 분야에서는
% R의 다양한 기능(특히 visualization) 또는 packages를 필요로 하지 않을 가능성이 높다. 왜냐하면 분석 방법이 단순하기 때문이다.
% 여기에서 분석 방법이 단순하다는 것은 그 이론적 배경이 쉽다는 것을 의미하는 것이 아니라 절차상의 얘기이다.
% 언급한 분야에서 학술연구에서 실행되는 분석의 양을 보면 대부분 [기술통계 - 상관관계 - 차이분석, 회귀분석, 구조방정식분석] 정도로
% 분석이 마무리 되기 때문이다. 그리고 대부분 표(table)로 결과를 보고하는 형식을 따르고 있다.
% 분석 모형 자체를 다루는 연구가 거의 없는 것이다. 따라서 R의 작업 방식은 그들에게는 불필요할 지도 모른다.

% 그러나 여전히 R의 다양한 source를 필요로 하는 아주 작은 규모의 학술연구 분야는 존재한다. 따라서 여기의 tips에는 이들을 고려하여
% R 사용에 도움이 될 만한 사항들을 정리하고자 한다.
% 개인적으로는 통계학 및 다른 분야의 연구 과정이 어떻게 진행되는지 잘 모르는 것도 한 몫하여 이에 해당하는 부분은 다른 이의 도움이
% 절실하다.

% 덧붙여 나는 윈도우 환경에서의 R tip에 주목하고자 한다(Mac이나 유닉스/리눅스 환경에서의 R 사용에 관한 tip도 다른 이의 도움이 필요한
% 부분이다). 그 이유는 내가 윈도우에서 SPSS와 SAS를 숙련시켜 왔기 때문에 대부분의 통계분석 초심자들도 윈도우 환경에서 R 을 접하게
% 될 것이라는 막연한 억측 때문이다. 시간이 지나면서 여러 사람들을 만나보게 꼭 그렇지만은 않다는 것을 알게 되었지만 최소한 경영학과
% 경제학 분야에 속한 사람들은 그럴 것이라는 확신 때문이다. 이것은 다른 분야의 사람들을 홀대하는 것이 아니라 나의 능력의 한계 때문이다.
% 실제로 나의 경우에도 최근에 C 언어, CentOS, 그리고 MySQL도 공부하기 시작했으니 오해는 하지 말기 바란다(누군가는 포기하라고도 했다.
% 왜냐하면 지금 그것들을 공부하기에는 분량이 너무 많고 난이도가 낮지 않기 때문이다). 그런데 결국에는 내 전공분야에서의 R 사용은
% 그 분량이 많지 않고 리눅스/유닉스 기반의 R 사용에 보다 많은 시간과 지면을 할애하게 될 것이라는 생각이 든다.

% R은 최근의 빅데이터라는 화두와 함께 큰 주목을 받고 있다. 하지만 대부분의 사람들, 특히 사회과학 전공자들은 이 개념적 정의도 제대로
% 알지 못하며 들어보지 못한  전문용어(SQL, Hadoop, 그리고 Mapreduce 등)에 혹하고 있는 실정이다.
% 나의 경우 R을 구글링(googling)으로 공부하였다. 왜냐하면 내가 재학하던 대학에서는 R 교육 프로그램이 없었기 때문이다. 처음 접하게 된
% 것은 미시경제분석 수업이었다. 그 수업에서 교수님이 수업을 진행하시는데 R을 사용한 것이다(SAS도 조금 사용했었으나 SPSS 는 잡동사니
% 취급을 하셨던 것으로 기억한다). 누군가에게 강의를 받는 식의 교육은 그것이 전부였다. 이후의 학습은 모두 맨땅에 해딩과 계란으로
% 바위깨기 식이었다. 이 R tips 과정은 이러한 어려움을 덜어주기 위한 방안이 될 수도 있을 것이다.
% 그러기를 한참 후 어느 정도 익숙해지고 나니, 또 연구자로서의 건방이 살아나 R 교육이 어떻게 이루어지고 있는 지가 궁금해졌다.
% 이 부분은 나중에 다시 써 볼라오,,,

통계소프트웨어 \texttt{R}을 사용하는데 이 문서가 도움이 되길 바랍니다. 
아래와 같은 내용에 중점을 두고자 하였습니다. 

\begin{itemize}
\item 원리중심의 문제해결을 설명하고자, 다양한 패키지들로부터 제공되는 함수들에 대한 사용법에 대한 설명은 가급적이면 피하려고 하였습니다.
따라서, 이 문서내에서는 특정 패키지가 해결할 수 있는 경우를 제외하고는 BASE (기본) 시스템을 이용하여 문제를 해결할 수 있도록 정리하려고 노력합니다(다른 문서를 통해 R에서 제공하는 여러 패키지를 활용하는 사례들을 소개할 것입니다).

\item R이라는 언어의 특징을 살린 설명을 하고자 하였습니다.  
예를들면, R의 가장 큰 특징이라고 할 수 있는 벡터라는 개념을 활용하여 원리 중심의 설명을 제공합니다. 

\item 통계와 전산에 관련된 전문용어는 쉽게 풀어쓰거나, 용어에 대한 이해를 돕기 위한 참고 자료를 제공합니다. 
\end{itemize}

본 문서와 관련한 프로젝트의 시작일은 2013년 4월 10일이며, iHELP Working Group 관리자에 의하여 수시로 갱신되고 있습니다.
따라서, 이 문서는 어떤 특정한 기간을 두고 완성되지 않으며, 오로지 업데이트된 버전만이 존재합니다.

이 문서는 2005년 이래로 R의 사용에 대한 본 문서의 지은이 개인의 경험과 R Documentation 만을 토대로 하여 작성되었으므로 아직 경험하지 못하여 다루지 못하는 부분이 있습니다.
이 문서의 지은이는 통계학에 대한 
%  B.Sc. (Statistics), M.Sc. (Statistics), 그리고 PhD. Biostatistics Candidate 의 
배경지식을 가지고 있으므로 문서의 내용이 다소 통계 및 수학 분야에 치중하였을 수 있습니다.
지은이는 이 문서가 보다 다양한 계층의 분들께서 R을 이용하여 하고자 하시는 일에 밑거름이 될 수 있는 도움이 되고자 합니다.
따라서, 이 문서를 읽고 있는 독자가 문서 내용에 대한 추가, 수정, 및 제안이 있다면 \href{mailto:gnustats@gmail.com}{gnustats@gmail.com} 또는 \href{mailto:ihelp-urquestion@lists.r-forge.r-project.org}{ihelp-urquestion@lists.r-forge.r-project.org} 의 주소로 이메일을 보내주신다면 감사하겠습니다. 

또한, 아래에 기재된 분들 (가나다 순)의 관심과 제안들이 없었다면 이 문서가 발전될 수 없었기에 그 감사의 말씀을 올리고 싶습니다. 
\begin{itemize}
\item \href{jhshin@assist.ac.kr}{신종화 교수님} (서울종합과학대학원 사회학과)
\item \href{mailto:buillee@hanmail.net}{이부일 박사님} (충남대학교 정보통계학과)
\item \href{mailto:muoe78@gmail.com}{정우준 박사님} (홍익대학교 경영학과)
\item \href{mailto:christ00@hanmail.net}{이성윤님}
% \item 이영섭 교수님 (동국대학교 통계학과)
\end{itemize}

\paragraph{이 문서를 읽는 방법:} 
\begin{itemize}
\item ``완전초보에요''라는 챕터와 ``기초프로그래밍과 운영체제'' 챕터는 이 문서를 읽기 전에 반드시 먼저 읽으셔야 합니다.  
\end{itemize}

%%%%%%%%%%%%%%%%%%%%%%%%%%%%%%%%%%%%%%%%%%%%%%%%%%%%%%%%%%%%%%%%%%%%%%%%
%
% CHAPTER
%
%%%%%%%%%%%%%%%%%%%%%%%%%%%%%%%%%%%%%%%%%%%%%%%%%%%%%%%%%%%%%%%%%%%%%%%%

\chapter{완전 초보에요}

이 문서에서 ``초보''라는 의미는 아래에 나열된 사항들중 두 가지 이상에 해당되시는 분들을 의미합니다. 

\begin{itemize}
\item R 이라는 프로그래밍 언어 이전에 다른 프로그래밍 언어에 대한 경험이 전무하신 분,
\item 유닉스와 리눅스 시스템에 익숙하지 않으신 분,
\item 기초 통계 분석에 대한 도움이 필요하신 분,
\item 그냥 무엇을 해야할지 막막함에 쌓여 계신분 
\end{itemize}

이에 해당하시는 분들은 꼭 ``사용전 반드시 알아야 할 7가지 숙지사항'' 섹션을 꼭 읽어주시길 부탁드립니다.
R을 사용하시는데 있어 숙지사항의 내용을 기억하신다면, 매우 도움이 될 것입니다. 


%%%%% SECTION START 
\section{꼭 먼저 알아야 할 7가지}

초보라고 생각하시는 분들께서는 아래의 내용들에 미리 알고 계시면 R을 사용하는데 도움이 됩니다.

\paragraph{데이터 입력과 처리:}  
R 에서는 이용되는 모든 데이터들에 대한 처리는 열방향으로 이루어집니다.
이 말의 뜻은 데이터의 입력 및 변형, 그리고 연산에 사용되는 데이터들에 대한 처리순서는 열방향으로 나열된 후에 이루어지는 것을 말합니다. 
예를들어, 1부터 12까지 12개의 정수로 이루어진 수열 (즉, $\{ 1, 2, 3, 4, 5, 6, 7, 8, 9, 10, 11, 12 \}$)을 R은 아래와 같이 이해합니다. 
열방향으로 이루어진다고 하더라도 수학적인 표현은 동일합니다(즉, 4행 3열 또는 4x3 행렬).
% 데이터 불러오기
% 보통 사람들은 R 콘솔에서 직접 입력한 데이터를 제외하고 그 외의 데이터는 *.txt, *.xlsx 이든 *.sav이든 모두 외부데이터로 생각하기
% 일쑤이다.
% 즉 foreign 팩키지를 사용해야 하는 경우를 제대로 구분하지 못한다. 그러니,
% > library(foreign)
% 을 선언할 줄 모르는 것은 당연하다.
%
% 그리고 인코딩이란 무엇인지를 모르는 사람은 보다 맞다.
%
\begin{Schunk}
\begin{Soutput}
1  5   9
2  6  10
3  7  11
4  8  12
\end{Soutput}
\end{Schunk}
%
따라서, 이 행렬에서 5라는 숫자값은 행렬의 1행 2열에 위치하고 있다고 할 수 있으며, 행렬의 5번째의 값이라고도 할 수 있습니다. 5번째라고 하는 것은 R이 열기준으로 데이터를 인식하는 이유 때문입니다(만약 어떤 프로그램이 행으로 데이터를 인식한다면 5번째의 값은 6이 됩니다). 

\paragraph{대화식 사용을 통한 분석결과를 확인하는 방법에 대해서:} 
R은 사용자가 주어지는 업무를 시키는 대로만 수행하는 유용한 프로그램일뿐, 그 이상의 내용은 수행하지 않는다는 점을 반드시 명심하셔야 합니다. 
따라서, R 프로그램을 시작하게 되면 아래와 같은 기호로 표시되는 프롬프트 (즉, 사용자의 명령어를 기다리는 기호)를 보여주게 됩니다. 
모든 명령어는 $>$ 기호 뒤에 작성하게 됩니다.  
가장 중요한 것은 분석자가 머릿속으로 상상하거나 기대하는 업무가 있다면, 분석자는 반드시 그 업무가 이루어지는 프로세스에 대해서 잘 알고 있어야 합니다.
따라서, R은 다른 통계 소프트웨어들과 같이 분석된 결과를 미리 보여주거나 혹은 분석이 된 모든 결과를 한 번에 다 보여주지 않습니다.
분석자가 확인하고자 하는 결과를 R에서 제공하는 함수를 통하여 중간 결과물들을 확인할 수 있습니다. 
우리는 분석자가 작업을 어떻게 진행하고 확인할 것인가에 대한 프로세스 차트를 먼저 작성한뒤 분석을 수행하기를 권장합니다.
%

\paragraph{통계분석의 절차:} 
분석이라 것은 데이터에 대한 이해를 통하여 데이터가 가진 특징을 수학적 표현으로서 설명하기 위한 과정입니다. 
따라서, 데이터에 대한 직관적인 이해를 위해서 시각화 작업과 통계 모형이 요구하는 데이터 형식을 만드는 것이 중요합니다. 
이를 전처리 과정이라고 하며, 통계분석은 크게 아래와 같은 절차를 밟아 이루어집니다.
%
	\begin{enumerate}
	\item 데이터 입출력과 클리닝, 그리고 분석 전처리 관련 테크닉들
	\item 분석전 탐색적 시각화 작업 
	\item 통계모형의 결정 및 적용 
	\item 모형 적용 후의 보고서 생성 및 시각화 작업
	\item 통계 모형 자체의 개발 또는 자동화 시스템 구축.
	\end{enumerate}
	
따라서, 이 문서는 위에서 설명한 과정에 해당하는 순서대로 챕터들을 구성하였습니다. 
%
%

\paragraph{한글 표현과 인코딩에 관련하여:}   
\texttt{R}은 한국어 사용자를 위한 한국어 인터페이스를 지원하고 있습니다. 
그러나, 사용자는 이러한 한국어 지원이 단순히 사용자의 편의를 위한 선택적 사항이라는 점을 반드시 알고 있어야 합니다.
또한, 분석시 본래의 정교한 표현은 번역된 한국어 보다는 원래의 영문이라는 점도 잊지 마셔야 합니다. 
%
현재 한국어에 관련된 작업은 UTF-8이라는 인코딩에 기반하여 한국어 품질관리 프로그램 (\url{http://ihelp.r-forge.r-project.org/lang_msg.html})을 통하여 이루어 지고 있으나, \texttt{R}을 한국어가 아닌 영문으로 설정하기 위해서는 다음의 주소를 눌러 그 내용을 확인해주시길 부탁드립니다. 
이 내용은 버전에 관계없이 일반적으로 통용되는 방법이나 윈도우즈 사용자에게 맞추어 작성되었습니다. 

\url{http://lists.r-forge.r-project.org/pipermail/ihelp-urquestion/2013-April/000003.html}

본래 데이터를 자국의 언어로 표현하는 방법은 UTF-8 이라는 인코딩을 이용하므로, 간혹 데이터가 올바르게 보이지 않을 경우가 있습니다. 
이러한 문제를 해결하기 위해서는 ``R을 지원하는 인코딩 중 올바르게 한국어를 표현해주는 코드를 찾는 방법'' (\url{http://lists.r-forge.r-project.org/pipermail/ihelp-urquestion/2013-April/000017.html}) 를 읽어보시길 바랍니다.

간혹, 잘못된 한글처리가 시스템 에러를 야기하는 경우가 있으므로, 이러한 경우에 한국어로 R을 사용하시는 분들께서는 다소 번거로우실지라도 보다 안정적이고 나은 R을 제공하기 위하여 그 내용을 \href{mailto:ihelp-urquestion@lists.r-forge.r-project.org}{ihelp-urquestion@lists.r-forge.r-project.org} 주소로 이메일을 보내주신다면 감사하겠습니다. 

단, 이러한 한글처리는 분석에 연관된 수치연산과는 아무런 관계가 없음을 반드시 알아주시길 부탁드립니다.
%

\paragraph{대소문자 구분에 관련하여:}  많은 분들이 이전에 SAS 소프트웨어를 사용하여 분석을 수행하셨을 것입니다.
SAS에서는 대소문자를 구분하지 않고 프로그램을 작성하게 되지만, R에서는 대소문자를 구분하므로 A 라는 변수와 a 라는 변수는 서로 다른 것임을 명심하시길 부탁드립니다. 
%

\paragraph{객체:}  R에서 다루어지는 모든 것을 객체라고 합니다.

\paragraph{함수의 사용에 관련하여:}
R보다도 일반적으로 프로그래밍을 다소 익숙하게 다룬다는 것은 하고자 하는 업무에서 어떤 함수가 적재적소에 쓰여야 할 지를 아는 것입니다. 
	따라서, 제공되어는 함수가 무엇이 있으며 어떤 함수가 언제 어떻게 사용되는가를 알고 있는 것은 작업하는데 도움을 줍니다. 
	함수를 사용할 때,  R은 지시된 인자(named argument)와 함께 사용하는 것이 좋습니다.  

\paragraph{패키지와 관련하여:}
R은 Add-on이라는 패키지 시스템을 이용합니다. 
이것은 기본 베이스 시스템에 추가로 필요한 기능들을 추가하는 의미입니다.
이러한 내용을 잘 모르는 상태에서 초보자가 가장 많이 겪는 실수는 어떤 함수를 사용하고자 할 때 ``xxx 함수가 없습니다'' 또는 ``xxx 함수를 찾을 수 없습니다''입니다.
이는 사용하고자 하는 함수가 R 기본 배포판에 포함되어 있지 않은 어떤 사용자에 의해서 제공된 특정한 패키지내에서 존재하기 때문입니다.
이런 경우에는 먼저 사용하고자 하는 함수가 어떤 패키지에 존재하는지 알아야 합니다.  
그리고, 해당 패키지를 설치했을 때에는 설치된 패키지를 사용할 수 있도록 로딩하는 과정을 거쳐야 합니다.

 \begin{Schunk}
 \begin{Soutput}
 > library(pkg_name)	
 \end{Soutput}
\end{Schunk}

패키지의 설치, 확인에 관련된 사항은 ``통계모형의 선택과 적용''이라는 챕터에 기록해두었습니다. 



%%%%% SECTION START 

\section{왜 R을 사용하나요?}

R을 사용하는 이유는 아마도 아래와 같은 이유이기 때문입니다. 
(R Documentation에는 없는 이 문서의 지은이 개인의 생각임을 명심하시길 바랍니다)

\begin{enumerate}
	\item R은 매우 다양한 분야에서 개발되고 적용되는 최신 통계기법을 적용할 수 있는 자유소프트웨어이기 때문입니다.
	\item 행렬기반의 객체지향적 프로그래밍 언어이기 때문입니다.
	\item 다른 소프트웨어들에 비교하여 문법적 사용의 자유롭기 때문일 것입니다.
\end{enumerate}


\section{통계소프트웨어의 종류}

R이라는 통계소프트웨어를 대체할 수 있는 다른 소프트웨어들은 다음과 같습니다. 

\begin{itemize}
\item S-PLUS (상업용 버전의 S 언어 소프트웨어)
\item MATLAB (R과 같은 행렬기반의 언어)
\item SPSS
\item Octave (MATLAB의 GNU 버전)
\item Python (프로그래밍 언어)
\item SAS
\item STATA
\end{itemize}
% Gretl


%%%%%%%%%%%%%%%%%%%%%%%%%%%%%%%%%%%%%%%%%%%%%%%%%%%%%%%%%%%%%%%%%%%%%%%%
%
% CHAPTER
%
%%%%%%%%%%%%%%%%%%%%%%%%%%%%%%%%%%%%%%%%%%%%%%%%%%%%%%%%%%%%%%%%%%%%%%%%


\chapter{사용환경에 익숙해지기}

R은 연구활동의 수행을 위한 도구일 뿐입니다. 
이 도구를 잘 활용한다는 것은 그만큼 주어진 사용환경에 익숙해져 있다는 의미와 같습니다.
따라서, 본 챕터에서는 R을 처음에 접하는 사용자들이 R이라는 작업도구를 사용함에 익숙해지도록 먼저 계산기능으로서의 측면을 벡터, 행렬, 그리고 배열이라는 것을 통하여 먼저 살펴보도록 하겠습니다.
이들을 설명하는 이유는 추후에 다루어질 모든 내용들이 이것들과 관계가 있기 때문입니다. 


\section{R의 시작과 종료}

먼저 R을 시작하면 아래와 같은 화면을 볼 수 있습니다.

\begin{Schunk}
\begin{Soutput}
R version 2.15.1 (2012-06-22) -- "Roasted Marshmallows"
Copyright (C) 2012 The R Foundation for Statistical Computing
ISBN 3-900051-07-0
Platform: i686-pc-linux-gnu (32-bit)

R is free software and comes with ABSOLUTELY NO WARRANTY.
You are welcome to redistribute it under certain conditions.
Type 'license()' or 'licence()' for distribution details.

  Natural language support but running in an English locale

R is a collaborative project with many contributors.
Type 'contributors()' for more information and
'citation()' on how to cite R or R packages in publications.

Type 'demo()' for some demos, 'help()' for on-line help, or
'help.start()' for an HTML browser interface to help.
Type 'q()' to quit R.

> 
\end{Soutput}
\end{Schunk}

그리고, $>$ 라는 기호 바로 뒤에 깜빡깜빡 거리는 커서를 볼 수 있을 것입니다. 
이것은 R이 사용자의 입력을 기다린다는 의미입니다. 
이 기호를 프롬프트라고 하고, 프롬프트 기호 바로 뒤에서부터 명령어를 작성합니다. 

사용자의 명령이 끝났음을 알려주는 것이 엔터키입니다. 
R이 입력된 명령어를 처리하여 사용자에게 명령어의 수행 결과를 보여주는 것은 아래와 같습니다.
이는 항상 $[1]$ 이라는 기호로 시작합니다. 
실제로 $[1]$ 이라는 것은 추후에 설명하겠지만 첫번째 객체를 의미하는 것입니다.

\begin{Schunk}
\begin{Soutput}
> 1
[1] 1
> 1+1
[1] 2
\end{Soutput}
\end{Schunk}

이렇게 사용하는 방식을 R 소트트웨어와 대화식으로 사용한다고 하며, 이러한 대화가 이루어지는 공간 (즉, 사용자가 R에게 질의응답을 하는 곳)을 콘솔이라고 합니다.

% 콘솔 이미지 입력

작업을 다 한 뒤에 종료를 하고자 한다면 q() 명령어를 이용하여 R의 사용을 종료할 수 있습니다. 

\begin{Schunk}
\begin{Soutput}
> q()
Save workspace image? [y/n/c]: n
\end{Soutput}
\end{Schunk}

이때 workspace image (작업공간의 이미지)를 저장하겠는가라는 메시지를 보게 될 것입니다. 
작업공간 이미지라는 것은 콘솔내에서 현재 세션에서 작업을 해온 명령어들의 기록과 데이터들을 보관하겠는가에 대한 질문입니다. 

%경우에 따라 다르겠지만 콘솔의 작업 내용을 저장하기 보다는 스크립트에 저장하는 방법이 보다 편리할 수 있습니다. 
%다른 경우는 특히 데이터가 크고 연산에 시간이 오래 걸리는 분석의 경우 콘솔의 작업내용을 저장하는 것을 고려할 수 있습니다.

\section{1차원 벡터와 벡터라이제이션}

R의 가장 큰 특징은 모든 연산에 관련된 기능들이 벡터방식으로 동작하는 것을 의미합니다. 
벡터라는 개념은 컴퓨터가 어떠한 임의의 것을 1차원에서 다루는 것을 의미합니다.
예를들어, 수학에서 1이라고 표시하는 것을 컴퓨터가 알아듣게 입력을 한다면 단순히 1 이라는 값만을 입력하면 됩니다. 
그러나, 실제 작업에서 우리는 동일한 연산을 반복해야 하는 경우가 있습니다. 
따라서, 숫자 1이 아닌 1, 3, 5, 7, 9 와 같은 여러개의 숫자들을 나열한 수열의 형태를 만듭니다. 
이제, 1,3,5,7,9는 수열의 구성요소가 되는 것입니다. 
이러한 수열의 표현방식을 R이 알아듣도록 하는 것을 우리는 벡터라는 개념이라고 이야기 합니다. 

벡터방식이란 개념을 이해하기 위해서 계산기로서 R은 어떤 기능을 가지고 있는지 먼저 이해해야 합니다.

\subsection{산술연산}

먼저 R은 아래와 같이 마치 간단한 전자계산기와 같이 사용하는 것도 가능합니다.

\begin{Schunk}
	\begin{Soutput}
# 사칙연산 (더하기, 빼기, 곱하기, 나누기)
	\end{Soutput}
\end{Schunk}

\begin{equation}
1 + 2 - 3 + 4*5 - 6/3
\end{equation}

이를 R로 수행하기 위해서는 다음과 같이 입력합니다. 

\begin{Schunk}
	\begin{Soutput}
> 1 + 2 - 3 + 4*5 - 6/3
[1] 18
	\end{Soutput}
\end{Schunk}

연산자라는 것은 어떤 특정 역할을 수행하는 기호이며, 산술연산에 대한  연산자 우선순위는 수학적 연산순서와 동일합니다. 

다음과 같은 다양한 수학적인 연산이 가능합니다. 

\begin{Schunk}
	\begin{Soutput}
# 제곱 
> 3^2
[1] 9

# 지수 
> exp(3)
[1] 20.08554
 
# 로그 
> log(3)
[1] 1.098612
 
# 파이상수 값 
> pi
[1] 3.141593

# 삼각함수 사인, 코사인, 탄젠트  
> sin(0)
[1] 0

> cos(0)
[1] 1

> tan(45)
[1] 1.619775
 
# 몫과 나머지 구하기 
> 15/4
[1] 3.75
> 15 %/% 4
[1] 3
> 15 %% 4
[1] 3
> 15 %% 3
[1] 0
> 15 %% 2
[1] 1
 
\end{Soutput}
\end{Schunk}

이제 R의 벡터단위의 연산이라 것을 알아보도록 합니다. 

먼저 아래와 같이 1, 2, 3, 4 라는 각각의 숫자들을 하나로 묶어 하나의 벡터 (즉, 수열)을 생성해보도록 합니다. 
이때, 이렇게 숫자들을 하나로 묶어 주는 것은 c()라는 함수를 통하여 이루어지게 됩니다.

\begin{Schunk}
\begin{Soutput}
> c(1,2,3,4)
[1] 1 2 3 4
\end{Soutput}
\end{Schunk}

그런데, 매번 이렇게 숫자들의 묶음을 일일이 손가락으로 입력해야 한다면 사용이 매우 번거로울 것입니다. 
그래서 R은 이러한 입력을 임시적으로 저장을 해주는 변수라는 기능을 제공합니다. 여기에서 R은 별도로 변수선언을 할 필요는 없습니다.
따라서 아래와 같은 방법은 R에서 내가 묶은 일련의 숫자들을 x 라고 불리는 변수에 저장을 하겠다는 의미로 받아들입니다.
이를 c(1,2,3,4)라는 벡터를 x에 대입 또는 할당한다고 말합니다.
따라서 이제부터는 변수 x를 입력하기만 하면 위에서 입력한 일련의 숫자들이 기억되고 있다는 것을 확인할 수 있습니다. 

\begin{Schunk}
\begin{Soutput}
> x <- c(1,2,3,4)
> x
[1] 1 2 3 4
\end{Soutput}
\end{Schunk}

위에서 할당을 $<-$ 기호를 사용하였으나, 이러한 할당은 아래와 같은 방법으로도 가능합니다. 

\begin{Schunk}
	\begin{Soutput}
> x <- c(1,2,3,4)
> x
[1] 1 2 3 4
> c(1,2,3,4) -> x
> x
[1] 1 2 3 4
> x = c(1,2,3,4)		
> x
[1] 1 2 3 4
\end{Soutput}
\end{Schunk}

이는 단순히 사용자의 편의를 위한 것뿐이니 마음에 들어하는 방식을 골라서 사용하시면 됩니다. 

이제 R이 정말로 x라는 것을 벡터로 인식하는지 \texttt{is.vector()}라는 함수를 사용하여 알아보도록 합니다. 

\begin{Schunk}
\begin{Soutput}
> is.vector(x)
[1] TRUE
\end{Soutput}
\end{Schunk}

한개의 숫자와 하나의 벡터를 구분짓는 특징은 몇개의 숫자들이 한데 묶였는가 이므로 이를 확인하기 위해서는 length()라는 함수를 사용합니다. 

\begin{Schunk}
\begin{Soutput}
> length(x)
[1] 4
\end{Soutput}
\end{Schunk}

\paragraph{벡터단위의 연산}  이제 벡터단위의 연산이 무엇인가 알아보도록 합니다. 
위에서 생성한 x 라는 변수에 3 을 더해봅니다. 
이때, 우리는 상식적으로 x라는 변수에 있는 벡터를 구성하는 모든 구성요소에 3이 더해진 결과를 볼 수 있기를 기대합니다. 

\begin{Schunk}
\begin{Soutput}
> x+3
[1] 4 5 6 7	
\end{Soutput}
\end{Schunk}

이제 x라는 벡터에 제곱을 해봅니다. 
여러분들은 벡터의 각 구성요소의 값들이 제곱이 되었음을 알 수 있습니다. 
\begin{Schunk}
\begin{Soutput}
> x^2
[1]  1  4  9 16
\end{Soutput}
\end{Schunk}

이번에는 10으로 나누어 보겠습니다.

\begin{Schunk}
\begin{Soutput}
> x/10
[1] 0.1 0.2 0.3 0.4
> 
\end{Soutput}
\end{Schunk}

이렇게 어떤 벡터에 특정 주어진 연산을 수행하고자 할때 벡터의 각 구성요소에 주어진 연산이 한 번에 모두 수행되는 것을 바로 벡터단위의 연산이라고 합니다. 
이를 좀 더 멋있게 부르는 말이 어떤 연산에 대한 벡터라이징이라고 하는 것입니다. 


\subsection{벡터를 생성하는 다양한 방법들}

위에서는 단순히 c()라는 함수를 이용하여 벡터를 생성하였으나, R은 다양한 보다 다양한 방법을 제공하고 있습니다. 

\paragraph{seq()함수의 사용} 먼저 아래와 같이 사용되는 seq()라는 함수가 있습니다. 

\begin{Schunk}
\begin{Soutput}
> seq(0, 1, 0.1)
 [1] 0.0 0.1 0.2 0.3 0.4 0.5 0.6 0.7 0.8 0.9 1.0
\end{Soutput}
\end{Schunk}

위에서 사용한 명령어는 0과 1 사이에 0.1 을 단위로 하는 시퀀스(즉, 수열)을 생성하라는 명령어입니다. 

여기에서 R을 사용하는 한가지 팁을 알려주고자 합니다.
seq() 함수의 예제와 같이 사용하고자 하는 함수에는 여러가지 입력받는 인자들이 있습니다. 
입력되는 인자들의 순서가 틀리면 R은 잘 못된 결과를 보여줍니다. 
따라서, 이러한 실수를 줄이고자 함수를 사용할때 가급적이면 인자명과 함께 사용하는 것이 좋습니다.  
인자명을 영문으로는 argument name 이라고도 할 수 있으나, R Documentation에서 이를 설명하거나 시스템 메시지에서 사용되는 경우는 주로 named argument(지시된 인자) 이라고 합니다. 
따라서, 영문 메시지를 읽을때 named argument 란 단어가 보인다면 이는 함수에 사용된 인자명을 말하는 것입니다. 

\begin{Schunk}
\begin{Soutput}
> seq(from=0, to=1, by=0.1)
 [1] 0.0 0.1 0.2 0.3 0.4 0.5 0.6 0.7 0.8 0.9 1.0
\end{Soutput}
\end{Schunk}

by 라는 인자명을 사용하지 않고, length.out 이라는 인자명을 이용한다면 아래와 같은 결과를 얻을 수 있습니다. 
이는 0 부터 1 사이에서 균등한 길이를 가지는 5개의 숫자를 생성한다는 의미입니다. 

\begin{Schunk}
\begin{Soutput}
> seq(from=0, to=1, length.out=5)
[1] 0.00 0.25 0.50 0.75 1.00
\end{Soutput}
\end{Schunk}


만약 by 라는 인자를 넣지 않는다면 seq()함수의 기본작동원리는 by=1이라고 가정하며, 이를 우리는 디폴트 (기본값)이라고 합니다. 

\begin{Schunk}
\begin{Soutput}
> seq(0, 10)
 [1]  0  1  2  3  4  5  6  7  8  9 10	
\end{Soutput}
\end{Schunk}

\paragraph{콜론 연산자의 사용} 

seq()라는 함수외에도 이러한 수열은 콜론 연산자를 이용하여 생성할 수도 있습니다. 
먼저 seq(0,5,1)과 같은 결과를 콜론을 이용하여  생성해보도록 합니다. 

\begin{Schunk}
\begin{Soutput}
> 0:5
[1] 0 1 2 3 4 5
\end{Soutput}
\end{Schunk}

이 수열을 거꾸로도 생성할 수 있습니다. 

\begin{Schunk}
\begin{Soutput}
> 5:0
[1] 5 4 3 2 1 0	
\end{Soutput}
\end{Schunk}

콜론의 경우 수의 증감은 1로 고정됩니다.

\paragraph{rep() 함수의 사용} 

수열을 생성하는 또 다른 방법은 rep() 함수를 활용하는 것인데, 이는 replicates (반복)이라는 의미입니다.
따라서, 1이라는 숫자를 5 번 반복하고자 한다면 아래와 같이 합니다. 

\begin{Schunk}
\begin{Soutput}
> rep(1,5)
[1] 1 1 1 1 1
\end{Soutput}
\end{Schunk}

이제 살짝 응용해봅니다.  
먼저 1,2,3 이라는 수열을 만든뒤 이 수열자체를 세번 반복하고자 한다면 아래와 같이 할 수 있습니다. 

\begin{Schunk}
\begin{Soutput}
> rep(1:3, 3)
[1] 1 2 3 1 2 3 1 2 3	
\end{Soutput}
\end{Schunk}
>rep(1:3, each=3)
[1] 1 1 1 2 2 2 3 3 3

조금 더 응용해 봅니다. 
0을 1번 반복하고, 1을 2번 반복하고, 2를 3번 반복한 수열을 생성해 봅니다. 

\begin{Schunk}
\begin{Soutput}
> c(rep(0,1), rep(1,2), rep(2,3))
[1] 0 1 1 2 2 2
\end{Soutput}
\end{Schunk}

이제 c(), seq(), 그리고 rep()를 모두 사용하여 벡터를 생성해 보겠습니다.

\begin{Schunk}
\begin{Soutput}
> a <- c(1, 2, 3, 4)
> b <- seq(1,4)
> c <- 1:4
>
> a
[1] 1 2 3 4
> b
[1] 1 2 3 4
> c
[1] 1 2 3 4
>
> d <- rep(seq(from=1, to=4), 3)
> d
[1] 1 2 3 4 1 2 3 4 1 2 3 4
>
> e <- rep(a, each=3)
> e
[1] 1 1 1 2 2 2 3 3 3 4 4 4
>
\end{Soutput}
\end{Schunk}

rep()에서 each의 사용에 따른 결과 차이를 발견하시기 바랍니다.

\section{행렬연산과 2차원}

여기까지 우리는 벡터라는 1차원 수열에 대해서 이야기를 했습니다. 
그러나, 지금부터는 행과 열이라는 개념을 이용하여 1차원에서 2차원의 숫자들의 배치에 대해서 이야기 할 것입니다. 
이러한 숫자들의 나열을 행렬(matrix)이라고 하며, R에서는 아래와 같은 다양한 연산기능을 제공하고 있습니다. 

먼저 행렬 연산에 대한 이해를 돕기 위해서 아래와 같은 간단한 행렬을 생각해 봅니다. 

\begin{equation}
A = 
\begin{bmatrix}
1 & 4 \\
2 & 5 \\
3 & 6
\end{bmatrix}
\end{equation}

여기에서 행렬의 구성요소는 $1,2,3,4, 5, 6$ 이며, 크기는 3행 2열입니다.

\paragraph{행렬의 생성:} 위에서 수학적으로 표현된 A 라는 행렬을 R에 입력하기 위해서는 아래와 같이 합니다. 

\begin{Schunk}
\begin{Soutput}
> M <- matrix(c(1,2,3,4, 5, 6), ncol=2)
> M
     [,1] [,2]
[1,]    1    4
[2,]    2    5
[3,]    3    6
>
\end{Soutput}
\end{Schunk}

그러나, 사용자는 때때로 행렬의 요소들을 행방향으로 나열하고자 할 수도 있습니다. 
\begin{Schunk}
\begin{Soutput}
> M1 <- matrix(c(1,2,3,4, 5, 6), ncol=2, byrow=TRUE)
> M1
     [,1] [,2]
[1,]    1    2
[2,]    3    4
[3,]    5    6

\end{Soutput}
\end{Schunk}

ncol은 열의 수를 지정해 주는 역할을 하는 것이고 byrow가 수가 입력되는 순서를 결정합니다.

그런데, 여기에서 반드시 알고 넘어가야 할 부분이 바로 벡터와 행렬간의 관계입니다. 
행렬을 생성하기 위해서는 첫번째 인자에 벡터를 사용합니다. 
이는 바로 R이 행렬을 어떻게 처리해 주는지 알게 되는 부분입니다. 

먼저, 컴퓨터는 벡터와 행렬이 무엇인지 구분을 하지 못합니다. 
단순히 숫자의 나열이라고만 알고 있습니다. 
따라서, 위에서 \texttt{c(1,2,3,4, 5, 6)}라는 벡터를 행렬임을 알게 해주기 위해서는 행과 열이라는 속성을 부여함으로서 
R이 행렬임을 알게 하는 것입니다.
다음과 같이 \texttt{attributes()}라는 함수를 이용하여 M이라는 행렬이 어떤 속성을 가지고 있는지 확인해 봅니다. 

\begin{Schunk}
\begin{Soutput}
> attributes(M)
$dim
[1] 3 2
\end{Soutput}
\end{Schunk}
% $

행과 열이라는 속성은 행렬을 생성시 인자명을 통해서 생성된 것이므로, attributes()로 확인했을 때 이들의 정보가 dim 에 저장되어 있음을 알 수 있습니다. 
그런데, 행과 열의 속성은 행렬의 크기를 나타내기도 하기 때문에 이들의 값을 따로 빼내어 사용하고 싶다면 dim() 이라는 함수를 이용하면 됩니다. 

\begin{Schunk}
\begin{Soutput}
> dim(M)
[1] 3 2
>
\end{Soutput}
\end{Schunk}

첫번째 요소는 행의 개수이고, 두번째 요소는 열의 개수입니다. 

또다른 예제는 아래와 같습니다. 

\begin{Schunk}
\begin{Soutput}

> M2 <- matrix(1:15, ncol=5, nrow=3)
> M2
     [,1] [,2] [,3] [,4] [,5]
[1,]    1    4    7   10   13
[2,]    2    5    8   11   14
[3,]    3    6    9   12   15
> attributes(M2)
$dim
[1] 3 5
\end{Soutput}
\end{Schunk}
% $

이제 행렬을 생성하였으니, 간단한 행렬의 연산에 대해서 살펴보도록 하겠습니다. 

\paragraph{행렬의 전치: }
2행 2열인 정방행렬을 열방향으로 나열하는 것을 행방향으로 나열하게 된다면, 이는 행렬의 전치를 의미하게 됩니다. 
따라서, 행렬의 전치를 수행했을대 동일한 결과를 가지게 될 것입니다. 

\begin{Schunk}
\begin{Soutput}
> M <- matrix(1:4, ncol=2)
> M
     [,1] [,2]
[1,]    1    3
[2,]    2    4
> t(M)
     [,1] [,2]
[1,]    1    2
[2,]    3    4
> 
\end{Soutput}
\end{Schunk}

t()는 전치(transpose)의 약어입니다.

\paragraph{행렬을 다시 벡터로 전환하기:}
어떤 경우는 다시 행렬을 벡터의 형식으로 불러와야 할 경우도 있습니다. 

\begin{Schunk}
\begin{Soutput}
> c(M)
[1] 1 2 3 4
> 
\end{Soutput}
\end{Schunk}


\paragraph{대각행렬 기능활용: } 행렬 A의 대각행렬의 원소들은 $1$과 $4$인데, 이를 얻는 방법은 아래와 같이 \texttt{diag()} 함수를 사용하는 것입니다. 

\begin{Schunk}
\begin{Soutput}
> diag(M)
[1] 1 4
\end{Soutput}
\end{Schunk}

만약, 대각행렬의 원소가 \texttt{c(3,4)}를 가지는 정방대각행렬을 생성하고자 한다면 아래와 같이 사용할 수 있습니다 .

\begin{Schunk}
\begin{Soutput}
> diag(c(3,4))
     [,1] [,2]
[1,]    3    0
[2,]    0    4
> 
\end{Soutput}
\end{Schunk}

또한, I 행렬을 생성하는데 사용할 수 있습니다. 

\begin{Schunk}
\begin{Soutput}
> diag(2)
     [,1] [,2]
[1,]    1    0
[2,]    0    1
>
\end{Soutput}
\end{Schunk}

본 챕터에서의 목적은 R의 기초적 사용에 익숙해지는 것이므로, 아래의 주제에 대해서는 ``수학/확률/수치해석'' 이라는 챕터를 살펴보아 주시길 바랍니다.
\begin{itemize}
	\item 행렬값 (determinant) 구하기 
	\item 역행렬 (inverse matri) 구하기 
	\item 역행렬을 이용하여 선형 연립방정식의 해를 구하기
	\item 크로넥터 프로덕트 (Kronecker product) 계산하기 
	\item 행렬의 분해 1 - singular value decomposition 
	\item 행렬의 분해 2 - cholesky decomposition 
	\item 행렬의 분해 3 - spectral decomposition
	\item 고유값 (eigen value)와 고유벡터 (eigen vector) 구하기  
\end{itemize}

여기까지 1개의 행렬을 이용한 기초 행렬 연산을 살펴보았습니다. 

\paragraph{행렬의 사칙연산} 이제 2개의 이상의 행렬을 이용한 사칙연산을 살펴봅니다.

\begin{Schunk}
\begin{Soutput}
> A <- matrix(c(1,2,3,4), ncol=2)
> B <- matrix(c(5,6,7,8), ncol=2)

# 행렬의 덧셈
> A + B 
     [,1] [,2]
[1,]    6   10
[2,]    8   12

# 행렬의 뺄셈
> A - B
     [,1] [,2]
[1,]   -4   -4
[2,]   -4   -4

# 행렬의 곱셈  
> A %*% B 
     [,1] [,2]
[1,]   23   31
[2,]   34   46

# 행렬의 구성요소 단위의 곱셈 
> A * B
     [,1] [,2]
[1,]    5   21
[2,]   12   32
> 

# 나머지 구하기
> B %% A
     [,1] [,2]
[1,]    0    1
[2,]    0    0

# 나누기 
> B / A
     [,1]     [,2]
[1,]    5 2.333333
[2,]    3 2.000000
 
\end{Soutput}
\end{Schunk}

위에서 살펴본 바와 같이 행렬은 모두 구성요소 단위의 연산을 수행하게 됩니다. 

\paragraph{행렬의 결합}  간혹 두 개 이상의 행렬들을 결합할 경우가 있습니다.
이런 경우에는 행방향 혹은 열방향의 결합이 아래와 같은 방법으로 가능합니다.

만약, 아래와 같이 A와 B 라는 행렬이 존재한다면, 

\begin{Schunk}
\begin{Soutput}
> A
     [,1] [,2]
[1,]    1    3
[2,]    2    4
> B
     [,1] [,2]
[1,]    5    7
[2,]    6    8
\end{Soutput}
\end{Schunk}

이 두개의 행렬을 행방향으로 결합하고자 한다면 rbind()라는 함수를 이용합니다.

\begin{Schunk}
\begin{Soutput}
> R <- rbind(A, B)
> R
     [,1] [,2]
[1,]    1    3
[2,]    2    4
[3,]    5    7
[4,]    6    8
\end{Soutput}
\end{Schunk}

열방향으로 결합을 하고자 한다면 cbind()함수를 이용합니다. 

\begin{Schunk}
\begin{Soutput}
> C <- cbind(A, B)
> C
     [,1] [,2] [,3] [,4]
[1,]    1    3    5    7
[2,]    2    4    6    8
> 
\end{Soutput}
\end{Schunk}



\section{배열과 다차원} 

여기까지 벡터와 행렬이라는 1차원과 2차원에 숫자들을 나열하는 방법을 살펴보았습니다.
이제 3차원에 숫자들을 나열하려면 어떻게 해야할까요? 
보다 더 나아가서 n 차원에 숫자를 나열하는 것은 어떻게 할까요?
이러한 방법을 제공하는 것이 바로 배열입니다. 

배열의 생성원리는 배열의 원리와 동일하게 벡터의 값을 입력한 뒤 차원이라는 속성을 이용하여 표현하는 것입니다.
아래의 예에서 8개의 원소로 구성된 a 라는 벡터를 각 차원이 2행 2열인 행렬을 가지는 2개의 차원으로 표시한 것입니다. 

\begin{Schunk}
\begin{Soutput}
> a <- 1:8
> A <- array(a, dim=c(2,2,2))
> A 
, , 1

     [,1] [,2]
[1,]    1    3
[2,]    2    4

, , 2

     [,1] [,2]
[1,]    5    7
[2,]    6    8
>
\end{Soutput}
\end{Schunk}

여기에서 dim=c(2,2,2)라고 입력된 차원정보는 (열의개수, 행의개수, 차원의 개수) 라는 형식을 가지고 있습니다. 
그런데 다음의 예제와 같이 만약 1에서 20까지의 정수를 3차원 배열에 입력할 때 차원과 열의 순서에 따라 차례대로 숫자가 입력되고 나머지 공간에는 다시 1부터 시작하여 숫자가 채워지는 것에 주의해야 합니다.

% 근데 비워두고 싶으면 어케 하지?
% (답변) 비워두는 방식은 없어. 그러나, NA 로 채워둘 수는 있어. 

\begin{Schunk}
\begin{Soutput}

> a <- c(1:20)
> A <- array(a, dim=c(3, 3, 3))
> A
, , 1

     [,1] [,2] [,3]
[1,]    1    4    7
[2,]    2    5    8
[3,]    3    6    9

, , 2

     [,1] [,2] [,3]
[1,]   10   13   16
[2,]   11   14   17
[3,]   12   15   18

, , 3

     [,1] [,2] [,3]
[1,]   19    2    5
[2,]   20    3    6
[3,]    1    4    7
\end{Soutput}
\end{Schunk}

\section{인덱싱과 라벨링}

여기까지는 아주 작은 크기의 벡터, 행렬, 그리고 배열을 예를 들어 설명했습니다.
그러나, 실제적으로는 매우 다양한 크기의 벡터, 행렬, 그리고 배열을 다루게 될 것이며, 이들을 구성하는 요소들중 특수한 구성요소들만을 사용해야 할 경우도 있을 것입니다.  

\subsection{인덱싱 하기}

위에서 언급한 바와 같이 벡터, 행렬, 배열은 모두 벡터를 기초로 하되, 그 속성의 지정만이 다른것입니다.
그리고, 속성에 따라 구성원소들이 배치가 될때 이들은 기본적으로 열방향으로 나열됩니다. 
왜 기본적으로 열방향으로 나열하는가에 대한 답변은 일반적인 통계처리는 변수를 중심으로 이루어지고, 이 변수는 일반적인 데이터 나열시 열방향으로 존재하기 때문입니다. 
즉, 개별 관측치는 여러개의 변수들을 1행에 나열하는 반면, 각 변수는 여러개들의 관측치를 1개의 열에 넣기 때문입니다.

\paragraph{벡터기반 인덱싱} 벡터의 구성요소를 선택하는 것을 인덱싱이라고 합니다. 
이 인덱싱은 구성요소가 주어진 벡터안에 몇 번째에 놓여있는가에 대한 위치에 대한 정보입니다.
그리고, 이렇게 파악된 위치 정보를 이용하여 구성요소를 선택한 값을 보여주는 것은 열린 중괄호 `['와 닫힌 중괄호 `]'를 이용하여 표시합니다. 
아래의 예를 살펴봅니다. 

먼저 101, 102, 103, 104, 105, 106, 107, 108, 109, 110, 111, 112 의 수열을 가진 x 라는 벡터를 생각합니다. 
 
\begin{Schunk}
\begin{Soutput}
> x <- 101:112
> x
 [1] 101 102 103 104 105 106 107 108 109 110 111 112
\end{Soutput}
\end{Schunk}

이제 x 벡터의 세번째 값을 알고 싶습니다. 

\begin{Schunk}
\begin{Soutput}
> x[3]
[1] 103
\end{Soutput}
\end{Schunk}

여기에서 x의 세번째 라는 것이 위치정보라는 인덱스이고, 세번째에 해당하는 x의 값인 103을 불러오는 것은 x 벡터를 인덱싱했다고 합니다. 
인덱싱은 벡터형식의 인덱스를 활용할 수도 있습니다.
만약, 3번째부터 7번째까지의 x의 값들을 가져오고 싶다면 아래와 같이 합니다. 

\begin{Schunk}
\begin{Soutput}
> x[3:7]
[1] 103 104 105 106 107
\end{Soutput}
\end{Schunk}

조금 더 자유롭게 인덱싱을 해보겠습니다. 
x의 홀수번째에 해당하는 값들만 뽑아봅니다. 

\begin{Schunk}
\begin{Soutput}
> x[c(1,3,5,7,9,11)]
[1] 101 103 105 107 109 111
\end{Soutput}
\end{Schunk}

위에서 보는 것과 같이 R은 벡터식 연산이라는 특징을 이용하여 활용할 수 있습니다.

\paragraph{벡터의 라벨을 이용한 인덱싱} 
인덱스를 이용하여 벡터의 구성요소를 인덱싱할 수 있으나, 벡터 구성요소 자체에 이름을 붙여 이들의 값을 불러올 수 도 있습니다. 
먼저 x 벡터의 구성요소들에 이름을 붙이기 위해서 names()라는 함수를 사용합니다. 

\begin{Schunk}
\begin{Soutput}
> names(x) <- c("a", "b", "c", "d", "e", "f", "g", "h", "i", "j", "k", "l")
> x
  a   b   c   d   e   f   g   h   i   j   k   l 
101 102 103 104 105 106 107 108 109 110 111 112 
\end{Soutput}
\end{Schunk}

이제 "a"와 "i" 라는 이름에 해당하는 x의 값을 불러옵니다. 

\begin{Schunk}
\begin{Soutput}
> x[c("a", "i")]
  a   i 
101 109 
\end{Soutput}
\end{Schunk}

\paragraph{행렬의 인덱싱} 
위에서는 1차원 벡터의 경우를 살펴보았으므로, 이제 2차원 행렬의 인덱싱을 살펴 봅니다. 
먼저, xm이라는 4행 3열인 행렬을 생각해 봅니다. 

\begin{Schunk}
\begin{Soutput}
> x <- 101:112
> xm <- matrix(x, ncol=3)
> xm
     [,1] [,2] [,3]
[1,]  101  105  109
[2,]  102  106  110
[3,]  103  107  111
[4,]  104  108  112
> 
\end{Soutput}
\end{Schunk}

위에서 설명한대로 행렬이 벡터라는 것을 이해하기 위해서는 아래를 살펴봅니다. 
먼저 xm 이라는 행렬의 2행 3열에 위치한 값은 110 임을 알 수 있습니다. 
만약 열방향 벡터대로 이를 처리한다면 110 이라는 값은 이 행렬이 가진 벡터의 위치가 10번째일 것입니다.

\begin{Schunk}
\begin{Soutput}
> xm[2,3]
[1] 110
> xm[10]
[1] 110
\end{Soutput}
\end{Schunk}

이제, 하나의 구성요소말고 본 행렬의 부분집합에 대해서 알아봅니다. 
먼저, 행렬 xm의 3번째 열에 해당하는 값들만 뽑습니다. 

\begin{Schunk}
\begin{Soutput}
> xm[,3]
[1] 109 110 111 112
\end{Soutput}
\end{Schunk}

이제는 2번째 행에 해당하는 값들만 뽑습니다. 

\begin{Schunk}
\begin{Soutput}
> xm[2,]
[1] 102 106 110
\end{Soutput}
\end{Schunk}

좀 더 나아가서, 1번째와 2번째 행에 해당하되 3번째 열에 해당하는 구성요소만 뽑고 싶습니다. 
\begin{Schunk}
\begin{Soutput}
> xm[1:2, 3]
[1] 109 110
\end{Soutput}
\end{Schunk}

좀 더 확장해봅니다. 1과 2행에 있으며 2와 3열에 있는 구성요소를 뽑고 싶습니다.

\begin{Schunk}
\begin{Soutput}
> xm[1:2, 2:3]
     [,1] [,2]
[1,]  105  109
[2,]  106  110
\end{Soutput}
\end{Schunk}

한 번 더 연습합니다.

\begin{Schunk}
\begin{Soutput}
> xm[c(1,4), c(1,3)]
     [,1] [,2]
[1,]  101  109
[2,]  104  112
> 
\end{Soutput}
\end{Schunk}

이제 이러한 구성요소들을 행렬의 행과 열의 이름을 활용해 보도록 합니다. 
이를 위해서 먼저, 행렬의 행과 열에 각각 이름을 붙여줘야 합니다. 
함수 \texttt{rownames()}와 \texttt{colnames()}는 이를 가능하게 해줍니다.

\begin{Schunk}
\begin{Soutput}
> rownames(xm) <- c("R1", "R2", "R3", "R4")
> colnames(xm) <- c("C1", "C2", "C3")
> xm
    C1  C2  C3
R1 101 105 109
R2 102 106 110
R3 103 107 111
R4 104 108 112
\end{Soutput}
\end{Schunk}

이제 2번째와 3번재 열에 해당하며 2번째와 4번째 행에 해당하는 구성요소를 행과 열의 이름을 이용하여 선택해 봅니다. 

\begin{Schunk}
\begin{Soutput}
> xm[c("R2", "R4"), c("C2", "C3")]
    C2  C3
R2 106 110
R4 108 112
\end{Soutput}
\end{Schunk}


\paragraph{배열의 인덱싱} 이제 배열의 경우를 살펴보도록 하겠습니다. 
먼저 2행 2열이 3차원이 놓은 구성요소를 가진 배열 A 을 생성합니다.

\begin{Schunk}
\begin{Soutput}
> a <- 101:112
> A <- array(a, dim=c(2,2,3))
> A
, , 1

     [,1] [,2]
[1,]  101  103
[2,]  102  104

, , 2

     [,1] [,2]
[1,]  105  107
[2,]  106  108

, , 3

     [,1] [,2]
[1,]  109  111
[2,]  110  112

\end{Soutput}
\end{Schunk}

이 배열을 잘 살펴보면 각 차원 내의 행렬이 모두 열의 방향으로 구성요소들의 값이 나열됨을 알 수 있습니다. 
따라서 7번째의 값은 2차원내의 1행 2열에 존재할 것입니다. 

\begin{Schunk}
\begin{Soutput}
> A[7]
[1] 107
> A[1,2,2]
[1] 107
\end{Soutput}
\end{Schunk}


이제부터 2차원에 있는 값들만 뽑아봅니다. 

\begin{Schunk}
\begin{Soutput}
> A[,,2]
     [,1] [,2]
[1,]  105  107
[2,]  106  108
\end{Soutput}
\end{Schunk}


그럼 차원에 관계없이 1행에 해당하는 값들만 뽑아봅니다.
\begin{Schunk}
\begin{Soutput}
> A[1,,]
     [,1] [,2] [,3]
[1,]  101  105  109
[2,]  103  107  111
\end{Soutput}
\end{Schunk}


마지막으로 차원에 관계없이 2행 1열에 해당하는 값들만 뽑아보겠습니다. 

\begin{Schunk}
\begin{Soutput}
> A[2,1,]
[1] 102 106 110
> 
\end{Soutput}
\end{Schunk}


\section{숫자말고 문자형, 그리고 모드}

여기까지 우리는 기초사용에 익숙해지기 위해서 숫자값들만을 이용하여 벡터, 행렬, 그리고 배열의 기초적인 조작법을 살펴보았습니다. 
R은 통계분석에 필요한 프로그래밍을 위한 보다 다양한 형태의 데이터 형식을 가지고 있습니다.
이들은 요인(factor), 데이터 프레임(data frame), 리스트(list) 라는 것이 있는데 이들에 대해서는 ``데이터형과 조작'' 이라는 챕터를 살펴보시길 바랍니다. 

여기에서는 ``데이터형과 조작'' 이라는 챕터에서 다루어질 데이터들의 종류에 대해서 한가지만 더 알고 넘어가도록 하겠습니다.
그것은 바로 문자형입니다. 
문자형이란 1,2,3,...과 같은 숫자들이 아닌 a,b,c, ... 와 같은 기호를 의미합니다.

R은 숫자와 문자라는 것을 사용자가 직접 알려주기 전까지는 알 수 없습니다. 
따라서, 문자형인 데이터를 사용할 때에는 꼭 큰따옴표를 이용하여 "a", "b", "c", ... 와 같은 형식으로 사용해야 합니다. 

문자형 구성요소 "a" "b" "c" "A" "B" "C"로 이루어진 벡터를 한 번 만들어 봅니다.

\begin{Schunk}
\begin{Soutput}
> a <- c("a", "b", "c", "A", "B", "C")
> a
[1] "a" "b" "c" "A" "B" "C"
\end{Soutput}
\end{Schunk}

이렇게 벡터로 잘 입력이 되었다면 이들의 값을 얻는 방법은 숫자형으로 이루어진 벡터에서 인덱싱하는 방법과 동일합니다. 
세번째 문자를 뽑으려면 아래와 같이 합니다. 

\begin{Schunk}
\begin{Soutput}
> a[3]
[1] "c"
\end{Soutput}
\end{Schunk}

간혹 대문자 "A"부터 "Z"까지 생성해보고 싶다면 아래와 같이 LETTERS라는 이미 주어진 벡터를 활용하세요.
이와 같이 이미 주어진 것들에 대한 것을 내장되어 있다고 합니다. 

% 내장 벡터 또는 데이터의 목록은 어떻게 확인하더라...

\begin{Schunk}
\begin{Soutput}
> a <- LETTERS
> a
 [1] "A" "B" "C" "D" "E" "F" "G" "H" "I" "J" "K" "L" "M" "N" "O" "P" "Q" "R" "S"
[20] "T" "U" "V" "W" "X" "Y" "Z"
> a[21:25]
[1] "U" "V" "W" "X" "Y"
> 
\end{Soutput}
\end{Schunk}

지금 본 섹션에서는 숫자형 데이터를 다루지 않고 문자형 데이터를 다룹니다. 
그럼 만약 어떤 객체가 주어졌을때 이 객체가 숫자형인지 문자형인지 어떻게 알 수가 있는 것일까요?
아래와 같이 is 계열의 함수를 통하여 확인할 수 있습니다. 

\begin{Schunk}
\begin{Soutput}
> is.numeric(a)
[1] FALSE

> is.numeric(x)
[1] TRUE

> is.character(a)
[1] TRUE
\end{Soutput}
\end{Schunk}

그럼, 실제로 이 데이터들은 어떻게 저장되는 것인가요? x 라는 것은 숫자형이므로 numeric 이고, a는 문자형이므로 character 라고 합니다. 
즉, 객체의 유형을 mode()라는 함수를 통하여 알 수 있습니다.  

\begin{Schunk}
\begin{Soutput}
> mode(x)
[1] "numeric"

> mode(a)
[1] "character"

\end{Soutput}
\end{Schunk}

문자형의 경우에는 그 유형은 character 라는 형식으로 저장할 수 있으나, 수치형일 경우에는 조금 더 세분화 됩니다. 
위에서 우리는 x라는 벡터가 1부터 8까지 이루어진 정수형 벡터를 활용했습니다.
따라서, x는 numeric 이라는 수치형을 가지고 있으며 정수라는 특별한 형식을 사용합니다. 
이를 확인할때 typeof() 라는 함수를 사용합니다. 

\begin{Schunk}
\begin{Soutput}
> x <- 1:8
> mode(x)
[1] "numeric"
> typeof(x)
[1] "integer"
\end{Soutput}
\end{Schunk}

이제 0부터 1사이에 0.1을 단위로 하는 수열을 생성해 본 뒤 이를 확인합니다. 

\begin{Schunk}
\begin{Soutput}
> x1 <- seq(from=0, to=1, by=0.1)
> x1
 [1] 0.0 0.1 0.2 0.3 0.4 0.5 0.6 0.7 0.8 0.9 1.0
> mode(x1)
[1] "numeric"
> typeof(x1)
[1] "double"
 
\end{Soutput}
\end{Schunk}

이렇게 R은 정수형과 부동소수형에 대한 형식을 수치형이라는 클래스 내에 가지고 있습니다. 
이러한 구분은 수치연산과 관련된 부분이므로 설명하지 않습니다. 

%R에서 인식하는 데이터형에 대한 설명은 어디에 넣지?


\section{결측값과 변수초기화} 

다음 챕터로 넘어가기전에 숫자형과 문자형 데이터를 다룰 때 반드시 알아두어야 할 결측치에 대한 설명을 한 뒤에 본 챕터를 마무리 합니다.

결측치라는 것은 아래와 같은 경우에 발생합니다. 

벡터를 생성하려고 하는데, 사용자가 무슨 값을 생성해야 할지 모릅니다. 
예를들어 1,2, 그리고... 값을 몰라서 입력을 못해서 머릿속에 ???? 인데, 그 다음의 값은 4 라는 것을 알고 있습니다. 
이런 경우에 R에 값을 입력하기 위해서는 아래와 같이 합니다. 

\begin{Schunk}
\begin{Soutput}
> x <- c(1,2,NA,4)
> x
[1]  1  2 NA  4
\end{Soutput}
\end{Schunk}

이처럼, R에서는 존재하지 않는 어떤 값 (즉, 결측치)를 NA 이라는 예약어를 이용하여 표시합니다. 
여기에서 NA란 Not Available 의 약자입니다. 

그렇다면 어떤 주어진 객체 x 라는 것이 있을 때 그 구성요소가 결측이라는 것을 알기 위해서는 is.na()이라는 함수를 사용하면 될 것입니다. 

\begin{Schunk}
\begin{Soutput}
> is.na(x)
[1] FALSE FALSE  TRUE FALSE
\end{Soutput}
\end{Schunk}

문자형의 경우에도 동일합니다. 

\begin{Schunk}
\begin{Soutput}
> a <- c("a", "b", NA, "d")
> a
[1] "a" "b" NA  "d"
> is.na(a)
[1] FALSE FALSE  TRUE FALSE
> 
\end{Soutput}
\end{Schunk}

결측값이 있더라도 입력한 값이 수치형이라면 is.numeric()은 TRUE를 반환합니다. 
이것은 입력한 값이 문자형이라도 마찬가지 입니다.

\begin{Schunk}
\begin{Soutput}
> is.numeric(x)
[1] TRUE

> is.character(a)
[1] TRUE
\end{Soutput}
\end{Schunk}

어떤 경우 (특히 수치연산시)에는 어떠한 연산의 결과를 저장해야 할 빈공간을 만들어야 할 경우가 있습니다.
이런 경우에는 상식적으로 어떤 값이 어떻게 입력될지 모르니까 NA로 만들어 넣어야겠군 이라고 생각할 수 있습니다. 
그런데, 이러한 경우 R에서는 수치형인지 문자형인지에 따라 해당 공간을 만들어 주는 기능이 있습니다. 

\begin{Schunk}
\begin{Soutput}
> x <- numeric(10)
> x
 [1] 0 0 0 0 0 0 0 0 0 0
> a <- character(10)
> a
 [1] "" "" "" "" "" "" "" "" "" ""
\end{Soutput}
\end{Schunk}



%%%%%%%%%%%%%%%%%%%%%%%%%%%%%%%%%%%%%%%%%%%%%%%%%%%%%%%%%%%%%%%%%%%%%%%%
%
% CHAPTER
%
%%%%%%%%%%%%%%%%%%%%%%%%%%%%%%%%%%%%%%%%%%%%%%%%%%%%%%%%%%%%%%%%%%%%%%%%


\chapter{기초 프로그래밍과 운영체제}

우리는 독자가 R을 과학적 분석을 위해 필요한 일련의 프로세스를 진행하는 프로그래밍 언어로서 이해하기를 권장합니다.
따라서, 이 챕터에서는 이러한 프로세스를 수행하는데 필요한 프로그래밍적 요소들에 대해서 다룰 것입니다. 


\section{에러 그리고 정확성}

우리는 이러한 내용들을 설명하기 전에 에러와 경고에 대한 내용을 먼저 다룰 것입니다. 
그 이유는 에러와 경고는 프로세스의 수행중 연구자가 원하는 방향과는 다른 방향으로 진행되고 있음을 알려주는 표시이기 때문입니다. 
또다른 이유는 분석을 위한 프로그래밍은 논리적인 절차를 어떻게 잘 구성하는가에 따라서 그 효율성과 프로그램의 가독성이 달라지기 때문입니다. 

먼저 아래와 같이 0이라는 값을 0으로 나누어 보는 간단한 예를 들어봅니다. 

\begin{Schunk}
\begin{Soutput}
> 0/0
[1] NaN
\end{Soutput}
\end{Schunk}

아하! 이런 경우에는 NaN (즉, Not a number 라는 약자)를 알려줍니다. 
그 이유는 수학적으로 0으로 어떠한 값도 나눌 수 없기 때문입니다. 
그리고, 이렇게 NaN 이라는 값이 포함된 어떠한 연산도 수행될 수 없기 때문에 NaN이라는 값을 출력하게 됩니다. 

\begin{Schunk}
\begin{Soutput}
> x <- 0/0
> x + 3
[1] NaN
\end{Soutput}
\end{Schunk}

수학적 정의하에 올바른 계산을 하고자 한다면 아래와 같이 해야 할 것입니다. 

\begin{Schunk}
\begin{Soutput}
> 0/1
[1] 0
> 
\end{Soutput}
\end{Schunk}

그런데, 사용자의 실수로 인하여 아래와 같이 3을 0으로 나누었다고 가정합니다. 
조금더 고급수학을 다루게 된다면 $n/0$ 이라는 것은 $\infty$ (무한값)이라는 것을 알고 계실 것입니다.
R은 이러한 연산을 기본값으로 하고 있습니다. 

\begin{Schunk}
\begin{Soutput}
> 3/0
[1] Inf
> pi/0
[1] Inf
\end{Soutput}
\end{Schunk}

이렇게 무한값을 포함하고 있는 어떠한 연산 역시 무한값을 돌려줍니다. 

\begin{Schunk}
\begin{Soutput}
> x <- 3/0
> x + 1.0
[1] Inf
\end{Soutput}
\end{Schunk}

따라서, 이러한 에러와 관계된 부분을 정확히 알고 R을 사용해야 합니다. 

\begin{Schunk}
\begin{Soutput}
> 1/0 + 1/0
[1] Inf
> 1/0 - 1/0
[1] NaN
\end{Soutput}
\end{Schunk}

NaN 이라는 약자가 이전에 설명한 NA와 혼돈스러울 수 있기 때문에 한 번 설명합니다. 
NaN은 숫자가 아닌 것을 말하는 것이고, NA는 존재하지 않는 값을 의미하는 것입니다. 

그럼, R은 어느정도로 정확한 수치를 돌려줄까요? 
아래와 같이 1을 어떠한 수로 나누어 봅니다. 

\begin{Schunk}
\begin{Soutput}
> 1/1
[1] 1
> 1/10
[1] 0.1
> 1/100
[1] 0.01
\end{Soutput}
\end{Schunk}

계산이 잘 되는것 같습니다. 
이제 좀 더 큰수를 사용해봅니다. 

\begin{Schunk}
\begin{Soutput}
> 1/1e300
[1] 1e-300
\end{Soutput}
\end{Schunk}

여기에서 1e300 이라는 표현은 10의 300자승이라는 사이언티픽 표현입니다. 

\begin{Schunk}
\begin{Soutput}
> 1/1e309
[1] 0
> 1/1e308
[1] 1e-308
\end{Soutput}
\end{Schunk}

아하! 1e309 이상의 숫자를 사용하면 이 연산은 0 으로 간주됨을 알 수 있습니다. 
따라서, 사용자는 이보다 큰 숫자를 사용할 때 주의를 기울여야 합니다. 
일반적으로는 R 은 아래와 같은 범위내에서 안정적인 연산을 하게 됩니다. 

\begin{Schunk}
\begin{Soutput}
> .Machine$double.xmin
[1] 2.225074e-308
> .Machine$double.xmax
[1] 1.797693e+308
\end{Soutput}
\end{Schunk}

그럼 아래의 경우에는 어떤 것일까요?

\begin{Schunk}
\begin{Soutput}
> log(0)
[1] -Inf
\end{Soutput}
\end{Schunk}

위의 연산은 수학적으로 log의 값이 0 일때 음의 무한값을 가지도록 정의되어 있기 때문에 올바른 연산입니다. 
그러나, 아래와 같이 log에 -1의 값을 넣으면 어떻게 될까요?

\begin{Schunk}
\begin{Soutput}
> log(-1)
[1] NaN
Warning message:
In log(-1) : NaNs produced
\end{Soutput}
\end{Schunk}

(본 메시지는 현 한국어 3.0.0 버전으로는 아래와 같으며, R-3.1.0에 더 정확한 메시지로 교정하겠습니다 -- 사용에 불편을 드려 죄송합니다).

\begin{Schunk}
\begin{Soutput}
> log(-1)
[1] NaN
경고 메세지가 손실되었습니다
In log(-1): NaNs가 생성되었습니다
\end{Soutput}
\end{Schunk}

R은 이렇게 수학적으로 정의되어 있지 않은 연산에 대해서 NaN을 표시하고 경고메시지를 보여줍니다. 
이러한 경고메시지가 왜 발생했는지 아는 것은 매우 중요합니다. 

만약, 어떠한 연산을 수행했을때 아주 많은 에러가 생겨서 이들을 확인해보고자 한다면 아래와 같이 warnings() 함수를 이용하시길 바랍니다. 

\begin{Schunk}
\begin{Soutput}
> warnings()
Warning message:
In log(-1) : NaNs produced
\end{Soutput}
\end{Schunk}

(본 메시지는 현 한국어 3.0.0 버전으로는 아래와 같으며, R-3.1.0에서 오탈자를 교정하겠습니다 -- 사용에 불편을 드려 죄송합니다).

\begin{Schunk}
\begin{Soutput}
> warnings()
경고메새지:
In log(-1): NaNs가 생성되었습니다
\end{Soutput}
\end{Schunk}


그렇다면, 이번에는 수학연산을 하는 함수에 문자를 입력해 보면 어떤 현상이 발생하나요? 
아래의 결과를 살펴보세요.

\begin{Schunk}
\begin{Soutput}
> a <- "a"
> a
[1] "a"
> log(a)
Error in log(a) : Non-numeric argument to mathematical function
> 
\end{Soutput}
\end{Schunk}

수학함수에 요구되어지는 숫자값이 입력되지 않다면서 연산자체를 거부합니다. 
이를 에러라고 합니다. 

여기에서 보았듯이 경고와 에러는 서로 다른 것입니다.
요약하면, 경고라는 것은 연산은 수행되지만 어떠한 수학적 정의에 맞지 않는 것이며, 에러라는 것은 발생하지 말아야할 일이 일어났다는 것을 의미합니다. 

이러한 경고와 에러를 다루는 것은 실제적인 프로그래밍을 할 때 매우 중요합니다. 
R은 이러한 것들을 다루기 위한 디버거라는 도구와 에러핸들링이라는 기능을 제공하고 있습니다.
디버거의 사용에 대해서는 현재 익숙해지기 위한 기초단계에 해당되지 않는 기술적 요소이기 때문에 이곳에서는 다루지 않고, 
``패키지 제작''이라는 챕터에서 자세히 설명하도록 하겠습니다. 
그러나, 에러핸들링 기능은 이곳 기초 프로그래잉에서 설명할 것입니다. 

본 섹션의 내용을 이해했다면 여러분은 이제 프로그래밍을 시작할 준비가 되었습니다. 

\section{프로그래밍과 함수}

우리가 중학교에서 수학시간에 함수라는 개념을 배울때 블랙박스에 비유하는 것은 잘 알고 계실 것입니다. 
이 블랙박스에 어떠한 입력을 넣어주면, 어떤 특정한 프로세스에 의하여 처리된 결과가 블랙박스의 출력으로 나오게 됩니다. 
예를들어 x 라는 것은 3.532 이라는 값을 가지고 있는데, 이를 반올림하는 경우를 생각해 봅니다.
R은 round()라는 함수를 제공합니다. 

\begin{Schunk}
\begin{Soutput}
> x <- 3.532
> round(x)
[1] 4
\end{Soutput}
\end{Schunk}

그런데, 내가 원하고자 했던 것은 실제로 소수점 두번째 자리에서 반올림하는 것입니다. 

\begin{Schunk}
\begin{Soutput}
> round(x, 1)
[1] 3.5
\end{Soutput}
\end{Schunk}

흠... 어쩌다 보니 소수점 세번째 자리에서 반올림 한 숫자를 쓰는 것이 필요하게 되어 아래와 같이 합니다.

\begin{Schunk}
\begin{Soutput}
> round(x, 2)
[1] 3.53
\end{Soutput}
\end{Schunk}

내가 수행하고자 하는 어떤 것을 round()라는 블랙박스와 이 블랙박스에 반올림 자리수라는 인자를 이용하여 조절할 수 있었습니다. 

여기에서 보는 것과 같이 함수를 작성하기 위해서는 입력, 목적, 출력이라는 세가지 요소가 필요합니다. 
R에서 제공하는 다양한 함수들은 실제로 이러한 방식으로 작성되어 있으며, 단지 사용자의 편의를 고려하여 라이브러리라고 불리는 함수들의 묶음을 제공하는 것입니다. 
즉, 반올림을 해주는 round(), 어떤 구간을 동일하게 나누어 주는 cut()이라는 함수, 작업디렉토리에 어떤 파일들이 있는가를 보여주는 ls()함수들이 이렇게 미리 작성되어 제공되어지는 내장함수들이라고 합니다.

그러나, 사용자의 목적에 따라서 내장함수만으로는 더 이상 작업을 할 수 없는 경우가 있습니다.
이럴때 사용자 자신이 원하는 함수를 작성하여 사용하는 블랙박스를 사용자정의 함수라고 합니다.
이렇게 사용자 정의 함수를 정의하는데 사용되는 문법은 아래와 같습니다. 

\begin{Schunk}
\begin{Soutput}
fn <- function(args){
	expressions
}
\end{Soutput}
\end{Schunk}

예를들어, 내가 만드는 함수의 목적이 주어진 두개의 숫자들을 이용하여 더하기를 수행한 뒤 그 결과를 돌려준다고 가정합니다.
R을 이용하여 아래와 같이 할 수 있는데, 내가 원하는 것은 plus()라는 사용자정의 함수를 만들어 보는 것입니다. 

\begin{Schunk}
\begin{Soutput}
> x <- 1
> y <- 3
> x + y
[1] 4
\end{Soutput}
\end{Schunk}

그렇다면 콘솔상에서 아래와 같이 입력합니다. 

\begin{Schunk}
\begin{Soutput}
> file.create("plus.R")
> file.edit("plus.R")
# 여기부터는 R이 아닌 R이 열어준 plus.R이라는 파일명을 가지는 외부 파일임.

plus <- function(x1, x2){
	y <- x1 + x2
	return(y)
}

# 저장후 종료 하면 다시 R로 돌아감.
> # 저장한 코드를 실행하기 위해서 아래와 같이 함.
> source("plus.R")
>
\end{Soutput}
\end{Schunk}

% 이렇게 작성된 것을 R콘솔에 붙여 넣습니다.  
% (이렇게 외부에서 작성된 R 프로그램을 실행시키는 법을 조금뒤에 설명할 것입니다).

% 이미지가 없으면 오히려 혼란을 줄 수 있으므로 일단 주석처리함. 이 의견에 동의하지 않으면 맘대로 수정해도 됨
% \begin{Schunk}
% \begin{Soutput}
% > plus <- function(x1, x2){
% + y <- x1 + x2
% + return(y)
% + }
%
%
%


이제 R 콘솔에서 내가 만든 plus()라는 함수를 사용해 봅니다. 

\begin{Schunk}
\begin{Soutput}
> plus(3,2)
[1] 5
> plus(10,30)
[1] 40
\end{Soutput}
\end{Schunk}

아하! 잘 됩니다. 
그럼 아래와 같이 따라해봅니다. 
\begin{Schunk}
\begin{Soutput}
> plus(x1=3, x2=5)
[1] 8
\end{Soutput}
\end{Schunk}

이렇게 사용자가 함수의 정의부분에 사용한 인자를 보통 formal argument (형식 인자)라고 합니다. 
그리고 위와 같이 정의된 인자의 이름을 함께 이용하여 사용할 수 있습니다. 
이 때 이렇게 사용하는 인자를  named argument (지시된 인자)와 함께 사용한다고 합니다. 
지시된 인자를 사용하는 이유는 함수의 정의에 사용되는 인자의 개수는 매우 많을 수도 있기 때문에 사용의 혼돈을 피하기 위해서 입니다. 
따라서, 우리는 어떤 함수를 사용할 때 지시된 인자명과 함께 사용하기를 권장합니다. 

그런데, 이 함수는 정의할 때 두개의 인자를 필요로 했습니다. 
따라서, 아래와 같이 하나만 사용이 되거나 아무것도 입력되지 않는다는 아래와 같은 메시지를 보여주게 됩니다. 

\begin{Schunk}
\begin{Soutput}
> plus(x1=3)
Error in x1 + x2 : 'x2' is missing
> plus()
Error in x1 + x2 : 'x1' is missing
\end{Soutput}
\end{Schunk}

이렇게 어떤 함수에서 사용되는 인자들의 목록을 확인해 보고 싶다면 arg() 함수를 이용하길 바랍니다. 

\begin{Schunk}
\begin{Soutput}
> args(plus)
function (x1, x2) 
NULL

\end{Soutput}
\end{Schunk}

만약 아래와 같이 함수의 이름만 입력하게 된다면 아래와 같이 작성된 사용자함수의 소스코드를 볼 수 있습니다. 

\begin{Schunk}
\begin{Soutput}
> plus
function(x1, x2){
y <- x1 + x2
return(y)
}
> 
\end{Soutput}
\end{Schunk}

따라서, 사용자가 R의 소스코드가 어떻게 작성되어 있는지 확인하고 싶다면 사용되는 함수명을 입력하면 됩니다. 
예를들어, 현재 세션의 있는 R의 객체를 삭제해주는 명령어인 rm() 이라는 함수의 소스코드를 확인해 보고 싶다면 아래와 같이 단순히 rm이라고 입력합니다.

\begin{Schunk}
\begin{Soutput}
> rm
function (..., list = character(), pos = -1, envir = as.environment(pos), 
    inherits = FALSE) 
{
    dots <- match.call(expand.dots = FALSE)$...
    if (length(dots) && !all(sapply(dots, function(x) is.symbol(x) || 
        is.character(x)))) 
        stop("... must contain names or character strings")
    names <- sapply(dots, as.character)
    if (length(names) == 0L) 
        names <- character()
    list <- .Primitive("c")(list, names)
    .Internal(remove(list, envir, inherits))
}
<bytecode: 0x9fd5864>
<environment: namespace:base>
> 
\end{Soutput}
\end{Schunk}
% $

이렇게 소스를 볼 수 있는 기능은 추후에 전산통계를 배우고자 하는 학생 또는 소프트웨어 개발자들에게 매우 도움이 될 것입니다. 

\subparagraph{참고사항}
어떤 함수의 소스코드를 알아보고자 kk, mean, sapply를 각각 입력했다고 한다면 각각의 메세지가 다르게 나타나는 것을 확인할 수 있습니다.
그러나, 실제로는 객체의 유형이 무엇인지 잘 알아야 합니다. 
현재 kk 라는 것은 한 번도 사용한 적이 없는 객체의 이름이기 때문에 아래와 같이 나옵니다. 

\begin{Schunk}
\begin{Soutput}
> kk
에러: 객체 'kk'를 찾을 수 없습니다
\end{Soutput}
\end{Schunk}

그러나, sapply()와 같이 R에서 미리 제공하는 내장함수의 경우에는 이미 함수가 정의되어 있는 것이기 때문에 아래와 같이 나옵니다. 

\begin{Schunk}
\begin{Soutput}
> sapply
function (X, FUN, ..., simplify = TRUE, USE.NAMES = TRUE) 
{
   FUN <- match.fun(FUN)
   answer <- lapply(X = X, FUN = FUN, ...)
   if (USE.NAMES && is.character(X) && is.null(names(answer))) 
       names(answer) <- X
   if (!identical(simplify, FALSE) && length(answer)) 
       simplify2array(answer, higher = (simplify == "array"))
   else answer
}
<bytecode: 0x00000000147a3290>
<environment: namespace:base>
\end{Soutput}
\end{Schunk}

% \begin{Schunk}
% \begin{Soutput}
% > mean
% function (x, ...) 
% UseMethod("mean")
% <bytecode: 0x0000000021e6edc8>
% <environment: namespace:base>
% \end{Soutput}
% \end{Schunk}


본 섹션에서는 아주 단순한 예제를 이용하여 함수의 개념과 사용자 정의함수를 작성하는 방법을 설명했습니다.  
이렇게 단위의 함수들이 모이고 모여서 어떤 하나의 큰 작업을 수행하도록 할 수 있습니다. 
그리고, 몇 개의 블랙박스가 서로 어떻게 연결 및  구성되며, 형상관리는 어떻게 할 것인지, 사용자의 편의를 고려해야 하는지, 얼마나 프로그램을 자잘하게 쪼개야 하는 등의 다양한 관점을 살리는 부분은 프로그램을 하면서 경험속에서 얻어지게 됩니다.
여기에서는 소프트웨어의 개발을 이야기 하는 것이 아니기 때문에 단순히 프로그램이라는 것을 어떤 분석을 수행하기 위한 일련의 과정을 컴퓨터가 이해할 수 있도록 하는 것으로 한정하도록 합니다.

\section{조건문과 논리연산자}

\paragraph{if 그리고 else} 
우리는 어떤 프로세스를 설계할 때 이런 경우에는 A라는 것을 수행하고, 저런 경우에는 B 라는 것을 수행한다라고 하는 로직을 사용합니다. 
R은 이러한 논리를 표현하기 위해서 if와 else 라는 ``만약 그렇다면'' 이것을 수행하고 ``그렇지 않다면'' 이것을 수행한다라는 조건문을 제공하고 있습니다.  
이것에 대한 문법적 형식은 아래와 같습니다. 

\begin{Schunk}
\begin{Soutput}
if(condition){
	expressions
}
else {
	expressions
}
\end{Soutput}
\end{Schunk}

이러한 조건문을 이해하기 위해서 위에서 사용한 plus()라는 사용자 정의 함수를 아래와 같이 확장해 보도록 합니다. 

\begin{itemize}
	\item 만약 x1에 입력받은 숫자가 10 보다 크다면 "x1 is greater than 10" (x1은 10보다 큽니다)라는 메시지를 함수의 결과값을 보여주기 전에 출력하고, 
	\item 그렇지 않다면 "x1 is less than or equal to 10" (x1은 10보다 작거나 같습니다) 라는 메시지를 함수의 결과값을 보여주기 전에 출력하도록 합니다.
\end{itemize}

이를 표현하자면, 아래와 같습니다. 

\begin{Schunk}
\begin{Soutput}
plus <- function(x1, x2){
	if(x1 <= 10) print("x1 is less than or equal to 10")
	if(x1 > 10) print("x1 is greater than 10")
	y <- x1 + x2
	return(y)
}
\end{Soutput}
\end{Schunk}

조금 바꿔봅니다. 

\begin{Schunk}
\begin{Soutput}
plus <- function(x1, x2){
	if(x1 <= 10) print("x1 is less than or equal to 10")
	else print("x1 is greater than 10")
	y <- x1 + x2
	return(y)
}
\end{Soutput}
\end{Schunk}

그 결과는 아래와 같습니다. 

\begin{Schunk}
\begin{Soutput}
> plus(3,5)
[1] "x1 is less than or equal to 10"
[1] 8
> plus(15, 4)
[1] "x1 is greater than 10"
[1] 19
> 
\end{Soutput}
\end{Schunk}

프로그램이 잘 수행됩니다. 

\paragraph{논리연산}
이제 다시 위의 문법을 살펴보도록 합니다. 
조건문 if 의 사용에 꼭 필요한 것은 조건문에 대한 판단입니다.
문법적 요소인  condition 이라는 것은 어떤 주어진 조건에 대한 참과 거짓을 판단하는 부분입니다.

위의 plus 함수에서 이 condition 에 해당하는 부분이 바로 $x1 < 10$ 이라는 표현입니다. 
그럼, 이 조건문은 어떻게 수행되는지 알기 위해서 $x1 < 10$의 결과를 출력하도록 print($x1<10$)으로 변경합니다.
 
\begin{Schunk}
\begin{Soutput}
plus <- function(x1, x2){
	if(print(x1 <= 10)) print("x1 is less than or equal to 10")
	else print("x1 is greater than 10")
	y <- x1 + x2
	return(y)
}
\end{Soutput}
\end{Schunk}

이를 확인해 보기 위해서 실행해 봅니다. 
\begin{Schunk}
\begin{Soutput}
> plus(3,5)
[1] TRUE
[1] "x1 is less than or equal to 10"
[1] 8
> 
\end{Soutput}
\end{Schunk}

아하! R은 이렇게 condition 부분에서 참과 거짓을 판단한 후 그 결과로서 돌려주는 TRUE 또는 FALSE 라는 값에 의해서 조건문을 실행하게 된다는 것을 알 수 있습니다. 
이러한 조건에 대한 수행적 결과를 우리는 논리적 연산을 수행한다고 합니다. 

아래의 예를 살펴봅니다.
먼저 1부터 10사이에 1단위로의 수열을 생성합니다.

\begin{Schunk}
\begin{Soutput}
> x <- 1:10
> x
 [1]  1  2  3  4  5  6  7  8  9 10
\end{Soutput}
\end{Schunk}

이제 벡터 x의 구성요소중 5 보다 큰 요소들에 대한 결정을 내리도록 하겠습니다. 

\begin{Schunk}
\begin{Soutput}
> x > 5
 [1] FALSE FALSE FALSE FALSE FALSE  TRUE  TRUE  TRUE  TRUE  TRUE
\end{Soutput}
\end{Schunk}

그럼, $x>5$와  $x>7$ 이라는 두개의 조건을 만족하는 결과는 어떻게 표현할까요?
두개의 조건을 서로 결합하는 것은 \& (엠퍼센트) 라는 기호를 이용하여 표기합니다. 
 
\begin{Schunk}
\begin{Soutput}
> x > 5 & x < 7
 [1] FALSE FALSE FALSE FALSE FALSE  TRUE FALSE FALSE FALSE FALSE
\end{Soutput}
\end{Schunk}

위와 같은 표현이 가능하다면 $x<3$ 또는 $x>5$ 라는 논리는 어떻게 표현될까요? 
이는 $|$ 를 이용하여 표시합니다. 

\begin{Schunk}
\begin{Soutput}
> x < 3 | x > 5
 [1]  TRUE  TRUE FALSE FALSE FALSE  TRUE  TRUE  TRUE  TRUE  TRUE
\end{Soutput}
\end{Schunk}

논리연산에는 다음과 같은 것들이 있습니다
\begin{Schunk}
\begin{Soutput}
<, >, <=, >=, &, &&, |, ||, 
\end{Soutput}
\end{Schunk}

이 논리연산은 추후에 설명하게 될 수학의 집합연산과도 관계가 깊습니다. 
이에 대한 설명은 ``수학/확률/수치연산'' 이라는 챕터에서 다루도록 하고, 여기에서는 단순한 논리연산만을 다루겠습니다. 

\paragraph{plus() 예제 확장: }

조건을 하나만 사용하는 것이 아니라 여러개를 사용할 수 있습니다. 
예를들어, 우리가 수집하는 두개의 데이터가 x 와  y 에 저장을 하는데, 실제적으로는 0 보다 큰 값들만이 존재합니다. 
따라서, plus() 함수를 사용할때,  

\begin{itemize}
\item 두개의 입력인자가 모두 양수인 경우에만 plus()연산을 수행하고,
\item 만약 그렇지 않다면 "Both x1 and x2 must be positive" 라는 메시지를 출력하는 것입니다.
\end{itemize}

아래와 같이 프로그램을 살짝 변경합니다. 

\begin{Schunk}
\begin{Soutput}
plus <- function(x1, x2){
	if( (x1 > 0) & (x2 > 0) ){
		y <- x1 + x2
		return(y)
	}
	else print("Both x1 and x2 must be positive")
}
\end{Soutput}
\end{Schunk}

이제 그 결과를 확인해 봅니다. 

\begin{Schunk}
\begin{Soutput}
> plus(3,2)
[1] 5
> plus(-3,2)
[1] "Both x1 and x2 must be positive"
> plus(3,-2)
[1] "Both x1 and x2 must be positive"
\end{Soutput}
\end{Schunk}

사용자가 예상한대로 프로그램이 잘 짜지고 있음을 확인할 수 있습니다. 

이제 한 단계 더 업그레이드 합니다. 
실제로 더하기라는 개념은 두개의 값을 필요로 합니다. 
그래서 plus()라는 함수를 사용할때 두 개의 인자들이 입력되어야 합니다.
따라서, 아래와 같이 해봅니다.

\begin{itemize}
\item 만약, 입력된 인자의 개수가 2개 아니라면 "Only two arguments are needed for this computation" (이 연산을 수행하기 위해서는 두개의 인자가 필요합니다) 라는 메시지를 출력하고, 
\item 입력된 인자의 개수가 2개라면 연산을 수행하되, 오로지 두개의 인자값들이 양수일때만 연산을 수행합니다. 
\end{itemize}

이를 수행하기 위해서는 여러분들은 missing()이라는 함수와 stop()이라는 함수를 알아야 합니다. 
missing()이라는 함수는 사용된 변수의 값이 있는지 없는지를 판단하며, stop()이라는 함수는 메시지를 보여준뒤 그 뒤에 있는 프로세스들은 모두 중단한 채로 함수를 빠져나가는 것입니다. 

\begin{Schunk}
\begin{Soutput}
plus <- function(x1, x2){
	if(missing(x1) | missing(x2)) stop("Two arguments are needed")
	
	if( (x1 > 0) & (x2 > 0) ){
		y <- x1 + x2
		return(y)
	}
	else print("Both x1 and x2 must be positive")
}
\end{Soutput}
\end{Schunk}

이제 확인을 해 봅니다. 

\begin{Schunk}
\begin{Soutput}
> plus(3)
Error in plus(3) : Two arguments are needed
> plus(-3,2)
[1] "Both x1 and x2 must be positive"
> plus(3,5)
[1] 8
\end{Soutput}
\end{Schunk}

모두 사용자의 로직대로 프로그램이 작성되었음을 확인할 수 있습니다. 


그런데 가끔 이런 경우가 있습니다. 
조건문이 어떤 특정한 값을 받고 이에 해당하는 경우만을 실행하고자 할때입니다. 
예를들면, 아래와 같습니다.

\begin{itemize}
\item 만약, 사용자가 "A"를 입력하면 "Hi! What's up?" 을 출력합니다.
\item 만약, 사용자가 "B"를 입력하면 "Hello~ World! I am R" 을 출력합니다.
\end{itemize}

이 경우에 어떤 독자는 if문을 배웠으니 아래와 같이 할 것입니다.

\begin{Schunk}
\begin{Soutput}
urInput <- function(x){
	if(x=="A") print("Hi! What's up?")
	if(x=="B") print("Hello~ World! I am R")
}
\end{Soutput}
\end{Schunk}

아래와 같이 결과를 확인해 봅니다. 
\begin{Schunk}
\begin{Soutput}
> urInput("A")
[1] "Hi! What's up?"
> urInput("B")
[1] "Hello~ World! I am R"
> urInput("C")
> urInput(3)
\end{Soutput}
\end{Schunk}

여기에서 "C" 또는 3 을 입력 했을 때는 아무것도 수행하지 않았음을 알 수 있습니다. 
이와 동일한 기능을 R은 switch()라는 함수를 통하여 수행할 수도 있습니다. 

\begin{Schunk}
\begin{Soutput}
urInput2 <- function(x){
	switch(x, 
		"A" = print("Hi! What's up?"),
		"B" = print("Hello~ World! I am R")
	)
}
\end{Soutput}
\end{Schunk}

이제 동일한 결과를 얻는지 확인해 봅니다. 

\begin{Schunk}
\begin{Soutput}
> urInput2("A")
[1] "Hi! What's up?"
> urInput2("B")
[1] "Hello~ World! I am R"
> urInput2("C")
>
\end{Soutput}
\end{Schunk}

이러한 조건적 실행을 수행하는 분기문에 대해서는 프로그래밍의 경험이 쌓여 가면서 자연스레 더욱 많은 것을 알게 됩니다. 
본 섹션에서는 분기문에 대해서 이정도로 마무리하고 다음 장으로 넘어가겠습니다. 


\section{반복문}

이전 섹션에서는 간단하게 2개의 인자들을 입력받아 더하기를 수행하는 함수 plus()를 작성하였습니다.
이번 섹션에서는 반복문이라는 개념을 이해하기 위하여 임의의 벡터 x를 입력받아 그 객체의 구성요소들의 합을 구하는 mySum()이라는 함수를 만들어 보도록 합니다. 

먼저 1부터 10까지의 정수로 이루어진 수열을 생성해 봅니다.

\begin{Schunk}
\begin{Soutput}
> x <- 1:10
> x
 [1]  1  2  3  4  5  6  7  8  9 10
> 
\end{Soutput}
\end{Schunk}

R은 객체 x의 구성요소들을 더하고자 할때 sum()이라는 함수를 이용합니다.

\begin{Schunk}
\begin{Soutput}
> sum(x)
[1] 55
\end{Soutput}
\end{Schunk}

실제 연산은 아래와 같이 표현할 수 있을 것입니다.

\begin{Schunk}
\begin{Soutput}
> x[1] + x[2] + x[3] + x[4] + x[5] + x[6] + x[7] + x[8] + x[9] + x[10]
[1] 55
\end{Soutput}
\end{Schunk}

\paragraph{for() 이용하기}

만약, 벡터의 구성요소가 100개라면 이렇게 모든 구성요소를 일일이 작성하여 합을 구하는것은 비효율적일 것입니다. 
이런 경우 for()라는 반복문을 이용합니다. 

합계를 구하는데 있어서 반복문이 사용되는 근본 원리는 벡터의 첫번째 구성요소부터 마지막 구성요소까지의 각각의 구성요소가 가지는 값을 순차적으로 얻을 수 있는가 입니다.
따라서, 몇번째 구성요소를 표시하는 지시자가 필요하며, 이 지시자에 해당하는 값을 불러오는 과정이 필요합니다.
마지막으로 지시자에 해당하는 값을 불러왔을 때, 이 값은 이전의 합계에 더해짐으로서 현재의 합계를 얻게 되는 것입니다.
이를 R 로 표현하면 아래와 같습니다. 

\begin{Schunk}
\begin{Soutput}
> x <- 1:10
> val <- numeric(1)    # 총계를 저장할 변수를 미리 생성합니다.
> for(idx in x){       # idx는 x의 몇번째 구성요소를 나타내는 지시자입니다
+ print(x[idx])        # idx번째에 해당하는 x의 값을 출력해봅니다 
+ val <- val + x[idx]  # 합계를 위해서 생성해 놓은 변수에 idx 번째에 해당하는 x의 값을 더함으로서 합계의 값을 갱신합니다.
+ }
[1] 1
[1] 2
[1] 3
[1] 4
[1] 5
[1] 6
[1] 7
[1] 8
[1] 9
[1] 10
> val                   # 총계를 확인합니다. 
[1] 55
> 
\end{Soutput}
\end{Schunk}

이렇게 합계를 구하는 기능을 함수로 만드는 것은 어렵지 않습니다.

\begin{Schunk}
\begin{Soutput}
> mySum <- function(x){
+ val <- numeric(1)
+ for(idx in x){
+ val <- val + x[idx]
+ }
+ return(val)
+ }
> x <- 1:10
> mySum(x)
[1] 55
\end{Soutput}
\end{Schunk}

\paragraph{while() 사용하기}
R은 for()라는 반복문과 동일한 기능을 가지는 while()이라는 다른 형식의 반복문을 제공합니다.
for()라는 반복문이 지시자를 이용하여 중괄호 안의 표현식들을 반복하게 되지만, while()은 조건문을 이용하여 중괄호 안의 표현식을 반복하게 됩니다. 
for()에 대한 이해를 돕기 위해 사용되었던 1부터 10사이의 정수의 합계를 구하는 예제는 while()문을 이용하여 아래와 같이 할 수 있습니다.

\begin{Schunk}
\begin{Soutput}
> x <- 1:10
> idx <- 1              # 벡터의 첫번째 구성요소를 지정하기 위한 초기값 지정 
> while(idx <= 10){     # 만약, 지시자의 값이 10보다 작다면 중괄호의 표현식을 반복 
+ print(x[idx])         # 지시자에 해당하는 x의 값을 출력함 
+ idx <- idx + 1        # 다음 구성요소를 지시하기 위해서 현재 지시자의 값을 하나 업데이트 함.
+ }
[1] 1
[1] 2
[1] 3
[1] 4
[1] 5
[1] 6
[1] 7
[1] 8
[1] 9
[1] 10
> 
\end{Soutput}
\end{Schunk}

이렇게 얻어진 각각의 구성요소들의 값에 대한 총계를 구하고자 한다면 아래와 같이 아래와 같이 할 수 있습니다.

\begin{Schunk}
\begin{Soutput}
> x <- 1:10
> idx <- 1
> val <- 0
> while(idx <= length(x)){
+ val <- val + x[idx]
+ print(val)
+ idx <- idx + 1
+ }
[1] 1
[1] 3
[1] 6
[1] 10
[1] 15
[1] 21
[1] 28
[1] 36
[1] 45
[1] 55
> 
\end{Soutput}
\end{Schunk}

이를 함수로 만든다면 아래와 같이 할 수 있습니다. 

\begin{Schunk}
\begin{Soutput}
> mySum2 <- function(x){
+ idx <- 1
+ val <- 0
+ while(idx <= length(x)){
+ val <- val + x[idx]
+ idx <- idx + 1
+ }
+ return(val)
+ }
> x <- 1:10
> mySum2(x)
[1] 55
\end{Soutput}
\end{Schunk}

\paragraph{next와 break 활용}

위에서 사용한 mySum()이라는 함수를 조금 더 활용해 보도록 합니다.

만약, 사용자가 1부터 10 사이의 정수들의 총계가 아닌, 1부터 7까지만의 합계를 알고 싶다고 가정합니다. 
이런 경우에 x의 구성요소의 개수는 10개이지만, 첫번째부터 7번째 구성요소까지만 더하기를 수행하고 그 반복을 중단하면 될 것입니다.
이러한 기능을 위하여 R은 반복문을 사용할 때 break 라는 기능을 제공하고 있습니다.
이는 분기문이 실행될 때 해당 반복문을 완전 중단하고자 하는 경우에 사용됩니다. 
아래의 프로그램을 살펴보시길 바랍니다. 

\begin{Schunk}
\begin{Soutput}
> x <- 1:10
> for(idx in x){
+ if(x[idx] == 8) break  # 만약, 지시자가 8번째라면 반복문을 중단함. 
+ print(x[idx])
+ }
[1] 1
[1] 2
[1] 3
[1] 4
[1] 5
[1] 6
[1] 7
> 
\end{Soutput}
\end{Schunk}

이제 x가 가지는 값들 중에서 홀수만을 골라서 합을 구하는 프로그램을 짜봅니다. 
먼저, 홀수만 더해진다면 아래와 같은 결과값을 가져야 할 것입니다. 

\begin{Schunk}
\begin{Soutput}
> x[1] + x[3] + x[5] + x[7] + x[9]
[1] 25
\end{Soutput}
\end{Schunk}

이를 해결하는 방법은 x 라는 구성요소를 for()라는 반복문을 이용하여 순차적으로 그 값을 얻어 합계를 내고자 할때, 그 값이 짝수라면 더하는 과정을 실행하지 않고 홀수라면 더하는 과정을 실행하면 될 것입니다.
R은 이러한 컨트롤을 위하여 next 라는 예약어를 제공하고 있습니다.
즉, 어떤 조건에 대하여 분기를 실행할 때 next가 이용된다면 현재의 반복과정이 실행되지 않고 바로 다음의 반복과정으로 넘어가게 됩니다. 
따라서, 아래와 같이 프로그램을 작성할 수 있습니다. 

\begin{Schunk}
\begin{Soutput}
> x <- 1:10
> val <- numeric(1)
> for(idx in x){
+ if(x[idx] %% 2 == 0) next
+ val <- val + x[idx]
+ }
> val
[1] 25
\end{Soutput}
\end{Schunk}

\paragraph{벡터라이제이션: }

실제로 이렇게 1부터 10사이의 홀수들의 합계를 구하는 것은 반복문을 사용하지 않고 1줄로 표현할 수 있습니다.
이렇게 벡터라이제이션의 기능을 활용하여 주어진 연산을 수행하는 것은 프로그램을 짧고 그 가독성을 높일 수 있습니다. 
가독성이란 프로그램이 얼마나 읽기가 쉬운가입니다.

\begin{Schunk}
\begin{Soutput}
> sum(x[x %% 2 != 0])
[1] 25
\end{Soutput}
\end{Schunk}

그러나, 이렇게 벡터라이제이션을 하는 것은 프로그래밍에 대한 경험이 쌓여야 합니다. 
여기에서는 기초 프로그래밍을 위한 개념을 이해하기 위하여 for()과 while()을 설명했으나, 이 이후로의 모든 내용은 for()와 while()을 이용한 반복문은 가급적이면 피할 것입니다.
바로 다음에 다루게 될 데이터 조작 및 수치해석에 관련된 챕터들에서는 R의 특징을 최대한 살린 벡터라이제이션 중심의 연산방식으로 표현함으로서 독자에게 도움이 되고자 하였습니다.

반복문에서 repeat 라는 기능이 있습니다. (이것에 대한 설명은 문서를 읽던 독자의 요청이 있으면 기재하도록 합니다).

\section{프로그램의 가독성}

많은 초보 프로그래머들은 R 프로그램을 작성할 때 반드시 따라야 할 코딩 스타일을 따르고자 합니다.
코딩스타일을 이야기 하는 것은 프로그램의 가독성 때문입니다. 
가독성이란 내가 작성한 프로그램을 다른 사람이 쉽게 이해할 수 있으며, 다른 사람이 작성한 프로그램을 내가 얼마나 이해할 수 있는가를 의미합니다.
이러한 가독성 때문에 프로그램을 작성할 때 어떤 작성 표준에 맞추어 짜야 한다는 것입니다.
그리고, 이러한 가독성은 결국 공동작업을 통한 협업을 할 때 그 힘을 발휘하게 됩니다. 

사실상 그러한 코딩스타일에 대한 표준은 사실상 존재하지 않지만, 가독성과 공동작업을 위하여 일종의 코딩컨벤션 (coding convention)을 사용합니다. 
\begin{itemize}
\item GNU C 프로그래밍 코딩 스탠다드 (\url{http://www.gnu.org/prep/standards/standards.html})이 도움이 될 것입니다. 
\item 구글에 포스팅되어 있는 구글의 R 스타일 가이드 (\url{http://google-styleguide.googlecode.com/svn/trunk/google-r-style.html})이 도움이 될 것입니다. 
\item 그리고 Github에 게시된 Hadley의 스타일 가이드 (\url{https://github.com/hadley/devtools/wiki/Style}) 등이 여러분들의 R 프로그래밍 스타일을 갖추는데 도움이 될 것입니다. 
\end{itemize}

프로그램의 가독성을 높이는 또 하나의 방법은 주석문을 많이 활용하는 것입니다.
예를들어, 위에서 우리가 작업한 plus() 함수는 만약 다른 사람이 보면 잘 이해하지 못할 것입니다. 
따라서 아래와 같이 저장해 두는 것이 좋습니다.

\begin{Schunk}
\begin{Soutput}
# 
# Authors: (프로그램 작성자 이름)
# Created on (작성날짜) 
# Modified on (최종수정일)
# 
# Description: (간단한 설명)
# 이 프로그램은 나의 첫번째 R 프로그램인 plus() 함수를 작성한 것입니다.
# 

# 두개의 입력인자를 가진는 plus()함수의 정의 

plus <- function(x1, x2){

	# 입력인자가 2개임을 확인합니다.
	if(missing(x1) | missing(x2)) stop("Two arguments are needed")
	
	# 입력인자의 값은 반드시 양수이어야 합니다. 
	if( (x1 > 0) & (x2 > 0) ){
		y <- x1 + x2
		return(y)
	}
	else print("Both x1 and x2 must be positive")
}
\end{Soutput}
\end{Schunk}


% http://www.wekaleamstudios.co.uk/posts/programming-with-r-checking-function-arguments/

\paragraph{프로그래밍을 도와주는 여러가지 유틸리티들: } 

본 섹션에서는 간단한 예제를 통하여 R에서 제공하는 프로그래밍 요소인 조건문, 반복문, 그리고 함수에 대해서 알아보았습니다. 
문서의 처음에 언급했던바와 같이 분석은 일련의 프로세스를 수행하는 것이고, 이러한 수행을 자동화 시켜주는 것에 있어서 프로그래밍은 매우 효과적인 도구입니다.
따라서, 어떤 일을 수행하도록 하는 명령어들이 복잡해지고 그 양이 많아진다면 외부파일에 스크립트를 작성후 배치처리를 하는 것이 좋습니다.
또한, 자신의 시스템 사용환경에 익숙하다면 프로그램을 작성하는데 더욱 도움이 됩니다. 
이러한 내용은 이 문서의 ``환경설정과 유틸리티'' 라는 챕터에 기록해 두었습니다. 

본 챕터는 사용에 익숙해지기 위한 아주 기초적인 내용만을 다루기 때문에 아래와 같은 내용은 ``패키지 작성'' 이라는 챕터에서 다루도록 하겠습니다. 

\begin{itemize}
	\item 렉시컬 스코핑 
	\item environemnt
	\item 제네릭 함수 
	\item 메소드와 클래스 
	\item S3와 S4 프로그래밍 그리고 R5
	\item 에러 핸들링 - try(), tryCatch()
	\item 운영체제와의 소통 - 파일과 디렉토리 관리에 필요한 유틸리티들 
\end{itemize}



\chapter{시간과 문자열}

이 부분을 하나의 챕터로 빼는 것이 더욱 나을 것 같음.

% 베리굿 노트: http://www.pitt.edu/~njc23/
% http://cm.bell-labs.com/cm/ms/departments/sia/Sbook/
% http://developer.r-project.org/methodDefinition.html
% http://www.biostat.jhsph.edu/~rpeng/teaching.html
% http://onertipaday.blogspot.ca/
% http://web.udl.es/Biomath/Bioestadistica/R/Manuals/r_in_a_nutshell.pdf
% http://web.udl.es/Biomath/Bioestadistica/R/Manuals/

% \item \texttt{do.call()} 함수를 사용하는 법에 대해서..
% http://cran.r-project.org/web/packages/rockchalk/vignettes/Rchaeology.pdf
% http://www.stat.berkeley.edu/classes/s133/all2011.pdf  (다운로드 해두었음 Desktop/tmpRsrc/all2011.pdf)
% http://www.stat.berkeley.edu/classes/s133/resources.html 
	

%%%%%%%%%%%%%%%%%%%%%%%%%%%%%%%%%%%%%%%%%%%%%%%%%%%%%%%%%%%%%%%%%%%%%%%%
%
% CHAPTER
%
%%%%%%%%%%%%%%%%%%%%%%%%%%%%%%%%%%%%%%%%%%%%%%%%%%%%%%%%%%%%%%%%%%%%%%%%



\chapter{데이터 조작과 관련하여}

이번 챕터부터는 실질적으로 데이터를 다루게되는 경우를 집중적으로 살펴보도록 하겠습니다.
데이터를 다룸에 있어서는 이전 챕터에서 다루었던, 벡터, 행렬, 그리고 배열이라는 데이터형 외에 요인, 데이터프레임과 리스트라는 형식을 반드시 알고 있어야 합니다.
이러한 형식의 제공은 통계적 프로그래밍의 관점에서 제공되어지는 것입니다. 
따라서, 데이터 프레임과 리스트에 대한 내용을 먼저 설명한 뒤에 데이터 클리닝에 필요한 다양한 조작법들에 대해서 알아보도록 합니다.

\section{리스트}
데이터 프레임을 설명하기 전에 리스트라는 데이터형을 먼저 설명합니다.
그 이유는 데이터 프레임은 리스트의 특수한 경우이기 때문입니다.
리스트는 벡터와 같은 방식으로 생성되고 사용되지만, 한가지 다른 점이 있습니다.
벡터의 경우에는 해당 벡터의 모든 구성요소는 모두 같은 데이터형을 가지고 있어야 합니다.
예를들어, 구성요소가 숫자라면 수치형 벡터라고 특징지을 수 있는데, 이는 해당벡터의 모든 구성요소가 예외없이 숫자형만을 가져야 하기 때문입니다.
문자형 벡터의 경우에도 마찬가지 입니다. 
벡터의 구성요소 각각이 모두 문자형만을 가져야만 합니다.
\begin{Schunk}
\begin{Soutput}
> x <- 1:5
> x
[1] 1 2 3 4 5
> y <- LETTERS[1:5]
> y
[1] "A" "B" "C" "D" "E"
> mode(x)
[1] "numeric"
> mode(y)
[1] "character"
\end{Soutput}
\end{Schunk}

그러나, 리스트는 구성요소의 데이터형에 구애받지 않습니다.
예를들면, 리스트의 첫번째 구성요소가 문자형 값을 가질때 두번째 구성요소는 숫자형 값을 가질 수 있습니다.
좀 더 나아가 리스트의 첫번째 구성요소가 문자형 벡터를 가질 때, 두번째 구성요소는 수치형 행렬을 가지고, 세번째 구성요소는 숫자형 배열을 가질 수도 있습니다.
그리고, 네번째 구성요소는 우리가 다음 섹션에서 다루게 될 데이터 프레임이라는 형식을 가질 수도 있으며, 다섯번째 구성요소가 현재 설명하고 있는 리스트형을 가질 수도 있습니다.

이러한 리스트는 list()함수를 이용하여 아래와 같은 방법으로 생성하게 됩니다.

\begin{Schunk}
\begin{Soutput}
> x <- 1:5
> y <- LETTERS[1:5]
> z <- matrix(c(1,2,3,4,5,6), ncol=3)
> xyz <- list(x,y,z)
> xyz
[[1]]
[1] 1 2 3 4 5

[[2]]
[1] "A" "B" "C" "D" "E"

[[3]]
     [,1] [,2] [,3]
[1,]    1    3    5
[2,]    2    4    6
 
\end{Soutput}
\end{Schunk}

이렇게 생성된 리스트의 구성요소들은 $[[$ 와 $]]$ 를 이용하여 가져올 수 있습니다.

\begin{Schunk}
\begin{Soutput}
> xyz[[1]]
[1] 1 2 3 4 5
> xyz[[2]]
[1] "A" "B" "C" "D" "E"
> xyz[[3]]
     [,1] [,2] [,3]
[1,]    1    3    5
[2,]    2    4    6
> 
\end{Soutput}
\end{Schunk}

또한, 리스트의 구성요소를 구성하는 구성요소들을 아래와 같은 방법으로 가져올 수 있습니다.
\begin{Schunk}
\begin{Soutput}
> xyz[[1]][3]
[1] 3
> xyz[[3]][2,3]
[1] 6
> xyz[[2]][3]
[1] "C"
> 
\end{Soutput}
\end{Schunk}

이렇게 주어진 리스트는 아래와 같이 mode()를 이용하여 확인된 것과 같이 list라는 형식을 가지고 있으며, is.list()라는 함수를 통하여 확인이 가능합니다.

\begin{Schunk}
\begin{Soutput}
> mode(xyz)
[1] "list"
> is.list(xyz)
[1] TRUE
\end{Soutput}
\end{Schunk}

그런데, 만약 리스트의 구성요소의 개수들이 많아진다면 이 리스트의 구조를 살펴보는 것이 리스트라는 데이터형을 다루는데 도움이 될 것이며, 이를 위해서는 str()이라는 함수를 사용합니다.

\begin{Schunk}
\begin{Soutput}
> str(xyz)
List of 3
 $ : int [1:5] 1 2 3 4 5
 $ : chr [1:5] "A" "B" "C" "D" ...
 $ : num [1:2, 1:3] 1 2 3 4 5 6
> 
\end{Soutput}
\end{Schunk}
%$ 

이러한 리스트에 이름을 붙이는 방법은 아래와 같습니다.

\begin{Schunk}
\begin{Soutput}
> mylist <- list(x=x, y=y, z=z)
> mylist
$x
[1] 1 2 3 4 5

$y
[1] "A" "B" "C" "D" "E"

$z
     [,1] [,2] [,3]
[1,]    1    3    5
[2,]    2    4    6
\end{Soutput}
\end{Schunk}
%$

이렇게 이름이 붙여진 리스트 mylist 라는 것은 아래와 같이 names()라는 함수를 이용하여 확인이 가능합니다.

\begin{Schunk}
\begin{Soutput}
> names(mylist)
[1] "x" "y" "z"
\end{Soutput}
\end{Schunk}

이렇게 이름이 부여된 이후에는 리스트 구성요소의 이름을 이용하여 불러올 수 있습니다.

\begin{Schunk}
\begin{Soutput}
> mylist$x
[1] 1 2 3 4 5
> mylist$y
[1] "A" "B" "C" "D" "E"
> mylist$z
     [,1] [,2] [,3]
[1,]    1    3    5
[2,]    2    4    6
> 
\end{Soutput}
\end{Schunk}
%$

R에서 제공하는 많은 통계관련 함수들은 이러한 리스트의 특징을 활용합니다.
list()는 이후에 설명하겠지만, 사용자 함수의 작성시에 여러개의 값들을 한번에 반환하고자 할때 return() 대신 이용됩니다.

\section{데이터 프레임}

이제 데이터 프레임이라는 데이터에 대해서 살펴보도록 합니다.
위에서 언급한 바와 같이 데이터 프레임이란 리스트의 특수한 경우입니다.
리스트의 생성과 활용방법은 동일하지만 두 가지 부분에서 다릅니다.
하나는 리스트의 구성요소들이 벡터형이어야 합니다. 
이때 벡터가 숫자형인지 문자형인지에 대한 종류에는 관계가 없습니다.
또 다른 하나는 모든 리스트의 구성요소들이 같은 길이를 가져야 합니다.
이와 같은 두개의 조건이 성립할때 리스트를 데이터 프레임이라고도 합니다.
이 데이터프레임은 통계분석에 있어서 데이터가 저장되어 있는 스프레드 형식을 가지기 때문에 매우 유용하게 활용될 수 있습니다.

다음은 데이터 프레임을 생성하는 방법인데, 리스트를 생성하는 방법과 동일하다는 것을 알 것입니다.
\begin{Schunk}
\begin{Soutput}
> v1 <- c(163, 178, 170, 167, 169)
> v2 <- c("f", "m", "m", "f", "m")
> mydata <- data.frame(v1,v2)
> mydata
   v1 v2
1 163  f
2 178  m
3 170  m
4 167  f
5 169  m
\end{Soutput}
\end{Schunk}

이 데이터프레임의 이름을 변경할 수 있으며, 이 또한 리스트의 이름을 변경하는 것과 동일하다는 것을 알 수 있습니다.

\begin{Schunk}
\begin{Soutput}
> names(mydata)
[1] "v1" "v2"
> names(mydata) <- c("Height", "Gender")
> mydata
  Height Gender
1    163      f
2    178      m
3    170      m
4    167      f
5    169      m
> 	
\end{Soutput}
\end{Schunk}

이 데이터프레임은 어떤 속성이 있는지 attributes() 함수를 이용하여 확인해 봅니다. 
\begin{Schunk}
\begin{Soutput}
> attributes(mydata)
$names
[1] "Height" "Gender"

$row.names
[1] 1 2 3 4 5

$class
[1] "data.frame"
>
\end{Soutput}
\end{Schunk}
%$

행의 이름 또한 아래와 같은 방법으로 변경이 가능합니다. 

\begin{Schunk}
\begin{Soutput}
> row.names(mydata)
[1] "1" "2" "3" "4" "5"
> row.names(mydata) <- c("ID-1", "ID-2", "ID-3", "ID-4", "ID-5")
> mydata
     Height Gender
ID-1    163      f
ID-2    178      m
ID-3    170      m
ID-4    167      f
ID-5    169      m
>
\end{Soutput}
\end{Schunk}

데이터 프레임은 리스트의 특수한 경우이기도 하지만, 행렬에 변수명이라는 속성을 붙인 것으로도 볼 수 있습니다.
따라서, 행렬에서 사용되는 함수들을 이용하여 차원, 열의 개수, 행의 개수들을 확인할 수 있습니다.

\begin{Schunk}
\begin{Soutput}
> dim(mydata)
[1] 5 2
> nrow(mydata)
[1] 5
> ncol(mydata)
[1] 2
> 
\end{Soutput}
\end{Schunk}


\section{요인}

위의 데이터 프레임에서 여러가지 속성들을 덧붙인 mydata의 구조를 str()을 이용하여 살펴보면 다음과 같습니다.

\begin{Schunk}
\begin{Soutput}
> str(mydata)
'data.frame':	5 obs. of  2 variables:
 $ Height: num  163 178 170 167 169
 $ Gender: Factor w/ 2 levels "f","m": 1 2 2 1 2
> 
\end{Soutput}
\end{Schunk}

그런데, Gender 라는 변수에 대해서 한가지 새로운 사실을 알게 됩니다. 
분명히 v2라는 벡터를 이용했는데, str()를 이용하여 구조를 확인한 결과로는 2개의 수준들을 가진 요인 (Factor w/ 2 levels)라고 정보가 나타납니다.
이는 R이 통계분석적인 측면에서 디자인되었기 때문에 가지는 특징입니다.
모든 문자형 벡터들은 요인으로 간주되어지며, 해당 벡터가 가지는 값의 범위를 수준으로 인식하게 됩니다.
이러한 요인은 주로 범주형 데이터를 표시할때 이용됩니다.


% http://www.statmethods.net/input/datatypes.html

%%%%%%%%%%%%%%%%%%%%%%%%%%%%%%%%%%%%%%%%%%%%%%%%%%%%%%%%%%%%%%%%%%%%%%%%
%
% SECTION
%
%%%%%%%%%%%%%%%%%%%%%%%%%%%%%%%%%%%%%%%%%%%%%%%%%%%%%%%%%%%%%%%%%%%%%%%%

\section{데이터 조작 실무예제}

이전 섹션까지 R에서 다루는 다양한 종류의 데이터형들에 대해서 알아보았습니다.
통계분석 실무에서는 분석자가 보통 얻게 되는 데이터는 분석에 사용되는 통계모형의 적용요건에 부합하는 경우는 드물기 때문에 분석자 스스로가 이러한 데이터를 형성하는 것은 필요한 기술중에 하나라고 할 수 있습니다.

서울종합과학대학원 사회학과의 신종화 교수님께서 본 섹션을 위해서 dart8.xls 이라는 데이터를 제공해 주셨습니다. 
실무에서 접할 수 있는 다양한 점들을 부각하고자, 신종화 교수님으로부터 전달받은 데이터에는 그 어떠한 수정도 이루어지지 않았습니다.
분석을 위한 데이터들을 만들기 위해서 R은 이 데이터를 어떻게 받아들이고 처리했는지 상세히 기록하고자 하였습니다.

\paragraph{데이터셋에 대한 간단한 설명:}  
제공된 데이터셋은 8개의 여행사들의 ``광고선전비'', ``교육훈련비'', ``매출액''을 2000년 12월부터 2011년 9월까지 월별로 기록한 내용이 dart8.xls 이라는 엑셀파일에 저장되어 있습니다. 
이 자료는 여행사별 자료는 개별 워크시트에 따로 저장이 되어 있으며, 변수명은 한글이 사용되어 있습니다.

본 데이터는 \href{http://korea.gnu.org/gnustats/dataset/dart8.xls}{여기 다운로드 링크}를 눌러 다운받을 수 있습니다.

\paragraph{엑셀 데이터 불러오기:}  
R을 이용하여 분석을 준비하고자 한다면 데이터를 R로 불러오는 것이 그 첫번째 작업일 것입니다.
이 데이터는 한 개의 엑셀파일에 있는 8개의 워크시트에 분산되어 있기 때문에 먼저 하나로 모두 모아야 합니다.
다행이도 각 워크시트에 정리되어 있는 데이터는 동일한 개수의 변수들이 있고, 변수들의 순서도 일치합니다.
이를 수행하기 위해서 gdata이라는 패키지를 먼저 불러옵니다.

\begin{Schunk}
\begin{Soutput}
> library(gdata)
\end{Soutput}
\end{Schunk}

\subparagraph{예상치 못한 문제의 발생과 원인:} 
먼저 엑셀 파일에 제대로 읽힐 수 있는가를 확인하기 위해서 하나의 워크시트를 테스트용으로 읽어봅니다.

\begin{Schunk}
\begin{Soutput}
> tmp <- read.xls(xls="dart8.xls", sheet=1)
Wide character in print at /usr/lib/R/site-library/gdata/perl/xls2csv.pl line 211.
Wide character in print at /usr/lib/R/site-library/gdata/perl/xls2csv.pl line 270.
> 
\end{Soutput}
\end{Schunk}

예상하지 않았던 문제가 발생했습니다. 
``wide character in print''라는 메시지는 R의 버그가 아니라, 엑셀데이터를 파싱하는데 사용되는 Perl이라는  언어가 데이터내에 유니코드문자가 있을때 발생시키는 메시지 입니다.
즉, 한글을 표현하는 멀티바이트 인코딩 때문에 발생하는 것입니다.
이러한 메시지가 나왔을지라도 실제로 데이터를 홖인해보시면 데이터에는 아무런 손상이 없음을 알 수 있습니다.

\subparagraph{XLConnect 패키지의 활용:}
그런데, 이러한 메시지를 보게되면 웬지 모르게 꺼림칙합니다. 
아니면 원본데이터의 변수명 자체를 없애거나 변경하거나 해야할 것입니다.
그러나, 데이터클리닝시에는 원본 데이터에는 절대로 손을 대어서는 안됩니다.
모든 클리닝 작업은 기록화 되어 추후에도 자동적으로 처리가 될 수있도록 해야합니다.
이것은 추후에 제3자에 의한 reproducible research 가 가능하도록 하여 분석의 객관성을 유지할 수 있습니다.

따라서, XLConnect 라는 운영체제에 관계없이 사용될 수 있는 또 다른 종류의 패키지를 아래와 같이 불러왔습니다.

\begin{Schunk}
\begin{Soutput}
> library(XLConnect)
\end{Soutput}
\end{Schunk}

그리고 아래와 같이 데이터를 읽어왔더니, 어떠한 메시지 없이 잘 수행되었음을 볼 수 있었습니다. 

\begin{Schunk}
\begin{Soutput}
> library(XLConnect)
> tmp <- readWorksheetFromFile(file="dart8.xls", sheet=1)
> 
\end{Soutput}
\end{Schunk}

\paragraph{데이터의 처음과 마지막 부분을 확인:} 
정말로 잘 수행되었는지를 확인하기 위해서 데이터의 처음과 마지막을 살펴볼 수 있는 head()와 tail()함수를 이용해 봅니다.

\begin{Schunk}
\begin{Soutput}
> head(tmp)
  구....분 광고선전비 교육훈련비      매출액
1  2000.12  161702806   18002000  5616224889
2  2001.03   80485618   28146500  7188763335
3  2001.06  170827271   12965900  7948588645
4  2001.09   65667863   26468000 11509839298
5  2001.12   27804868   16838062  7799015935
6  2002.03   81945640   12752112 10491229385
> tail(tmp)
   구....분 광고선전비 교육훈련비      매출액
39  2010.06 2611994098   34151731 48451368536
40  2010.09 1723098483          0 66245202818
41  2010.12 2930265435   98244331 54941857566
42  2011.03 2449466000   36710000 63509496808
43  2011.06 2818383000   58809000 47554958058
44  2011.09 2357446000   27714000 66186330843
> 
\end{Soutput}
\end{Schunk}

원본데이터 dart8.xls와 비교를 해보니 불러온 데이터에 아무런 오류가 없음을 확인할 수 있었습니다.

%%% To 정우준님,
%%% head()와 tail()은 옵션 n을 조정하여 출력하는 레코드의 개수를 조정할 수 있습니다.
%%% 그 내용을 넣어주세요.

\paragraph{엑셀 파일내 모든 워크시트 다 불러오기:} 
그러나, 이것은 하나의 워크시트만을 불러온 것으로 전체 워크시트를 불러오고자 하는 우리의 목적을 달성한 것은 아닙니다.
따라서 아래와 같이 워크시트를 모두 불러오는 오도록 합니다.

\begin{Schunk}
\begin{Soutput}
> wb <- loadWorkbook("dart8.xls")
> wb
[1] "dart8.xls"
> tmp <- readWorksheet(wb, sheet=getSheets(wb))
>
\end{Soutput}
\end{Schunk}

각각의 워크시트가 리스트 tmp의 각 구성요소에 성공적으로 불러들여졌습니다. 
실제로 이것은 아래와 같이 반복문의 개념을 통해 이루어진 것입니다.

\subparagraph{모든 워크시트의 이름 확인:} 
각각의 워크시트를 읽어오기 위해서는 먼저 어떤 이름을 가진 워크시트가 몇개가 있는지 알아야 할 것입니다.

\begin{Schunk}
\begin{Soutput}
> wid <- getSheets(wb)
> wid
[1] "하나투어"     "레드캡투어"   "모두투어"     "세중"         "참좋은레져"  
[6] "롯데관광개발" "자유투어"     "비티앤아이"  
> 
\end{Soutput}
\end{Schunk}

이 이름의 목록이 이미 불러들어온 목록과 일치함을 알 수 있습니다.

\begin{Schunk}
\begin{Soutput}
> names(tmp)
[1] "하나투어"     "레드캡투어"   "모두투어"     "세중"         "참좋은레져"  
[6] "롯데관광개발" "자유투어"     "비티앤아이"  
> 
\end{Soutput}
\end{Schunk}

워크시트를 순차적으로 불러오는 반복문을 수행해 봅니다.

\begin{Schunk}
\begin{Soutput}
> tmp <- list()
> for(idx in getSheets(wb)) tmp[[idx]] <- readWorksheetFromFile(file="dart8.xls", sheet=idx)
> tmp
$하나투어
   구....분 광고선전비 교육훈련비      매출액
1   2000.12  161702806   18002000  5616224889
2   2001.03   80485618   28146500  7188763335
3   2001.06  170827271   12965900  7948588645
4   2001.09   65667863   26468000 11509839298
5   2001.12   27804868   16838062  7799015935
6   2002.03   81945640   12752112 10491229385
7   2002.06  303080492    8428100 10962473266
8   2002.09  438342395    3475560 18635461755
9   2002.12  528533759    8138020 12683418184
10  2003.03  429523269   14535990 14859884927
11  2003.06   64015168    5109500  7029384084
12  2003.09  336786062   11394917 20892617700
13  2003.12  438942853   14812754 15647266644
14  2004.03  315572853   10338300 17635564925
15  2004.06  758663210   21191288 15474196887
16  2004.09 1153100003    5783080 26788322244
17  2004.12 1034522413   54309865 19861453701
18  2005.03  641446421    5388960 23268279482
19  2005.06 1165252715   45028986 22965942213
20  2005.09  992627324   11348510 37673171188
21  2005.12 1372387152   62163640 27122123547
22  2006.03  920580625   13672836 39746045641
23  2006.06 2123930825  132961683 31913477123
24  2006.09 1928612977    2985880 48605117233
25  2006.12 2393203854  102367110 46035346970
26  2007.03 1239215427   35432682 49819661853
27  2007.06 2058854025   94954440 42425180001
28  2007.09 1741477478          0 60634299844
29  2007.12 1771196490   40011868 46418905817
30  2008.03 1432606264   10453124 57624282255
31  2008.06 2098126845   65203394 43908392041
32  2008.09 1398280115   15837118 43516939794
33  2008.12  727493963   -3735654 27729106510
34  2009.03  528828105   10678454 30625278205
35  2009.06 1005134062    4777258 29636267486
36  2009.09  679725152    5375226 34845539027
37  2009.12  528153534  -20830938 28792370983
38  2010.03 1055742790   44554342 48480421492
39  2010.06 2611994098   34151731 48451368536
40  2010.09 1723098483          0 66245202818
41  2010.12 2930265435   98244331 54941857566
42  2011.03 2449466000   36710000 63509496808
43  2011.06 2818383000   58809000 47554958058
44  2011.09 2357446000   27714000 66186330843

$레드캡투어
   구....분  광고선전비 교육훈련비      매출액
1   2000.03     1840000          0  1093168868
2   2000.06     1134000          0  1515376363
3   2000.09      930000    1246020  2130147815
4   2000.12    10009090          0  3973329930
5   2001.03     5440000     100000  1261010772
6   2001.06     4180000     110000  3054480728
7   2001.09      644372          0  1695942024
8   2001.12          NA          0  4096816401
9   2002.03       80000      50000  1843152110
10  2002.06     5863845      80000  1434683008
11  2002.09      160000     400000   767838212
12  2002.12           0    1320000  2202840022
13  2003.03      840000     180000  1018896498
14  2003.06      100000      80000   817451869
15  2003.09           0     130000   302535025
16  2003.12        9000      44000   204728129
17  2004.03     2000000          0   552025936
18  2004.06     3578412     240000   953800877
19  2004.09           0     892430  1187171653
20  2004.12           0     -97570   673807053
21  2005.03      840000     160000  1281797282
22  2005.06           0          0  1637418175
23  2005.09      550000          0  1219981048
24  2005.12           0      80000   604751948
25  2006.03      840000      80000   568053703
26  2006.06      550000     240000   605465938
27  2006.09     1980000          0   938502536
28  2006.12     1625000     120000  1270813542
29  2007.03   767663233   27479356 13912283508
30  2007.06  1043098560   62698927 14327194117
31  2007.09  1479171609   43398756 16370510221
32  2007.12   525165387   57918044 26673357429
33  2008.03           0          0 19926326125
34  2008.06  1826496649   96007272 20911613216
35  2008.09  2748341000  145562000 19269912551
36  2008.12 -1432312659  -59481092 18218384709
37  2009.03   215422000    5027000 20855159928
38  2009.06   181660000   54300000 20806465770
39  2009.09   503662000   30857000 20201843391
40  2009.12    22083000   52101000 21278162428
41  2010.03   285347000    3554000 26266456633
42  2010.06   349981000   30268000 33396639702
43  2010.09   377363000   50701000 28589391364
44  2010.12   279352000   94395000 29306664228
45  2011.03   396275000   34736000 35928440908
46  2011.06   385406000   47663000 35180189334
47  2011.09   279924000  101540000 33937697440

$모두투어
   구....분 광고선전비 교육훈련비      매출액
1   2005.03  240939162          0  8144659648
2   2005.06  463698765          0  8548149085
3   2005.09  548353360          0 12789229280
4   2005.12   55201184          0  9392450163
5   2006.03  720601983          0 14671603224
6   2006.06  868056093          0 12358657708
7   2006.09  986967530    2075000 19989393935
8   2006.12 1944355509   19036000 19364220706
9   2007.03 1603949325   21327441 22693131593
10  2007.06 1486981449   13137318 19369500867
11  2007.09 3423372271   53335300 29525164926
12  2007.12 1260295463   28193029 22763656306
13  2008.03 1204193071   13200204 26530405258
14  2008.06 2679972645   66144746 20939237126
15  2008.09 1460678899   34482480 23136419134
16  2008.12  701137900   12212590 12659502899
17  2009.03  544481614    4611600 13037790654
18  2009.06  420150066   18224856 14039046595
19  2009.09  586547017    6552350 17831640182
20  2009.12  461808648    5522500 16466995715
21  2010.03  627109367   11581130 25003802066
22  2010.06  896659514   29407378 26329123344
23  2010.09 1265995826   20890280 36466418562
24  2010.12 1176649774   16647180 29286304144
25  2011.03 1119329000   21106000 33856914381
26  2011.06 1262617000   23868000 25788504600
27  2011.09 1066938000   16640000 36339683324

$세중
   구....분 광고선전비 교육훈련비      매출액
1   2000.09  -14386068    3890650   910409285
2   2000.12  993896486     159218  2817571796
3   2001.03  127116000    6420000  1883837000
4   2001.06  469787614    4204553  1674407342
5   2001.09  405677848    9285253  2043272736
6   2001.12  120953120   19780923  1992872884
7   2002.03   86683346   10914207  1454849568
8   2002.06  206628774   24193922  1404032357
9   2002.09   60978586    4247382  1178532818
10  2002.12   78896343    1647320  1499189369
11  2003.03   32975680    1100000  1072460156
12  2003.06   44874732     860000  1584639526
13  2003.09   31769502    1365000  1219759186
14  2003.12  756978698   61904440  3380350787
15  2004.03   44584722    2977870   756603288
16  2004.06  163037461    4677680  1959745267
17  2004.09  166148967    2033485  1276020776
18  2004.12  635019271    2850530  3146868792
19  2005.03   53200074     900000  1096624195
20  2005.06  163908857     412000  1446127886
21  2005.09   70423589     589500   900739681
22  2005.12  105881433     819320  1724624667
23  2006.03   72483619     500208  3086832585
24  2006.06   95598931     680000  3699008908
25  2006.09 1061239746    5708314 18492449432
26  2006.12 2606931469   60229861 35565924672
27  2007.03  685084202   16190280 16958575156
28  2007.06 1279207876   29395300 19918091583
29  2007.09 1152432010   17369340 19269681940
30  2007.12 1349318078   -7782930 16195353771
31  2008.03  930137416   29007679 16463319852
32  2008.06 1006599592    1090739 18276554526
33  2008.09  267514376    9583390 19159107089
34  2008.12  131415228   24538080 18050882379
35  2009.03  128959936    1252642 13597706178
36  2009.06  164430118    5023976 15680114832
37  2009.09   56881950    6980850 16130093114
38  2009.12   76691949    7046329 15292534175
39  2010.03   92244120   15798800 14808015873
40  2010.06  190779700   23429324 19124988108
41  2010.09   48997402   -4759221 20282845926
42  2010.12  102439249    4092070 20409449651
43  2011.03          0          0 19384007562
44  2011.06  341933000          0 22377775711
45  2011.09   19735000          0 21832230367

$참좋은레져
   구....분 광고선전비 교육훈련비      매출액
1   2007.03     500000          0  2915134989
2   2007.06    2836412          0  6782580656
3   2007.09   67680500         NA  4926211503
4   2007.12   24490909         NA  3857238621
5   2008.03   70500000         NA  5790815204
6   2008.06   15638356         NA 11098677865
7   2008.09  332926664         NA 11705620840
8   2008.12  401777158         NA  6093075622
9   2009.03  505167621         NA 11135171990
10  2009.06  672752955         NA 14040480647
11  2009.09  563375028         NA 14608039919
12  2009.12  754611174         NA  7019677299
13  2010.03  663249619         NA  8464425371
14  2010.06  670122403         NA 13644390979
15  2010.09  734878924         NA 14717211161
16  2010.12  694355548         NA  7595046012
17  2011.03  773227000         NA 14640171827
18  2011.06  743366000         NA 17297683586
19  2011.09  610507000         NA 14351258818

$롯데관광개발
   구....분 광고선전비 교육훈련비      매출액
1   2006.03 1036504876    6881040  8989686669
2   2006.06 2045872542    5399250 10486867068
3   2006.09 2739658080   15348250 14941497865
4   2006.12 1254813126   21589250 12175786065
5   2007.03 1195205643   27700180 12366948305
6   2007.06 1309156992   26722010 11500679409
7   2007.09 1062194930   32002591 16877064858
8   2007.12  986694351   19226091 10845317936
9   2008.03  977711617   23188682 12954724212
10  2008.06  984506080   24090040 11001911869
11  2008.09 1158096243   36495000 10797072664
12  2008.12  486885406   25375633  6545406791
13  2009.03  477624164   19607350  5582468824
14  2009.06  569612024   20698260  7088618820
15  2009.09  600097366   28783950  7726065331
16  2009.12  546760303   18322830  5515137791
17  2010.03  660158348   28187000  7761111255
18  2010.06  613649450   17553344  8633822363
19  2010.09  609214711   30269450 12561340891
20  2010.12  665034444   18269420  9821126136
21  2011.03  607102000          0  9707491187
22  2011.06  670256000   58419000 11048769593
23  2011.09  796352000   27963000 13488179269

$자유투어
   구....분 광고선전비 교육훈련비      매출액
1   2001.09    3360000    5544980  4696875020
2   2001.12    1620000    2584050  8194414929
3   2002.03   27836700   24824070  2846210082
4   2002.06    7883500    5848337  5351832866
5   2002.09   -2487500    1392550  3061660200
6   2002.12   20300000    3866420  1726503925
7   2003.03   16939000   10214408  1189291101
8   2003.06    7540000    3859000  2233223000
9   2003.09    7364000    1955000  1014622000
10  2003.12    1799636    2518441  3207216649
11  2004.03    8724000    3903000  1802926000
12  2004.06    7234000     540000  1986166000
13  2004.09    4484000     581000  1066335000
14  2004.12    8747995     459883  5431549827
15  2005.03    3140000     160000   203880000
16  2005.06          0    2974000   464884000
17  2005.09    1035000     500000   357113000
18  2005.12  594003623     489750  2622344849
19  2006.03  844918000     400000  3761433000
20  2006.06 1159472000     184000  3316223000
21  2006.09 1640866000     720000  4112823000
22  2006.12 1147109888     914990  3657119429
23  2007.03 1167079621     180000  4505787439
24  2007.06 1193941000     100000  3820233000
25  2007.09 1105316438          0  5894269520
26  2007.12 1007781948     620000  4010276376
27  2008.03  975582113     100000  4803096281
28  2008.06 1043178000    2090000  4483386000
29  2008.09 1310815272    2480800  4250004693
30  2008.12  510565418    1347200  2469971423
31  2009.03  653303451    1066590  4823797414
32  2009.06  565423497          0  2740884428
33  2009.09  650179780    -706010  7022767720
34  2009.12  740856031     514920 11683666130
35  2010.03  817154204     516680 11317681906
36  2010.06  914617215     364020  6840916664
37  2010.09  932169708     852300 10455493093
38  2010.12 1020382617    1204920  5568666350
39  2011.03 1025380000     796000  8350833119
40  2011.06  906785000     820000  6310593776
41  2011.09 2857454000    1669000  5680437873

$비티앤아이
   구....분 광고선전비 교육훈련비     매출액
1   2001.03  119792000    8233359 2955754046
2   2001.06  271838219   12481915 2717441792
3   2001.09  113744005   13981556 2192595823
4   2001.12  125040916    8300270 2639250179
5   2002.03   92779137   13502440 1742604837
6   2002.06  253821435    8046800 2489897462
7   2002.09  114540480   10029658 2086142031
8   2002.12  115121935    5321615 2830037688
9   2003.03  106609957   11754857 1786877110
10  2003.06  160728146   14233740 2135968443
11  2003.09   86676615   17696670 1846643602
12  2003.12  180364176   14576780 2711030930
13  2004.03  107319947    7782450 1483498874
14  2004.06  178475843   16184590 2240846626
15  2004.09   92313337   19306137 2600847541
16  2004.12  209758381   14388830 2531327791
17  2005.03  184206999    6283925 2201535575
18  2005.06  220888403   12288850 2747664546
19  2005.09  145568323   13268710 2346511534
20  2005.12  263592370   15133596 2804331806
21  2006.03  130276877    8108340 1482134189
22  2006.06  193554491   11101620 1893886573
23  2006.09  119956546    6036140 1791486824
24  2006.12  286486447    3967950 2406356544
25  2007.03   65900692    3990710 1180081629
26  2007.06  207270104    6455710 1411557360
27  2007.09  133377642    8215550 1185888133
28  2007.12  206995961    5550920 1450245912
29  2008.03  106548846    2848200 1012965573
30  2008.06  230633746    6965970 1274014295
31  2008.09  205953497   13846645 3704152805
32  2008.12  375688451    3531000 3474731068
33  2009.03   81561788   13427260 2542768350
34  2009.06  122025637     260000 2489723263
35  2009.09  120160963     720000 2677270912
36  2009.12  173144605          0 3465555998
37  2010.03  116821788    2240000 2270390985
38  2010.06  209095000   25614000 3770621023
39  2010.09  148188000    6260000 4986588998
40  2010.12  165032212    5740000 1431440795
41  2011.03  180268000    1300000 5490062976
42  2011.06  198959000     598000 3262398180
43  2011.09  144049000     500000 1392998665

> 
\end{Soutput}
\end{Schunk}

\paragraph{리스트로 불러들인 데이터를 하나로 합치기:}
모든 데이터가 성공적으로 불러들여왔음을 확인할 수 있었습니다.
또한 모든 워크시트는 동일한 개수의 변수명 목록을 가지고 있으며, 이들은 모두 같은 변수명을 가집니다.
그런데, 이 데이터는 현재 리스트라는 데이터형식에 들어있습니다.
분석을 위해서는 데이터프레임에 하나로 통합된 데이터가 좋을 것입니다.

따라서, 아래와 같이 수행합니다. 

\begin{Schunk}
\begin{Soutput}
> mydata <- do.call(rbind, tmp)
> head(mydata)
           구....분 광고선전비 교육훈련비      매출액
하나투어.1  2000.12  161702806   18002000  5616224889
하나투어.2  2001.03   80485618   28146500  7188763335
하나투어.3  2001.06  170827271   12965900  7948588645
하나투어.4  2001.09   65667863   26468000 11509839298
하나투어.5  2001.12   27804868   16838062  7799015935
하나투어.6  2002.03   81945640   12752112 10491229385
> 
> tail(mydata)
              구....분 광고선전비 교육훈련비     매출액
비티앤아이.38  2010.06  209095000   25614000 3770621023
비티앤아이.39  2010.09  148188000    6260000 4986588998
비티앤아이.40  2010.12  165032212    5740000 1431440795
비티앤아이.41  2011.03  180268000    1300000 5490062976
비티앤아이.42  2011.06  198959000     598000 3262398180
비티앤아이.43  2011.09  144049000     500000 1392998665
> 
\end{Soutput}
\end{Schunk}

이제서야 하나로 잘 정리된 데이터로 만들어졌습니다.

\paragraph{변수명 바꾸기}
그런데, 첫번째 변수명이 원본데이터에서는 ``구    분'' 이라고 되어 있으나,  불러들인 데이터에서는 ``구....분''이라고 되어 있습니다. 
이는 XLConnect 패키지에서 변수명을 처리할때 빈공간 (화이트 스페이스)를 ... 으로 대체했기 때문입니다.
점 하나가 스페이스 하나입니다.
그런데, 생각을 해보니 ``구분'' 이라는 변수명이 데이터를 표현하는데 적절하지 않은 것 같습니다.
이 변수의 값들은 날짜를 의미하기 때문에 ``분기'' 라고 변경하는 것이 더욱 적절할 것입니다.

그래서, 아래와 같이 변수명을 변경합니다.

\begin{Schunk}
\begin{Soutput}
> names(mydata)[1] <- c("년도별분기")
> names(mydata)
[1] "년도별분기" "광고선전비" "교육훈련비" "매출액"    
> 
\end{Soutput}
\end{Schunk}

\paragraph{데이터 구조확인:}
이제 데이터의 구조를 살펴봅니다.

\begin{Schunk}
\begin{Soutput}
> str(mydata)
'data.frame':	289 obs. of  4 variables:
 $ 년도별분기: num  2000 2001 2001 2001 2001 ...
 $ 광고선전비: num  1.62e+08 8.05e+07 1.71e+08 6.57e+07 2.78e+07 ...
 $ 교육훈련비: num  18002000 28146500 12965900 26468000 16838062 ...
 $ 매출액    : num  5.62e+09 7.19e+09 7.95e+09 1.15e+10 7.80e+09 ...
> 
\end{Soutput}
\end{Schunk}

총 289개의 관측치가 4개의 변수로부터 측정되었음을 확인할 수 있었습니다. 
그런데, 데이터형이 data.frame입니다.  
그 이유는 이전에  do.call()를 이용하여 한데 묶었기 때문입니다.
정말 데이터프레임일까요?  이전에 설명했듯이 데이터프레임은 리스트의 특수한 경우이기 때문입니다.

\begin{Schunk}
\begin{Soutput}
> is.data.frame(mydata)
[1] TRUE
> is.list(mydata)
[1] TRUE
\end{Soutput}
\end{Schunk}

\paragraph{중복을 확인하기:}
그런데, 합쳐진 데이터 mydata를 다시 살펴보니 데이터가 어떤 워크시트로부터 몇 번째 데이터인지를 구분해주는 지시자가 없습니다.
이 지시자의 특징은 각 행별로 절대로 중복이 없는 유일한 값이어야 한다는 점입니다.
이것을 우리는 프라이머리키(primary key)라고 합니다. 
그런데, mydata의 행이름을 보니, 이 정보를 포함하고 있습니다.
먼저, mydata의 행의 이름이 어떠한 중복이 있는지 확인해 봅니다.

\begin{Schunk}
\begin{Soutput}
> rownames(mydata)
  [1] "하나투어.1"      "하나투어.2"      "하나투어.3"      "하나투어.4"     
  [5] "하나투어.5"      "하나투어.6"      "하나투어.7"      "하나투어.8"     
  [9] "하나투어.9"      "하나투어.10"     "하나투어.11"     "하나투어.12"    
 [13] "하나투어.13"     "하나투어.14"     "하나투어.15"     "하나투어.16"    
 [17] "하나투어.17"     "하나투어.18"     "하나투어.19"     "하나투어.20"    
 [21] "하나투어.21"     "하나투어.22"     "하나투어.23"     "하나투어.24"    
 [25] "하나투어.25"     "하나투어.26"     "하나투어.27"     "하나투어.28"    
 [29] "하나투어.29"     "하나투어.30"     "하나투어.31"     "하나투어.32"    
 [33] "하나투어.33"     "하나투어.34"     "하나투어.35"     "하나투어.36"    
 [37] "하나투어.37"     "하나투어.38"     "하나투어.39"     "하나투어.40"    
 [41] "하나투어.41"     "하나투어.42"     "하나투어.43"     "하나투어.44"    
 [45] "레드캡투어.1"    "레드캡투어.2"    "레드캡투어.3"    "레드캡투어.4"   
 [49] "레드캡투어.5"    "레드캡투어.6"    "레드캡투어.7"    "레드캡투어.8"   
 [53] "레드캡투어.9"    "레드캡투어.10"   "레드캡투어.11"   "레드캡투어.12"  
 [57] "레드캡투어.13"   "레드캡투어.14"   "레드캡투어.15"   "레드캡투어.16"  
 [61] "레드캡투어.17"   "레드캡투어.18"   "레드캡투어.19"   "레드캡투어.20"  
 [65] "레드캡투어.21"   "레드캡투어.22"   "레드캡투어.23"   "레드캡투어.24"  
 [69] "레드캡투어.25"   "레드캡투어.26"   "레드캡투어.27"   "레드캡투어.28"  
 [73] "레드캡투어.29"   "레드캡투어.30"   "레드캡투어.31"   "레드캡투어.32"  
 [77] "레드캡투어.33"   "레드캡투어.34"   "레드캡투어.35"   "레드캡투어.36"  
 [81] "레드캡투어.37"   "레드캡투어.38"   "레드캡투어.39"   "레드캡투어.40"  
 [85] "레드캡투어.41"   "레드캡투어.42"   "레드캡투어.43"   "레드캡투어.44"  
 [89] "레드캡투어.45"   "레드캡투어.46"   "레드캡투어.47"   "모두투어.1"     
 [93] "모두투어.2"      "모두투어.3"      "모두투어.4"      "모두투어.5"     
 [97] "모두투어.6"      "모두투어.7"      "모두투어.8"      "모두투어.9"     
[101] "모두투어.10"     "모두투어.11"     "모두투어.12"     "모두투어.13"    
[105] "모두투어.14"     "모두투어.15"     "모두투어.16"     "모두투어.17"    
[109] "모두투어.18"     "모두투어.19"     "모두투어.20"     "모두투어.21"    
[113] "모두투어.22"     "모두투어.23"     "모두투어.24"     "모두투어.25"    
[117] "모두투어.26"     "모두투어.27"     "세중.1"          "세중.2"         
[121] "세중.3"          "세중.4"          "세중.5"          "세중.6"         
[125] "세중.7"          "세중.8"          "세중.9"          "세중.10"        
[129] "세중.11"         "세중.12"         "세중.13"         "세중.14"        
[133] "세중.15"         "세중.16"         "세중.17"         "세중.18"        
[137] "세중.19"         "세중.20"         "세중.21"         "세중.22"        
[141] "세중.23"         "세중.24"         "세중.25"         "세중.26"        
[145] "세중.27"         "세중.28"         "세중.29"         "세중.30"        
[149] "세중.31"         "세중.32"         "세중.33"         "세중.34"        
[153] "세중.35"         "세중.36"         "세중.37"         "세중.38"        
[157] "세중.39"         "세중.40"         "세중.41"         "세중.42"        
[161] "세중.43"         "세중.44"         "세중.45"         "참좋은레져.1"   
[165] "참좋은레져.2"    "참좋은레져.3"    "참좋은레져.4"    "참좋은레져.5"   
[169] "참좋은레져.6"    "참좋은레져.7"    "참좋은레져.8"    "참좋은레져.9"   
[173] "참좋은레져.10"   "참좋은레져.11"   "참좋은레져.12"   "참좋은레져.13"  
[177] "참좋은레져.14"   "참좋은레져.15"   "참좋은레져.16"   "참좋은레져.17"  
[181] "참좋은레져.18"   "참좋은레져.19"   "롯데관광개발.1"  "롯데관광개발.2" 
[185] "롯데관광개발.3"  "롯데관광개발.4"  "롯데관광개발.5"  "롯데관광개발.6" 
[189] "롯데관광개발.7"  "롯데관광개발.8"  "롯데관광개발.9"  "롯데관광개발.10"
[193] "롯데관광개발.11" "롯데관광개발.12" "롯데관광개발.13" "롯데관광개발.14"
[197] "롯데관광개발.15" "롯데관광개발.16" "롯데관광개발.17" "롯데관광개발.18"
[201] "롯데관광개발.19" "롯데관광개발.20" "롯데관광개발.21" "롯데관광개발.22"
[205] "롯데관광개발.23" "자유투어.1"      "자유투어.2"      "자유투어.3"     
[209] "자유투어.4"      "자유투어.5"      "자유투어.6"      "자유투어.7"     
[213] "자유투어.8"      "자유투어.9"      "자유투어.10"     "자유투어.11"    
[217] "자유투어.12"     "자유투어.13"     "자유투어.14"     "자유투어.15"    
[221] "자유투어.16"     "자유투어.17"     "자유투어.18"     "자유투어.19"    
[225] "자유투어.20"     "자유투어.21"     "자유투어.22"     "자유투어.23"    
[229] "자유투어.24"     "자유투어.25"     "자유투어.26"     "자유투어.27"    
[233] "자유투어.28"     "자유투어.29"     "자유투어.30"     "자유투어.31"    
[237] "자유투어.32"     "자유투어.33"     "자유투어.34"     "자유투어.35"    
[241] "자유투어.36"     "자유투어.37"     "자유투어.38"     "자유투어.39"    
[245] "자유투어.40"     "자유투어.41"     "비티앤아이.1"    "비티앤아이.2"   
[249] "비티앤아이.3"    "비티앤아이.4"    "비티앤아이.5"    "비티앤아이.6"   
[253] "비티앤아이.7"    "비티앤아이.8"    "비티앤아이.9"    "비티앤아이.10"  
[257] "비티앤아이.11"   "비티앤아이.12"   "비티앤아이.13"   "비티앤아이.14"  
[261] "비티앤아이.15"   "비티앤아이.16"   "비티앤아이.17"   "비티앤아이.18"  
[265] "비티앤아이.19"   "비티앤아이.20"   "비티앤아이.21"   "비티앤아이.22"  
[269] "비티앤아이.23"   "비티앤아이.24"   "비티앤아이.25"   "비티앤아이.26"  
[273] "비티앤아이.27"   "비티앤아이.28"   "비티앤아이.29"   "비티앤아이.30"  
[277] "비티앤아이.31"   "비티앤아이.32"   "비티앤아이.33"   "비티앤아이.34"  
[281] "비티앤아이.35"   "비티앤아이.36"   "비티앤아이.37"   "비티앤아이.38"  
[285] "비티앤아이.39"   "비티앤아이.40"   "비티앤아이.41"   "비티앤아이.42"  
[289] "비티앤아이.43"  
\end{Soutput}
\end{Schunk}

그런데 이렇게 일일이 눈으로는 확인할 수 없지 않겠나... 하는 생각이 불현듯 떠오릅니다.
그래서 중복이 있고 없고를 한번에 알 수 있는 길이 없을까 하는 생각을 합니다. 

\begin{Schunk}
\begin{Soutput}
> all(!duplicated(rownames(mydata)))
[1] TRUE
\end{Soutput}
\end{Schunk}

duplicated()라는 함수는 중복을 체크하여 TRUE 또는 FALSE를 알려줍니다.
!(느낌표)는 반대라는 의미를 나타내는 연산자입니다.
즉, !duplicated()란 중복이 없나요? 를 물어보는 것입니다.
그리고 all()이라는 함수는 벡터내에 있는 값이 모두 TRUE인지를 확인해줍니다.
이 결과가 TRUE이므로 행의 이름이 중복이 되지 않았음을 확인하였습니다.
따라서, 이 정보를 프라이머리 키로 사용해도 될 것 같습니다.

\paragraph{문자열를 주어진 문자를 이용하여 분리하기:} 
그런데 생각을 해보니 여행사별로 분석을 수행할 수 있는데 이를 구분해 줄 수 있는 변수가 없습니다.
따라서, ``여행사''라는 변수를 새로 만들어 mydata 데이터셋에 넣고자 합니다.
이를 수행하기 위해서는 strpsplit() 함수를 이용하여 아래와 같이 행이름의 문자열을 어떤 특수한 문자에 의해서 나누어 주는 것입니다.

\begin{Schunk}
\begin{Soutput}
> head(strsplit(rownames(mydata1), ".", fixed=TRUE))
[[1]]
[1] "하나투어" "1"       

[[2]]
[1] "하나투어" "2"       

[[3]]
[1] "하나투어" "3"       

[[4]]
[1] "하나투어" "4"       

[[5]]
[1] "하나투어" "5"       

[[6]]
[1] "하나투어" "6"       

>
\end{Soutput}
\end{Schunk}

그리고, 이렇게 리스트로 쪼개어진 변수명을 do.call()함수를 이용하여 행렬의 형태로 재조합한 것을 활용하는 것입니다. 

\begin{Schunk}
\begin{Soutput}
> head(do.call(rbind, strsplit(rownames(mydata), ".", fixed=TRUE)))
     [,1]       [,2]
[1,] "하나투어" "1" 
[2,] "하나투어" "2" 
[3,] "하나투어" "3" 
[4,] "하나투어" "4" 
[5,] "하나투어" "5" 
[6,] "하나투어" "6" 
> 
\end{Soutput}
\end{Schunk}

\paragraph{데이터프레임에 변수 추가하기}

그리고 여행사라는 변수를 생성합니다.
행이름은 더이상 필요하지 않으므로 삭제합니다.

\begin{Schunk}
\begin{Soutput}
> mydata$"여행사" <- do.call(rbind, strsplit(rownames(mydata), ".", fixed=TRUE))[,1]
> mydata$"번호" <- do.call(rbind, strsplit(rownames(mydata), ".", fixed=TRUE))[,2]
> rownames(mydata) <- NULL
> head(mydata)
  년도별분기 광고선전비 교육훈련비      매출액   여행사 번호
1    2000.12  161702806   18002000  5616224889 하나투어    1
2    2001.03   80485618   28146500  7188763335 하나투어    2
3    2001.06  170827271   12965900  7948588645 하나투어    3
4    2001.09   65667863   26468000 11509839298 하나투어    4
5    2001.12   27804868   16838062  7799015935 하나투어    5
6    2002.03   81945640   12752112 10491229385 하나투어    6
> 
\end{Soutput}
\end{Schunk}

이러한 방법으로 년도별 분기 변수를 좀 더 상세화 할 수 있을 것입니다.

\begin{Schunk}
\begin{Soutput}
> yrQ <- as.data.frame(do.call(rbind, strsplit(as.character(mydata$"년도별분기"), ".", fixed=TRUE)))
> names(yrQ) <- c("년도", "월")
> mydata <- data.frame(mydata, yrQ)
> head(mydata)
  년도별분기 광고선전비 교육훈련비      매출액   여행사 번호 년도 월
1    2000.12  161702806   18002000  5616224889 하나투어    1 2000 12
2    2001.03   80485618   28146500  7188763335 하나투어    2 2001 03
3    2001.06  170827271   12965900  7948588645 하나투어    3 2001 06
4    2001.09   65667863   26468000 11509839298 하나투어    4 2001 09
5    2001.12   27804868   16838062  7799015935 하나투어    5 2001 12
6    2002.03   81945640   12752112 10491229385 하나투어    6 2002 03
> 
\end{Soutput}
\end{Schunk}
% $

\paragraph{결측치 확인하고 제거하기}
그런데, 데이터에 결측치들이 얼마나 있는지 살펴보아야 할 것입니다.
만약 있다면 어디에서 어떤 변수에서 결측치가 있으며, 이들을 삭제할 것인지 결정해야 합니다. 
그래서 원본데이터 tmp를 살펴보았더니, 아래와 같이 NA 가 있습니다. 

\begin{Schunk}
\begin{Soutput}
> tmp$"참좋은레져"
   구....분 광고선전비 교육훈련비      매출액
1   2007.03     500000          0  2915134989
2   2007.06    2836412          0  6782580656
3   2007.09   67680500         NA  4926211503
4   2007.12   24490909         NA  3857238621
5   2008.03   70500000         NA  5790815204
6   2008.06   15638356         NA 11098677865
7   2008.09  332926664         NA 11705620840
8   2008.12  401777158         NA  6093075622
9   2009.03  505167621         NA 11135171990
10  2009.06  672752955         NA 14040480647
11  2009.09  563375028         NA 14608039919
12  2009.12  754611174         NA  7019677299
13  2010.03  663249619         NA  8464425371
14  2010.06  670122403         NA 13644390979
15  2010.09  734878924         NA 14717211161
16  2010.12  694355548         NA  7595046012
17  2011.03  773227000         NA 14640171827
18  2011.06  743366000         NA 17297683586
19  2011.09  610507000         NA 14351258818
> 
\end{Soutput}
\end{Schunk}
%$

그럼 하나로 뭉친 mydata 파일에서 어떻게 이러한 데이터를 찾아야 할까요?
is.na() 함수의 사용은 아래와 같은 결과를 줍니다. 

\begin{Schunk}
\begin{Soutput}
> head(is.na(mydata))
     년도별분기 광고선전비 교육훈련비 매출액 여행사  번호  년도    월
[1,]      FALSE      FALSE      FALSE  FALSE  FALSE FALSE FALSE FALSE
[2,]      FALSE      FALSE      FALSE  FALSE  FALSE FALSE FALSE FALSE
[3,]      FALSE      FALSE      FALSE  FALSE  FALSE FALSE FALSE FALSE
[4,]      FALSE      FALSE      FALSE  FALSE  FALSE FALSE FALSE FALSE
[5,]      FALSE      FALSE      FALSE  FALSE  FALSE FALSE FALSE FALSE
[6,]      FALSE      FALSE      FALSE  FALSE  FALSE FALSE FALSE FALSE
> 
\end{Soutput}
\end{Schunk}

그렇다면 TRUE 라고 된 부분이 결측일 것입니다.  
데이터를 한 눈에 살펴볼 수 없기 때문에 아래와 같이 합니다. 

\begin{Schunk}
\begin{Soutput}
> idx <- which(is.na(mydata))
> mydata[idx%%nrow(mydata), ]
    년도별분기 광고선전비 교육훈련비      매출액     여행사 번호 년도 월
52     2001.12         NA          0  4096816401 레드캡투어    8 2001 12
166    2007.09   67680500         NA  4926211503 참좋은레져    3 2007 09
167    2007.12   24490909         NA  3857238621 참좋은레져    4 2007 12
168    2008.03   70500000         NA  5790815204 참좋은레져    5 2008 03
169    2008.06   15638356         NA 11098677865 참좋은레져    6 2008 06
170    2008.09  332926664         NA 11705620840 참좋은레져    7 2008 09
171    2008.12  401777158         NA  6093075622 참좋은레져    8 2008 12
172    2009.03  505167621         NA 11135171990 참좋은레져    9 2009 03
173    2009.06  672752955         NA 14040480647 참좋은레져   10 2009 06
174    2009.09  563375028         NA 14608039919 참좋은레져   11 2009 09
175    2009.12  754611174         NA  7019677299 참좋은레져   12 2009 12
176    2010.03  663249619         NA  8464425371 참좋은레져   13 2010 03
177    2010.06  670122403         NA 13644390979 참좋은레져   14 2010 06
178    2010.09  734878924         NA 14717211161 참좋은레져   15 2010 09
179    2010.12  694355548         NA  7595046012 참좋은레져   16 2010 12
180    2011.03  773227000         NA 14640171827 참좋은레져   17 2011 03
181    2011.06  743366000         NA 17297683586 참좋은레져   18 2011 06
182    2011.09  610507000         NA 14351258818 참좋은레져   19 2011 09
\end{Soutput}
\end{Schunk}

이와 같은 논리를 이용하여 R은 결측치에 해당하는 레코드를 지워주는 na.exclude()라는 함수를 제공합니다. 

\begin{Schunk}
\begin{Soutput}
> mydatax <- na.exclude(mydata)
> mydatax[163:170, ]
    년도별분기 광고선전비 교육훈련비      매출액       여행사 번호 년도 월
164    2007.03     500000          0  2915134989   참좋은레져    1 2007 03
165    2007.06    2836412          0  6782580656   참좋은레져    2 2007 06
183    2006.03 1036504876    6881040  8989686669 롯데관광개발    1 2006 03
184    2006.06 2045872542    5399250 10486867068 롯데관광개발    2 2006 06
185    2006.09 2739658080   15348250 14941497865 롯데관광개발    3 2006 09
186    2006.12 1254813126   21589250 12175786065 롯데관광개발    4 2006 12
187    2007.03 1195205643   27700180 12366948305 롯데관광개발    5 2007 03
188    2007.06 1309156992   26722010 11500679409 롯데관광개발    6 2007 06
> 
\end{Soutput}
\end{Schunk}

\paragraph{데이터프레임에서 변수삭제하기:}
``년도''와 ``월''이라는 변수를 따로 생성하였기 때문에 이제 ``년도별분기''라는 변수는 불필요하므로 변수를 삭제하도록 합니다.

\begin{Schunk}
\begin{Soutput}
> mydatax <- mydatax[c(FALSE, rep(TRUE, 7))]
> head(mydatax)
  광고선전비 교육훈련비      매출액   여행사 번호 년도 월
1  161702806   18002000  5616224889 하나투어    1 2000 12
2   80485618   28146500  7188763335 하나투어    2 2001 03
3  170827271   12965900  7948588645 하나투어    3 2001 06
4   65667863   26468000 11509839298 하나투어    4 2001 09
5   27804868   16838062  7799015935 하나투어    5 2001 12
6   81945640   12752112 10491229385 하나투어    6 2002 03
> 
\end{Soutput}
\end{Schunk}
 
그러고 보니, ``월''이라는 변수는 분기별로 데이터를 모은 것이므로 ``분기''로 변형하는 것이 좋을 듯 합니다.
먼저, ``월''이라는 변수가 정말 3,6,9,12 월에 해당하는 값들만 가지고 있는지 확인을 해야할 것입니다.

\begin{Schunk}
\begin{Soutput}
> names(table(mydatax$"월"))
[1] "03" "06" "09" "12"
\end{Soutput}
\end{Schunk}
%$ 

따라서, ``월''이라는 변수를 ``분기''라는 변수로 변경합니다.
또한, 문자형을 요인형으로 변경하면서, 수준에 따라 라벨링을 함께 합니다.

\begin{Schunk}
\begin{Soutput}
> mydatax$"월" <- factor(mydatax$"월", levels=c("03", "06", "09", "12"), labels=c("1분기", "2분기", "3분기", "4분기"))
> names(mydatax)[7] <- c("분기")
> head(mydatax)
  광고선전비 교육훈련비      매출액   여행사 번호 년도  분기
1  161702806   18002000  5616224889 하나투어    1 2000 4분기
2   80485618   28146500  7188763335 하나투어    2 2001 1분기
3  170827271   12965900  7948588645 하나투어    3 2001 2분기
4   65667863   26468000 11509839298 하나투어    4 2001 3분기
5   27804868   16838062  7799015935 하나투어    5 2001 4분기
6   81945640   12752112 10491229385 하나투어    6 2002 1분기
> 
\end{Soutput}
\end{Schunk}

\paragraph{분할표 생성해보기:}
이제 간단한 분기와 년도에 따른 contingency table을 생성해봅니다.

\begin{Schunk}
\begin{Soutput}
> ftable(mydatax$"분기", mydatax$"년도")
       2000 2001 2002 2003 2004 2005 2006 2007 2008 2009 2010 2011
                                                                  
1분기     1    4    5    5    5    6    7    8    7    7    7    7
2분기     1    4    5    5    5    6    7    8    7    7    7    7
3분기     2    5    5    5    5    6    7    7    7    7    7    7
4분기     3    4    5    5    5    6    7    7    7    7    7    0
> 
\end{Soutput}
\end{Schunk}


이 분할표를 여행사별로 출력해봅니다.

\begin{Schunk}
\begin{Soutput}
> ftable(mydatax$"여행사", mydatax$"분기", mydatax$"년도")
                    2000 2001 2002 2003 2004 2005 2006 2007 2008 2009 2010 2011
                                                                               
세중         1분기     0    1    1    1    1    1    1    1    1    1    1    1
             2분기     0    1    1    1    1    1    1    1    1    1    1    1
             3분기     1    1    1    1    1    1    1    1    1    1    1    1
             4분기     1    1    1    1    1    1    1    1    1    1    1    0
하나투어     1분기     0    1    1    1    1    1    1    1    1    1    1    1
             2분기     0    1    1    1    1    1    1    1    1    1    1    1
             3분기     0    1    1    1    1    1    1    1    1    1    1    1
             4분기     1    1    1    1    1    1    1    1    1    1    1    0
모두투어     1분기     0    0    0    0    0    1    1    1    1    1    1    1
             2분기     0    0    0    0    0    1    1    1    1    1    1    1
             3분기     0    0    0    0    0    1    1    1    1    1    1    1
             4분기     0    0    0    0    0    1    1    1    1    1    1    0
자유투어     1분기     0    0    1    1    1    1    1    1    1    1    1    1
             2분기     0    0    1    1    1    1    1    1    1    1    1    1
             3분기     0    1    1    1    1    1    1    1    1    1    1    1
             4분기     0    1    1    1    1    1    1    1    1    1    1    0
레드캡투어   1분기     1    1    1    1    1    1    1    1    1    1    1    1
             2분기     1    1    1    1    1    1    1    1    1    1    1    1
             3분기     1    1    1    1    1    1    1    1    1    1    1    1
             4분기     1    0    1    1    1    1    1    1    1    1    1    0
참좋은레져   1분기     0    0    0    0    0    0    0    1    0    0    0    0
             2분기     0    0    0    0    0    0    0    1    0    0    0    0
             3분기     0    0    0    0    0    0    0    0    0    0    0    0
             4분기     0    0    0    0    0    0    0    0    0    0    0    0
비티앤아이   1분기     0    1    1    1    1    1    1    1    1    1    1    1
             2분기     0    1    1    1    1    1    1    1    1    1    1    1
             3분기     0    1    1    1    1    1    1    1    1    1    1    1
             4분기     0    1    1    1    1    1    1    1    1    1    1    0
롯데관광개발 1분기     0    0    0    0    0    0    1    1    1    1    1    1
             2분기     0    0    0    0    0    0    1    1    1    1    1    1
             3분기     0    0    0    0    0    0    1    1    1    1    1    1
             4분기     0    0    0    0    0    0    1    1    1    1    1    0
> 

\end{Soutput}
\end{Schunk}
%$


매번 데이터셋이름을 같이 쓰기가 너무 불편합니다.
따라서, 아래와 같이 with()를 사용해봅니다.

\begin{Schunk}
\begin{Soutput}
> with(mydatax, ftable(분기, 년도))
      년도 2000 2001 2002 2003 2004 2005 2006 2007 2008 2009 2010 2011
분기                                                                  
1분기         1    4    5    5    5    6    7    8    7    7    7    7
2분기         1    4    5    5    5    6    7    8    7    7    7    7
3분기         2    5    5    5    5    6    7    7    7    7    7    7
4분기         3    4    5    5    5    6    7    7    7    7    7    0
> 
\end{Soutput}
\end{Schunk}


\paragraph{데이터의 선택적 부분지정:}
이제 ``하나투어''에 해당하는 자료를 뽑고, 그 중에서도 ``4분기''에 해당하는 레코드를 뽑고자 합니다.

\begin{Schunk}
\begin{Soutput}
> subset(x=mydatax, subset=(여행사=="하나투어" & 분기=="4분기"))
   광고선전비 교육훈련비      매출액   여행사 번호 년도  분기
1   161702806   18002000  5616224889 하나투어    1 2000 4분기
5    27804868   16838062  7799015935 하나투어    5 2001 4분기
9   528533759    8138020 12683418184 하나투어    9 2002 4분기
13  438942853   14812754 15647266644 하나투어   13 2003 4분기
17 1034522413   54309865 19861453701 하나투어   17 2004 4분기
21 1372387152   62163640 27122123547 하나투어   21 2005 4분기
25 2393203854  102367110 46035346970 하나투어   25 2006 4분기
29 1771196490   40011868 46418905817 하나투어   29 2007 4분기
33  727493963   -3735654 27729106510 하나투어   33 2008 4분기
37  528153534  -20830938 28792370983 하나투어   37 2009 4분기
41 2930265435   98244331 54941857566 하나투어   41 2010 4분기
> 
\end{Soutput}
\end{Schunk}


위에서 조건에 맞는 레코드들을 추출했지만, 변수가 모두 필요한 것은 아닙니다.
따라서, 교육훈련비, 년도, 분기 세가지 변수만 뽑아봅니다.

\begin{Schunk}
\begin{Soutput}
> subset(x=mydatax, subset=(여행사=="하나투어" & 분기=="4분기"), select=c(교육훈련비, 년도, 분기))
   교육훈련비 년도  분기
1    18002000 2000 4분기
5    16838062 2001 4분기
9     8138020 2002 4분기
13   14812754 2003 4분기
17   54309865 2004 4분기
21   62163640 2005 4분기
25  102367110 2006 4분기
29   40011868 2007 4분기
33   -3735654 2008 4분기
37  -20830938 2009 4분기
41   98244331 2010 4분기
> 
\end{Soutput}
\end{Schunk}

\paragraph{그룹별 연산하기:}
평균교육훈련비를 연도별로 산출한 뒤, 연도별 분할표를 생성해봅니다.

\begin{Schunk}
\begin{Soutput}
> as.data.frame(as.table(with(mydatax, tapply(교육훈련비, 년도, mean))))
   Var1     Freq
1  2000  3328270
2  2001 10320313
3  2002  7423926
4  2003  9416275
5  2004  8417142
6  2005  7416239
7  2006 15046704
8  2007 22127430
9  2008 21516616
10 2009 10721259
11 2010 21786660
12 2011 21945286
> 
\end{Soutput}
\end{Schunk}

\paragraph{연산자 활용:}
여기에서부터는 2008, 2009, 2010 년에서 1분기와 3분기에 해당하는 ``다트8''이라는 데이터를 생성하여 작업하도록 하겠습니다.
그 이유는 단순히 결과를 효과적으로 보여주기 위해서입니다.

\begin{Schunk}
\begin{Soutput}
다트8 <- subset(mydata, subset=(년도 %in% c("2008", "2009", "2010") & 분기 %in% c("1분기", "3분기")))
다트8

> 다트8
    광고선전비 교육훈련비      매출액       여행사 번호 년도  분기
30  1432606264   10453124 57624282255     하나투어   30 2008 1분기
32  1398280115   15837118 43516939794     하나투어   32 2008 3분기
34   528828105   10678454 30625278205     하나투어   34 2009 1분기
36   679725152    5375226 34845539027     하나투어   36 2009 3분기
38  1055742790   44554342 48480421492     하나투어   38 2010 1분기
40  1723098483          0 66245202818     하나투어   40 2010 3분기
77           0          0 19926326125   레드캡투어   33 2008 1분기
79  2748341000  145562000 19269912551   레드캡투어   35 2008 3분기
81   215422000    5027000 20855159928   레드캡투어   37 2009 1분기
83   503662000   30857000 20201843391   레드캡투어   39 2009 3분기
85   285347000    3554000 26266456633   레드캡투어   41 2010 1분기
87   377363000   50701000 28589391364   레드캡투어   43 2010 3분기
104 1204193071   13200204 26530405258     모두투어   13 2008 1분기
106 1460678899   34482480 23136419134     모두투어   15 2008 3분기
108  544481614    4611600 13037790654     모두투어   17 2009 1분기
110  586547017    6552350 17831640182     모두투어   19 2009 3분기
112  627109367   11581130 25003802066     모두투어   21 2010 1분기
114 1265995826   20890280 36466418562     모두투어   23 2010 3분기
149  930137416   29007679 16463319852         세중   31 2008 1분기
151  267514376    9583390 19159107089         세중   33 2008 3분기
153  128959936    1252642 13597706178         세중   35 2009 1분기
155   56881950    6980850 16130093114         세중   37 2009 3분기
157   92244120   15798800 14808015873         세중   39 2010 1분기
159   48997402   -4759221 20282845926         세중   41 2010 3분기
168   70500000         NA  5790815204   참좋은레져    5 2008 1분기
170  332926664         NA 11705620840   참좋은레져    7 2008 3분기
172  505167621         NA 11135171990   참좋은레져    9 2009 1분기
174  563375028         NA 14608039919   참좋은레져   11 2009 3분기
176  663249619         NA  8464425371   참좋은레져   13 2010 1분기
178  734878924         NA 14717211161   참좋은레져   15 2010 3분기
191  977711617   23188682 12954724212 롯데관광개발    9 2008 1분기
193 1158096243   36495000 10797072664 롯데관광개발   11 2008 3분기
195  477624164   19607350  5582468824 롯데관광개발   13 2009 1분기
197  600097366   28783950  7726065331 롯데관광개발   15 2009 3분기
199  660158348   28187000  7761111255 롯데관광개발   17 2010 1분기
201  609214711   30269450 12561340891 롯데관광개발   19 2010 3분기
232  975582113     100000  4803096281     자유투어   27 2008 1분기
234 1310815272    2480800  4250004693     자유투어   29 2008 3분기
236  653303451    1066590  4823797414     자유투어   31 2009 1분기
238  650179780    -706010  7022767720     자유투어   33 2009 3분기
240  817154204     516680 11317681906     자유투어   35 2010 1분기
242  932169708     852300 10455493093     자유투어   37 2010 3분기
275  106548846    2848200  1012965573   비티앤아이   29 2008 1분기
277  205953497   13846645  3704152805   비티앤아이   31 2008 3분기
279   81561788   13427260  2542768350   비티앤아이   33 2009 1분기
281  120160963     720000  2677270912   비티앤아이   35 2009 3분기
283  116821788    2240000  2270390985   비티앤아이   37 2010 1분기
285  148188000    6260000  4986588998   비티앤아이   39 2010 3분기
> 
\end{Soutput}
\end{Schunk}

현재 여행사는 데이터를 그룹화 할 수 있는 고유한 키라고 할 수 있습니다. 

\paragraph{정렬하기:}
그런데, 이 데이터에 특징이 하나 있다면 그것은 동일한 여행사로부터 매년 분기별로 여러번 반복하여 얻은 데이터라는 것입니다. 
따라서, 간혹 데이터를 여행사별로 정렬하기보다는 년도별로 정렬하고 싶을 경우가 있습니다.
이런 경우는 아래와 같이 합니다. 

\begin{Schunk}
\begin{Soutput}
다트8[order(다트8$년도),]

    광고선전비 교육훈련비      매출액       여행사 번호 년도  분기
30  1432606264   10453124 57624282255     하나투어   30 2008 1분기
32  1398280115   15837118 43516939794     하나투어   32 2008 3분기
77           0          0 19926326125   레드캡투어   33 2008 1분기
79  2748341000  145562000 19269912551   레드캡투어   35 2008 3분기
104 1204193071   13200204 26530405258     모두투어   13 2008 1분기
106 1460678899   34482480 23136419134     모두투어   15 2008 3분기
149  930137416   29007679 16463319852         세중   31 2008 1분기
151  267514376    9583390 19159107089         세중   33 2008 3분기
168   70500000         NA  5790815204   참좋은레져    5 2008 1분기
170  332926664         NA 11705620840   참좋은레져    7 2008 3분기
191  977711617   23188682 12954724212 롯데관광개발    9 2008 1분기
193 1158096243   36495000 10797072664 롯데관광개발   11 2008 3분기
232  975582113     100000  4803096281     자유투어   27 2008 1분기
234 1310815272    2480800  4250004693     자유투어   29 2008 3분기
275  106548846    2848200  1012965573   비티앤아이   29 2008 1분기
277  205953497   13846645  3704152805   비티앤아이   31 2008 3분기
34   528828105   10678454 30625278205     하나투어   34 2009 1분기
36   679725152    5375226 34845539027     하나투어   36 2009 3분기
81   215422000    5027000 20855159928   레드캡투어   37 2009 1분기
83   503662000   30857000 20201843391   레드캡투어   39 2009 3분기
108  544481614    4611600 13037790654     모두투어   17 2009 1분기
110  586547017    6552350 17831640182     모두투어   19 2009 3분기
153  128959936    1252642 13597706178         세중   35 2009 1분기
155   56881950    6980850 16130093114         세중   37 2009 3분기
172  505167621         NA 11135171990   참좋은레져    9 2009 1분기
174  563375028         NA 14608039919   참좋은레져   11 2009 3분기
195  477624164   19607350  5582468824 롯데관광개발   13 2009 1분기
197  600097366   28783950  7726065331 롯데관광개발   15 2009 3분기
236  653303451    1066590  4823797414     자유투어   31 2009 1분기
238  650179780    -706010  7022767720     자유투어   33 2009 3분기
279   81561788   13427260  2542768350   비티앤아이   33 2009 1분기
281  120160963     720000  2677270912   비티앤아이   35 2009 3분기
38  1055742790   44554342 48480421492     하나투어   38 2010 1분기
40  1723098483          0 66245202818     하나투어   40 2010 3분기
85   285347000    3554000 26266456633   레드캡투어   41 2010 1분기
87   377363000   50701000 28589391364   레드캡투어   43 2010 3분기
112  627109367   11581130 25003802066     모두투어   21 2010 1분기
114 1265995826   20890280 36466418562     모두투어   23 2010 3분기
157   92244120   15798800 14808015873         세중   39 2010 1분기
159   48997402   -4759221 20282845926         세중   41 2010 3분기
176  663249619         NA  8464425371   참좋은레져   13 2010 1분기
178  734878924         NA 14717211161   참좋은레져   15 2010 3분기
199  660158348   28187000  7761111255 롯데관광개발   17 2010 1분기
201  609214711   30269450 12561340891 롯데관광개발   19 2010 3분기
240  817154204     516680 11317681906     자유투어   35 2010 1분기
242  932169708     852300 10455493093     자유투어   37 2010 3분기
283  116821788    2240000  2270390985   비티앤아이   37 2010 1분기
285  148188000    6260000  4986588998   비티앤아이   39 2010 3분기
> 
\end{Soutput}
\end{Schunk}
% $

\paragraph{중복되는 데이터의 처음과 끝 확인하기}
또 다른 경우는 각 년도별로 첫번째 레코드가 무엇인지 마지막 레코드가 무엇인지 알고 싶을 경우가 있습니다. 
이런 경우는 아래와 같이 할 수 있습니다. 

\begin{Schunk}
\begin{Soutput}
다트8.1 <- 다트8[order(다트8$년도),]
다트8.1$first <- !duplicated(다트8.1$년도)
다트8.1

    광고선전비 교육훈련비      매출액       여행사 번호 년도  분기 first
30  1432606264   10453124 57624282255     하나투어   30 2008 1분기  TRUE
32  1398280115   15837118 43516939794     하나투어   32 2008 3분기 FALSE
77           0          0 19926326125   레드캡투어   33 2008 1분기 FALSE
79  2748341000  145562000 19269912551   레드캡투어   35 2008 3분기 FALSE
104 1204193071   13200204 26530405258     모두투어   13 2008 1분기 FALSE
106 1460678899   34482480 23136419134     모두투어   15 2008 3분기 FALSE
149  930137416   29007679 16463319852         세중   31 2008 1분기 FALSE
151  267514376    9583390 19159107089         세중   33 2008 3분기 FALSE
168   70500000         NA  5790815204   참좋은레져    5 2008 1분기 FALSE
170  332926664         NA 11705620840   참좋은레져    7 2008 3분기 FALSE
191  977711617   23188682 12954724212 롯데관광개발    9 2008 1분기 FALSE
193 1158096243   36495000 10797072664 롯데관광개발   11 2008 3분기 FALSE
232  975582113     100000  4803096281     자유투어   27 2008 1분기 FALSE
234 1310815272    2480800  4250004693     자유투어   29 2008 3분기 FALSE
275  106548846    2848200  1012965573   비티앤아이   29 2008 1분기 FALSE
277  205953497   13846645  3704152805   비티앤아이   31 2008 3분기 FALSE
34   528828105   10678454 30625278205     하나투어   34 2009 1분기  TRUE
36   679725152    5375226 34845539027     하나투어   36 2009 3분기 FALSE
81   215422000    5027000 20855159928   레드캡투어   37 2009 1분기 FALSE
83   503662000   30857000 20201843391   레드캡투어   39 2009 3분기 FALSE
108  544481614    4611600 13037790654     모두투어   17 2009 1분기 FALSE
110  586547017    6552350 17831640182     모두투어   19 2009 3분기 FALSE
153  128959936    1252642 13597706178         세중   35 2009 1분기 FALSE
155   56881950    6980850 16130093114         세중   37 2009 3분기 FALSE
172  505167621         NA 11135171990   참좋은레져    9 2009 1분기 FALSE
174  563375028         NA 14608039919   참좋은레져   11 2009 3분기 FALSE
195  477624164   19607350  5582468824 롯데관광개발   13 2009 1분기 FALSE
197  600097366   28783950  7726065331 롯데관광개발   15 2009 3분기 FALSE
236  653303451    1066590  4823797414     자유투어   31 2009 1분기 FALSE
238  650179780    -706010  7022767720     자유투어   33 2009 3분기 FALSE
279   81561788   13427260  2542768350   비티앤아이   33 2009 1분기 FALSE
281  120160963     720000  2677270912   비티앤아이   35 2009 3분기 FALSE
38  1055742790   44554342 48480421492     하나투어   38 2010 1분기  TRUE
40  1723098483          0 66245202818     하나투어   40 2010 3분기 FALSE
85   285347000    3554000 26266456633   레드캡투어   41 2010 1분기 FALSE
87   377363000   50701000 28589391364   레드캡투어   43 2010 3분기 FALSE
112  627109367   11581130 25003802066     모두투어   21 2010 1분기 FALSE
114 1265995826   20890280 36466418562     모두투어   23 2010 3분기 FALSE
157   92244120   15798800 14808015873         세중   39 2010 1분기 FALSE
159   48997402   -4759221 20282845926         세중   41 2010 3분기 FALSE
176  663249619         NA  8464425371   참좋은레져   13 2010 1분기 FALSE
178  734878924         NA 14717211161   참좋은레져   15 2010 3분기 FALSE
199  660158348   28187000  7761111255 롯데관광개발   17 2010 1분기 FALSE
201  609214711   30269450 12561340891 롯데관광개발   19 2010 3분기 FALSE
240  817154204     516680 11317681906     자유투어   35 2010 1분기 FALSE
242  932169708     852300 10455493093     자유투어   37 2010 3분기 FALSE
283  116821788    2240000  2270390985   비티앤아이   37 2010 1분기 FALSE
285  148188000    6260000  4986588998   비티앤아이   39 2010 3분기 FALSE
> 
\end{Soutput}
\end{Schunk}
% $

이와 유사한 논리로 각 년도별 마지막 레코드를 활용하고자 하는 지시자를 생성할 수도 있습니다. 


\begin{Schunk}
\begin{Soutput}
다트8.1$last <- !duplicated(다트8.1$년도, fromLast=TRUE)
다트8.1

    광고선전비 교육훈련비      매출액       여행사 번호 년도  분기 first  last
30  1432606264   10453124 57624282255     하나투어   30 2008 1분기  TRUE FALSE
32  1398280115   15837118 43516939794     하나투어   32 2008 3분기 FALSE FALSE
77           0          0 19926326125   레드캡투어   33 2008 1분기 FALSE FALSE
79  2748341000  145562000 19269912551   레드캡투어   35 2008 3분기 FALSE FALSE
104 1204193071   13200204 26530405258     모두투어   13 2008 1분기 FALSE FALSE
106 1460678899   34482480 23136419134     모두투어   15 2008 3분기 FALSE FALSE
149  930137416   29007679 16463319852         세중   31 2008 1분기 FALSE FALSE
151  267514376    9583390 19159107089         세중   33 2008 3분기 FALSE FALSE
168   70500000         NA  5790815204   참좋은레져    5 2008 1분기 FALSE FALSE
170  332926664         NA 11705620840   참좋은레져    7 2008 3분기 FALSE FALSE
191  977711617   23188682 12954724212 롯데관광개발    9 2008 1분기 FALSE FALSE
193 1158096243   36495000 10797072664 롯데관광개발   11 2008 3분기 FALSE FALSE
232  975582113     100000  4803096281     자유투어   27 2008 1분기 FALSE FALSE
234 1310815272    2480800  4250004693     자유투어   29 2008 3분기 FALSE FALSE
275  106548846    2848200  1012965573   비티앤아이   29 2008 1분기 FALSE FALSE
277  205953497   13846645  3704152805   비티앤아이   31 2008 3분기 FALSE  TRUE
34   528828105   10678454 30625278205     하나투어   34 2009 1분기  TRUE FALSE
36   679725152    5375226 34845539027     하나투어   36 2009 3분기 FALSE FALSE
81   215422000    5027000 20855159928   레드캡투어   37 2009 1분기 FALSE FALSE
83   503662000   30857000 20201843391   레드캡투어   39 2009 3분기 FALSE FALSE
108  544481614    4611600 13037790654     모두투어   17 2009 1분기 FALSE FALSE
110  586547017    6552350 17831640182     모두투어   19 2009 3분기 FALSE FALSE
153  128959936    1252642 13597706178         세중   35 2009 1분기 FALSE FALSE
155   56881950    6980850 16130093114         세중   37 2009 3분기 FALSE FALSE
172  505167621         NA 11135171990   참좋은레져    9 2009 1분기 FALSE FALSE
174  563375028         NA 14608039919   참좋은레져   11 2009 3분기 FALSE FALSE
195  477624164   19607350  5582468824 롯데관광개발   13 2009 1분기 FALSE FALSE
197  600097366   28783950  7726065331 롯데관광개발   15 2009 3분기 FALSE FALSE
236  653303451    1066590  4823797414     자유투어   31 2009 1분기 FALSE FALSE
238  650179780    -706010  7022767720     자유투어   33 2009 3분기 FALSE FALSE
279   81561788   13427260  2542768350   비티앤아이   33 2009 1분기 FALSE FALSE
281  120160963     720000  2677270912   비티앤아이   35 2009 3분기 FALSE  TRUE
38  1055742790   44554342 48480421492     하나투어   38 2010 1분기  TRUE FALSE
40  1723098483          0 66245202818     하나투어   40 2010 3분기 FALSE FALSE
85   285347000    3554000 26266456633   레드캡투어   41 2010 1분기 FALSE FALSE
87   377363000   50701000 28589391364   레드캡투어   43 2010 3분기 FALSE FALSE
112  627109367   11581130 25003802066     모두투어   21 2010 1분기 FALSE FALSE
114 1265995826   20890280 36466418562     모두투어   23 2010 3분기 FALSE FALSE
157   92244120   15798800 14808015873         세중   39 2010 1분기 FALSE FALSE
159   48997402   -4759221 20282845926         세중   41 2010 3분기 FALSE FALSE
176  663249619         NA  8464425371   참좋은레져   13 2010 1분기 FALSE FALSE
178  734878924         NA 14717211161   참좋은레져   15 2010 3분기 FALSE FALSE
199  660158348   28187000  7761111255 롯데관광개발   17 2010 1분기 FALSE FALSE
201  609214711   30269450 12561340891 롯데관광개발   19 2010 3분기 FALSE FALSE
240  817154204     516680 11317681906     자유투어   35 2010 1분기 FALSE FALSE
242  932169708     852300 10455493093     자유투어   37 2010 3분기 FALSE FALSE
283  116821788    2240000  2270390985   비티앤아이   37 2010 1분기 FALSE FALSE
285  148188000    6260000  4986588998   비티앤아이   39 2010 3분기 FALSE  TRUE
> 
\end{Soutput}
\end{Schunk}

\paragraph{데이터를 종횡과 횡형으로 변형하기}
이렇게 처음과 마지막 레코드를 확인할 수 있는 지시자를 이용하여 어떤 분석자는 ``다트8.1''과 같은 데이터가 주어졌을때, 각 여행사별로 2008년 1분기 매출액과 2010년 4분기의 매출액을 비교하여 그 차이를 알아내기 위해서 아래와 같은 데이터를 조작할 수 있습니다.

\begin{Schunk}
\begin{Soutput}
다트8.2 <- 다트8.1[c(FALSE, FALSE, TRUE, TRUE, FALSE, TRUE, FALSE, FALSE, FALSE)]
다트8.3 <- 다트8.2[order(다트8.2$여행사, 다트8.2$년도), ]
다트8.3$first <- !duplicated(다트8.3$여행사)
다트8.3$last <- !duplicated(다트8.3$여행사, fromLast=TRUE)
다트8.4 <- subset(다트8.3, subset=(first == TRUE | last == TRUE))

> 다트8.4
         매출액       여행사 년도 first  last
149 16463319852         세중 2008  TRUE FALSE
159 20282845926         세중 2010 FALSE  TRUE
30  57624282255     하나투어 2008  TRUE FALSE
40  66245202818     하나투어 2010 FALSE  TRUE
104 26530405258     모두투어 2008  TRUE FALSE
114 36466418562     모두투어 2010 FALSE  TRUE
232  4803096281     자유투어 2008  TRUE FALSE
242 10455493093     자유투어 2010 FALSE  TRUE
77  19926326125   레드캡투어 2008  TRUE FALSE
87  28589391364   레드캡투어 2010 FALSE  TRUE
168  5790815204   참좋은레져 2008  TRUE FALSE
178 14717211161   참좋은레져 2010 FALSE  TRUE
275  1012965573   비티앤아이 2008  TRUE FALSE
285  4986588998   비티앤아이 2010 FALSE  TRUE
191 12954724212 롯데관광개발 2008  TRUE FALSE
201 12561340891 롯데관광개발 2010 FALSE  TRUE

\end{Soutput}
\end{Schunk}
%$

처음과 마지막 레코드를 명시하는 지시자는 불필요하므로 데이터로부터 제거합니다.
 
\begin{Schunk}
\begin{Soutput}
다트8.5 <- 다트8.4[,-c(4:5)]

> 다트8.5
         매출액       여행사 년도
149 16463319852         세중 2008
159 20282845926         세중 2010
30  57624282255     하나투어 2008
40  66245202818     하나투어 2010
104 26530405258     모두투어 2008
114 36466418562     모두투어 2010
232  4803096281     자유투어 2008
242 10455493093     자유투어 2010
77  19926326125   레드캡투어 2008
87  28589391364   레드캡투어 2010
168  5790815204   참좋은레져 2008
178 14717211161   참좋은레져 2010
275  1012965573   비티앤아이 2008
285  4986588998   비티앤아이 2010
191 12954724212 롯데관광개발 2008
201 12561340891 롯데관광개발 2010
> 
\end{Soutput}
\end{Schunk}

그런데, 데이터가 종형으로 배열되어 있기 때문에 2008년과 2010년 매출액의 차이를 쉽게 구할 수 없습니다. 
그래서, 아래와 같이 데이터를 횡형으로 재배열 해야 합니다. 

\begin{Schunk}
\begin{Soutput}
> 다트8.6 <- reshape(다트8.5, timevar="년도", idvar="여행사", direction="wide")
          여행사 매출액.2008 매출액.2010
149         세중 16463319852 20282845926
30      하나투어 57624282255 66245202818
104     모두투어 26530405258 36466418562
232     자유투어  4803096281 10455493093
77    레드캡투어 19926326125 28589391364
168   참좋은레져  5790815204 14717211161
275   비티앤아이  1012965573  4986588998
191 롯데관광개발 12954724212 12561340891
> 
\end{Soutput}
\end{Schunk}

이제서야 원하는 차이를 구할 수 있습니다.

\begin{Schunk}
\begin{Soutput}
> 다트8.6$차이 <- with(다트8.6, 매출액.2010 - 매출액.2008)
> 다트8.6
          여행사 매출액.2008 매출액.2010       차이
149         세중 16463319852 20282845926 3819526074
30      하나투어 57624282255 66245202818 8620920563
104     모두투어 26530405258 36466418562 9936013304
232     자유투어  4803096281 10455493093 5652396812
77    레드캡투어 19926326125 28589391364 8663065239
168   참좋은레져  5790815204 14717211161 8926395957
275   비티앤아이  1012965573  4986588998 3973623425
191 롯데관광개발 12954724212 12561340891 -393383321
> 
\end{Soutput}
\end{Schunk}
% $

위에서는 종형으로 이루어진 데이터를 횡형으로 변경하였으나, 우리는 이 횡형으로 된 데이터를 다시 종형으로도 되돌릴 수 있습니다. 
이와 같이 하기 위해서는 아래와 같이 하면 됩니다.

\begin{Schunk}
\begin{Soutput}
> reshape(다트8.6,  v.names=c("매출액"), varying=c("매출액.2008", "매출액.2010"), direction="long", timevar=c("년도"), times=c("2008", "2010"), ids=row.names(다트8.6))
               여행사 년도      매출액  id
149.2008         세중 2008 16463319852 149
30.2008      하나투어 2008 57624282255  30
104.2008     모두투어 2008 26530405258 104
232.2008     자유투어 2008  4803096281 232
77.2008    레드캡투어 2008 19926326125  77
168.2008   참좋은레져 2008  5790815204 168
275.2008   비티앤아이 2008  1012965573 275
191.2008 롯데관광개발 2008 12954724212 191
149.2010         세중 2010 20282845926 149
30.2010      하나투어 2010 66245202818  30
104.2010     모두투어 2010 36466418562 104
232.2010     자유투어 2010 10455493093 232
77.2010    레드캡투어 2010 28589391364  77
168.2010   참좋은레져 2010 14717211161 168
275.2010   비티앤아이 2010  4986588998 275
191.2010 롯데관광개발 2010 12561340891 191
> 
\end{Soutput}
\end{Schunk}


 
현재의 데이터셋을 가지고 보여줄 수 있는 추가적인 사항들 -- (지금 이것들 전부다 문자열과 관계되는 부분임) 
\begin{itemize}
\item 두 문자형 변수 결합하기
\item 특정 문자열 뽑아내기
\item 변수의 길이 파악하기
\end{itemize}

아래와 같은 내용을 보여주기 위해서는 다른 데이터셋이 필요함 
\begin{itemize}
\item 주어진 데이터셋으로부터 랜덤샘플 추출하기
\item 데이터셋 합치기와 머지하기 
\item 대소문자 전환 
\item 시간과 날짜 데이터 다루기 
\end{itemize}



\section{추가적인 유용한 조작팁들}
 
\paragraph{결측치를 바로 윗값으로 채워넣기: } 아래와 같이 주어진 데이터에 변수 ID는 결측값 없이 모든 값이 완전하게 잘 들어가 있는데, Week 변수에는 각 ID의 첫번째 레코드에만 해당하는 부분에 값이 들어가 있고 나머지부분에는 \texttt{NA}값이 들어가 있습니다. 

\begin{Schunk}
\begin{Soutput}
mydata <- data.frame(ID=c(rep(1,4), rep(2,4), rep(3,2)), Week=c(15, NA, NA, NA, 18, NA, NA, NA, 20, NA))

> mydata		

   ID Week
1   1   15
2   1   NA
3   1   NA
4   1   NA
5   2   18
6   2   NA
7   2   NA
8   2   NA
9   3   20
10  3   NA
\end{Soutput}
\end{Schunk}

이와 같은 데이터를 아래와 같이 자동으로 채워주려면 어떻게 해야 할까요? 	
	
\begin{Schunk}
\begin{Soutput}
   ID Week
1   1   15
2   1   15
3   1   15
4   1   15
5   2   18
6   2   18
7   2   18
8   2   18
9   3   20
10  3   20
\end{Soutput}
\end{Schunk}
	

이를 수행하는데에는 여러 가지 종류의 함수들이 다양한 패지키 안에 존재합니다.  
그러나, 이를 수행하는 기본 알고리즘은 동일하며, R 기본시스템만으로 작성이 가능합니다. 
아래의 함수를 복사하여 사용하시면 됩니다. 

\begin{Schunk}
	\begin{Soutput}
fill <- function(x, first, last){
	n <- last-first+1
	for(i in c(1:length(first))) x[first[i]:last[i]] <- rep(x[first[i]], n[i])
	return(x)
}
	\end{Soutput}
\end{Schunk}




%%%%%%%%%%%%%%%%%%%%%%%%%%%%%%%%%%%%%%%%%%%%%%%%%%%%%%%%%%%%%%%%%%%%%%%%
%
%
% CHAPTER: INPUT AND OUTPUT 으로 변경이 가능한 섹션임 
%
%
%%%%%%%%%%%%%%%%%%%%%%%%%%%%%%%%%%%%%%%%%%%%%%%%%%%%%%%%%%%%%%%%%%%%%%%%

\section{데이터 파일 입출력}

데이터 입력과 출력은 R을 이용한 분석에서의 첫번째 단계와 마지막 단계라고도 할 수 있습니다. 
데이터가 입력된 형식은 매우 다양하지만,  일반적으로 R을 이용하여 데이터의 입출력을 하는데 있어 안전한 방법은 \texttt{.csv}이라는 파일확장자를 가진 파일을 이용하는 것입니다.
따라서, 가급적이면 다른종류의 파일확장자를 .csv로 먼저 변경한 뒤에 사용하는 것이 좋습니다. 

\begin{Schunk}
\begin{Soutput}
mydata <- read.table(file="./filename.csv", header=TRUE, sep=",")
\end{Soutput}
\end{Schunk}

여기에서 filename.csv 은 파일명입니다.

\paragraph{입력과 관련된 문제해결법}

\texttt{read.table()} 함수를 이용하여 데이터를 불러오는데 있어서 많이 발생되는 오류는 ``데이터 파일을 작업디렉토리로부터 찾을 수 없다'' 또는 ``데이터 파일이 존재한 파일경로가 올바르지 않다'' 라는 것입니다. 

파일을 입력받을 때 R은 일반적으로 첫번째 인자에 주어진 파일명과 현재 작업디렉토리의 파일경로를 함께 묶어 절대경로를 생성한 뒤, 이 절대경로를 이용하여 파일명을 찾습니다. 
이러한 원리때문에 운영체제가 영어가 아닌 컴퓨터의 경우, 이 절대경로를 올바르게 생성하지 못할 경우가 있습니다. 
또한, 파일경로명에 띄어쓰기가 있는 경우 및 특수문자가 포함된 경우에 이러한 문제가 발생할 경우가 있습니다. 
따라서, 사용자는 간혹 문법에서 틀린 점도 없고, 불러오고자 하는 데이터 파일도 올바른 파일경로에 위치하고 있음에도 불구하고, 데이터를 찾을 수 없다는 에러 메시지를 보게 되는 경우가 있습니다. 
이러한 경우에 보다 안전한 방법으로 \texttt{read.table()} 사용하고자 한다면 아래와 같이 \texttt{file.choose()} 함수 또는 \texttt{file.path()} 함수를 이용하시길 바랍니다.

\begin{Schunk}
\begin{Soutput}
mydata <- read.table(file.choose(), header=TRUE, sep=",")
\end{Soutput}
\end{Schunk}

\texttt{file.choose()}는 탐색기를 띄워 사용자가 원하고자 하는 파일을 찾을 수 있도록 도와줍니다. 

\begin{Schunk}
\begin{Soutput}
mydata <- read.table(file.path(), header=TRUE, sep=",")
mydata1 <- read.table(file=url(site_address), header=TRUE, sep=",")
\end{Soutput}
\end{Schunk}

\texttt{file.path()}는 절대경로를 보다 안전하게 R이 이해할 수 있도록 도와줍니다. 


% http://www.statmethods.net/input/importingdata.html

\begin{Schunk}
\begin{Soutput}
age <- scan()
32 33 39 28 20 20
\end{Soutput}
\end{Schunk}


\subsection{출력}

\paragraph{저장하기}
\subparagraph{.RData}
\subparagraph{.CSV}
\subparagraph{.HTML}
\begin{Schunk}
\begin{Soutput}
write(t(mydata), file="./where/should/be/saved", ncolumns)
\end{Soutput}
\end{Schunk}

\paragraph{데이터셋 또는 변수에 주석첨가하기}






\chapter{환경설정 및 유틸리티}

이번 챕터에서는 R을 사용하는데 있어 알아두면 도움이 되는 유틸리티에 대해서 알아봅니다.


\section{파일과 스크립트 관리}

R에서 콘솔로 작업을 하는데 프로그램의 크기가 늘어난다면 프로그램의 작성에 매우 어려움을 느끼게 될 것입니다.
따라서, R에서 제공하는 스크립트 에디터가 아닌 외부 텍스트에디터를 이용하여 스크립트를 작성 뒤 실행하면 좋을 것입니다.

\paragraph{초기 메시지 없이 R 실행하기}
먼저 아래와 같이 R을 실행시킵니다. 
\texttt{--quiet}라는 옵션을 주게 되면 처음에 출력되는 시작 메시지를 표시하지 않습니다. 

\begin{Schunk}
\begin{Soutput}
gnustats@CHL072:~/Desktop$ R --quiet
> 
\end{Soutput}
\end{Schunk}
% $

\paragraph{현재 위치 확인하기:}
일단은 내가 현재 어디에 있는지 알아야 할 것입니다.
\begin{Schunk}
\begin{Soutput}
> getwd()
[1] "/home/gnustats/Desktop"
> 
\end{Soutput}
\end{Schunk}

\paragraph{작업디렉토리 만들기:}
현재 나의 바탕화면에서 재밌는 것들이 많이 놓여 있기 때문에 작업에 혼돈을 주지 않기 위해 \texttt{testR}이라는 작업디렉토리를 생성할 것입니다. 
\begin{Schunk}
\begin{Soutput}
> dir.create("testR")
\end{Soutput}
\end{Schunk}

\paragraph{작업디렉토리 지정하기}
그리고 난 다음에 실제로 작업을 위해서 작업디렉토리로 들어갑니다. 
그리고 나서 제대로 들어왔는지 확인해 봅니다. 

\begin{Schunk}
\begin{Soutput}
> setwd("testR")
> getwd()
[1] "/home/gnustats/Desktop/testR"
>
\end{Soutput}
\end{Schunk}

\paragraph{작업디렉토리 내 파일확인하기:}
방금 막 만들어 놓은 작업디렉토리이기 때문에 아무 파일도 존재하지 않습니다. 
이를 list.files()라는 함수를 이용하여 확인합니다. 

\begin{Schunk}
\begin{Soutput}
> list.files()
character(0)
\end{Soutput}
\end{Schunk}

\paragraph{스크립트 생성하기}
이제 스크립트를 작성합니다. 
\begin{Schunk}
\begin{Soutput}
> file.create("myRcode.R")
[1] TRUE
> list.files()
[1] "myRcode.R"
\end{Soutput}
\end{Schunk}

\paragraph{스크립트 확인하기}
이렇게 myRcode.R 이라는 파일만 생성을 했지 프로그램에는 아무런 내용이 없을 것입니다. 
이를 확인하기 위해서는 file.show() 함수를 사용합니다. 

\begin{Schunk}
\begin{Soutput}
> file.show("myRcode.R")
>
\end{Soutput}
\end{Schunk}

아무런 내용이 없으므로 아무것도 없이 END 라는 메시지만 보일 것입니다.

\paragraph{에디터 설정하기: }
그런데, 내가 파일을 무슨 에디터를 이용하여 작성하는지를 모릅니다. 
그래서 확인해 보도록 하겠습니다.

\begin{Schunk}
\begin{Soutput}
> options()$editor
[1] "vi"
\end{Soutput}
\end{Schunk}
% $
그런데 나는 vi 라는 에디터를 싫어합니다.  
그 이유는 나는 emacs 또는 nano 를 사용하기 때문입니다.
그래픽 에디터로는 geany 또는 gedit 을 사용합니다. 
그래서 바꾸고 싶습니다. 

\begin{Schunk}
\begin{Soutput}
> options(editor="nano")
> options()$editor
[1] "nano"
> 
\end{Soutput}
\end{Schunk}
% $ 

이렇게 바뀜을 확인할 수 있었습니다.

\paragraph{파일 편집 후 콘솔 내에서 실행하기}
그래서 이제 아까 생성한 myRcode.R을 편집하도록 하겠습니다. 

\begin{Schunk}
\begin{Soutput}
> file.edit("myRcode.R")
\end{Soutput}
\end{Schunk}

이렇게 했더니 커서가 깜빡이면서 nano가 열리면서 스크립트를 작성할 준비가 되었습니다.
그래서 아래와 같이 입력하고 저장하고 파일에디터를 닫았습니다. 

\begin{Schunk}
\begin{Soutput}
print("Hello R!  This is my first R program")
\end{Soutput}
\end{Schunk}

이제 저정한 스크립트를 실행하기 위해서 아래와 같이 source() 함수를 사용합니다.

\begin{Schunk}
\begin{Soutput}
> source("myRcode.R")
[1] "Hello R!  This is my first R program"
> 
\end{Soutput}
\end{Schunk}

\paragraph{파일 복사하기:}
이제 이 스크립트를 보관하기 위해서 복사를 해 두겠습니다.

\begin{Schunk}
\begin{Soutput}
> list.files()
[1] "myRcode.R"
> file.copy("myRcode.R", "myRcode.copied.R")
[1] TRUE
> list.files()
[1] "myRcode.copied.R" "myRcode.R"       
> 
\end{Soutput}
\end{Schunk}

아하! 잘 복사가 되었음을 확인하였습니다.

\paragraph{파일 삭제하기:}
그런데, 생각해 보니 이 파일을 대체 왜 복사했는지 모르겠습니다. 
그래서 지워보고 싶습니다. 

\begin{Schunk}
\begin{Soutput}
> file.remove("myRcode.copied.R")
[1] TRUE
> list.files()
[1] "myRcode.R"
> 
\end{Soutput}
\end{Schunk}

이제 다시 파일을 아래와 같이 수정해 봅니다. 
\begin{Schunk}
\begin{Soutput}
print("Hello R!  This is my first R program")
print("안녕 난 R 이야!")
x <- 3
y <- 2
cat(x, "+", y, "=", x+y, "\n")
\end{Soutput}
\end{Schunk}

\paragraph{스크립트를 터미널에서 실행하기: }

먼저 R을 종료하고 testR이라는 작업디렉토리에 들어값니다.  
\begin{Schunk}
\begin{Soutput}
> q()
Save workspace image? [y/n/c]: n
gnustats@CHL072:~/Desktop$ cd testR
gnustats@CHL072:~/Desktop/testR$ ls
myRcode.R
\end{Soutput}
\end{Schunk}

myRcode.R 파일이 작업디렉토리에 있는 것을 확인한 뒤에, 아래와 같이 명령어를 입력합니다. 

\begin{Schunk}
\begin{Soutput}
gnustats@CHL072:~/Desktop/testR$ R CMD BATCH --quiet myRcode.R
gnustats@CHL072:~/Desktop/testR$ ls
myRcode.R  myRcode.Rout
\end{Soutput}
\end{Schunk}

아하! 새로운 myRcode.Rout 이라는 파일이 새로이 생긴 것을 알 수 있으며 내용을 확인해 보니 아래와 같습니다. 

\begin{Schunk}
\begin{Soutput}
gnustats@CHL072:~/Desktop/testR$ cat myRcode.Rout
> print("Hello R!  This is my first R program")
[1] "Hello R!  This is my first R program"
> print("안녕 난 R 이야!")
[1] "안녕 난 R 이야!"
> x <- 3
> y <- 2
> cat(x, "+", y, "=", x+y, "\n")
3 + 2 = 5 
> 
> proc.time()
   user  system elapsed 
  0.208   0.016   0.206 
gnustats@CHL072:~/Desktop/testR$ 
\end{Soutput}
\end{Schunk}
% $ 

프로그램의 결과가 프로그램이 수행된 시간과 함께 myRcode.Rout 이라는 파일에 저장되었음을 알 수 있습니다. 
\texttt{--slave}라는 옵션을 아래와 같이 주게 사용한다면 저장되는 결과물은 아래와 같습니다. 
  
\begin{Schunk}
\begin{Soutput}
gnustats@CHL072:~/Desktop/testR$ R CMD BATCH --quiet --slave myRcode.R
gnustats@CHL072:~/Desktop/testR$ cat myRcode.Rout
[1] "Hello R!  This is my first R program"
[1] "안녕 난 R 이야!"
3 + 2 = 5 
> proc.time()
   user  system elapsed 
  0.224   0.012   0.224 
gnustats@CHL072:~/Desktop/testR$ 
\end{Soutput}
\end{Schunk}
% $

\subparagraph{쉘 프로그램으로 실행하기:}

쉘프로그램을 작성하시는 분들은 다음과 같이 R 스크립트를 작성하고 수행할 수 있습니다. 
myRcode.R 프로그램을 아래와 같이 변경합니다.

\begin{Schunk}
\begin{Soutput}
gnustats@CHL072:~/Desktop/testR$ cat myRcode.R
#!/usr/bin/Rscript --vanilla

print("Hello R!  This is my first R program")
print("안녕 난 R 이야!")
x <- 3
y <- 2
cat(x, "+", y, "=", x+y, "\n")
gnustats@CHL072:~/Desktop/testR$ 
\end{Soutput}
\end{Schunk}

이제 쉘 프로그램과 같이 실행할 수 있습니다. 
\begin{Schunk}
\begin{Soutput}
gnustats@CHL072:~/Desktop/testR$ chmod u+x myRcode.R
gnustats@CHL072:~/Desktop/testR$ ./myRcode.R
[1] "Hello R!  This is my first R program"
[1] "안녕 난 R 이야!"
3 + 2 = 5 
\end{Soutput}
\end{Schunk}

\paragraph{디렉토리 변경:}
이제 다른 작업을 해야하므로 기존 작업과 혼돈하지 않기 위해서 또하나의 디렉토리를 생성해 봅니다. 
그렇고 보니, 이런식으로 작업을 하게 된다면 디렉토리를 이동하는것이 중요할 것입니다. 
그러나 이것은 위에서 언급한 setwd()이면 충분합니다. 
아래의 예제를 살펴보세요.

\begin{Schunk}
\begin{Soutput}
> # 현재 디렉토리 확인 
> getwd()
[1] "/home/gnustats/Desktop/testR"
> 
> # tmp 디렉토리 생성 
> dir.create("tmp")
>
> # tmp 디렉토리로 이동 
> setwd("tmp")
> 
> # 현재 디렉토리 다시 확인 
> getwd()
[1] "/home/gnustats/Desktop/testR/tmp"
> 
> # 다시 상위 디렉토리 testR로 이동 
> setwd("..")
>
> # 현재 디렉토리 다시 확인 
> getwd()
[1] "/home/gnustats/Desktop/testR"
\end{Soutput}
\end{Schunk}


\section{객체관리}

\paragraph{객체확인: } R에서 생성되고 다루어지는 그 모든 것들을 객체라고 합니다. 
객체라는 개념을 보다 정확하게 이해하기 위해서는 클래스라는 개념을 알아야 하는데, 본 파트에서는 R의 사용에 익숙해지는 것이 주목적이므로 다루지 않습니다.
그러나, 객체지향적 프로그래밍을 위한 클래스와 객체, 그리고 메소드라는 개념에 대해서는 ``패키지 작성''이라는 챕터에서 다루도록 하겠습니다. 

현재 세션내에서 사용되고 있는 모든 객체들의 목록을 확인해 보고 싶을 때는 ls()함수를 이용합니다. 
R을 시작하면 어떠한 작업도 하지 않았기 때문에 생성된 객체가 없을 것입니다. 

\begin{Schunk}
\begin{Soutput}
gnustats@CHL072:~/Desktop/testR$ R --quiet
> ls()
character(0)
\end{Soutput}
\end{Schunk}
% $ 
따라서, ls() 함수를 입력했을 때 아무것도 보여줄 것이 없다는 의미의 character(0) 이라는 것이 메시지가 나옵니다.
이러한 표현은 ls()의 결과값이 문자형이기 때문입니다.
(ls는 list의 약어입니다).

이번에는 이전에 작성한 myRcode.R 프로그램을 수행해봅니다.
그리고 나서 어떤 객체가 있는지 살펴봅니다.
\begin{Schunk}
\begin{Soutput}
> source("myRcode.R")
[1] "Hello R!  This is my first R program"
[1] "안녕 난 R 이야!"
3 + 2 = 5 
> ls()
[1] "x" "y"
\end{Soutput}
\end{Schunk}

아하! 이전 프로그램에서 x와 y라는 변수를 생성하여 값을 대입했었기 때문에 객체가 존재합니다. 
따라서 "x"와 "y"라는 객체가 나열됨을 확인할 수 있습니다.
이번에는 z 라는 객체를 생성해 봅니다.
그리고 다시 어떤 객체들이 존재하는지 확인해 봅니다. 

\begin{Schunk}
\begin{Soutput}
> z <- 8
> ls()
[1] "x" "y" "z"
\end{Soutput}
\end{Schunk}

한개의 객체를 더 생성해 봅니다. 

\begin{Schunk}
\begin{Soutput}
> a <- "너도 R이니?  나도 R이야"
> ls()
[1] "a" "x" "y" "z"
> 
\end{Soutput}
\end{Schunk}

\paragraph{객체삭제:} 만약, 작업을 하면서 너무 많은 양의 객체들로 인하여 R의 메모리 사용에 영향이 생길때 객체를 삭제하고 싶을 경우가 있습니다. 
이럴때는 rm()이라는 함수를 사용합니다.  
(rm 은 remove의 약자입니다).

\begin{Schunk}
\begin{Soutput}
> # 현재 세션의 객체들을 확인합니다. 
> ls()
[1] "a" "x" "y" "z"
>
> # 객체 x를 삭제해봅니다 
> rm("x")
> ls()
[1] "a" "y" "z"
> 
> # 객체 x를 삭제한 뒤에 다시 x를 사용하고자 하면 어떻게 될까요?
> x
Error: object 'x' not found
> 
\end{Soutput}
\end{Schunk}

객체를 확인할때는 ls()를 사용하지만 "객체"라는 말 그대로 objects()라는 함수도 동일한 역할을 수행합니다. 

\begin{Schunk}
\begin{Soutput}
	
> # 현재 세션의 객체들 확인하기 
> objects()
[1] "a" "y" "z"
>
> # 이번엔 "y"와 "z"를 한 번에 삭제하기 
> rm(list=c("y", "z"))
> objects()
[1] "a"
> 
\end{Soutput}
\end{Schunk}

현 세션내의 작업공간 (workspace)의 모든 객체들을 삭제하기 위해서는 아래와 같이 할 수 있습니다. 

\begin{Schunk}
\begin{Soutput}
> rm(list=ls())
> objects()
character(0)
> 
\end{Soutput}
\end{Schunk}


\paragraph{데이터 확인:}

\paragraph{데이터 편집:}

\paragraph{데이터 입력:}

\paragraph{작업공간 저장하기:}

\paragraph{작업공간 불러오기:}

\paragraph{작업내역 불러오기:}

\paragraph{작업내역 저장하기:}



\section{환경설정 및 관리}

\paragraph{sessionInfo()}
\paragraph{검색경로를 확인해본단} search()
\paragraph{chooseCRANmirror()}
\paragraph{options(repos="URL")}
\paragraph{options(width=)}
\paragraph{홈 디렉토리 찾기}
\paragraph{환경변수 찾아 확인하기}
\paragraph{인코딩확인}



\subsection{패키지 관리}

\paragraph{패키지를 설치후 불러오지 않았다}
\paragraph{install.packages()}
\paragraph{remove.packages()}
\paragraph{update.packages()}
\paragraph{패키지 도움말을 보려면 } help(package="pkg_{name}")
\paragraph{비네트를 확인하려면 }  
vignette()
vignette(package="pkg_{name}")
\paragraph{내장된 데이터셋에 접근하자} data(car93, package="MASS")
\paragraph{library()}
\paragraph{installed.packages()}
\paragraph{함수의 사용예제를 알고 싶다면 } example()을 사용하세요. 
\paragraph{함수에 어떤 인자가 반드시 들어가야 하는지 확인해야 한다면}  args()를 이용하세요.
\paragraph{R에서 제공되는 도움말 문서를 확인하고자 한다면:} help.start()를 이용하세요.
\paragraph{연산시에 R을 종료하지 않고 중단하고자 할때:}  Ctrl+C 를 눌러보세요.

\paragraph{함수명을 쓰고 괄호를 잊는다}
\paragraph{윈도우에서 파일경로명 작성시 역슬래쉬를 사용하지 않았다}

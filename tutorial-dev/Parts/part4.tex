
%%%%%%%%%%%%%%%%%%%%%%%%%%%%%%%%%%%%%%%%%%%%%%%%%%%%%%%%%%%%%%%%%%%%%%%%%%%%%%%%%%%
%
%
%
%%%%%%%%%%%%%%%%%%%%%%%%%%%%%%%%%%%%%%%%%%%%%%%%%%%%%%%%%%%%%%%%%%%%%%%%%%%%%%%%%%%

\chapter{탐색적 데이터 분석}

여기에서 말하는 탐색적 분석이란 분석 초기에 단순히 데이터의 특징을 보는데 사용됩니다. 
또한, 이 과정은 데이터 클리닝과도 연관이 있습니다. 

\section{기술통계량 요약}

\paragraph{평균과 분산과 같은 기초요약 함수들}
\begin{Schunk}
\begin{Soutput}
output
\end{Soutput}
\end{Schunk}

\paragraph{그룹별 평균산출}
\begin{Schunk}
\begin{Soutput}
output
\end{Soutput}
\end{Schunk}

\paragraph{5분위수 구하기}
\begin{Schunk}
\begin{Soutput}
output
\end{Soutput}
\end{Schunk}

\paragraph{퀀타일}
\begin{Schunk}
\begin{Soutput}
output
\end{Soutput}
\end{Schunk}

\paragraph{표준화와 스케일링}
\begin{Schunk}
\begin{Soutput}
output
\end{Soutput}
\end{Schunk}

\paragraph{신뢰구간}
\begin{Schunk}
\begin{Soutput}
output
\end{Soutput}
\end{Schunk}

\section{분할표와 카이제곱 검정}
\begin{Schunk}
\begin{Soutput}
output
\end{Soutput}
\end{Schunk}

\section{z-검정}
\begin{Schunk}
\begin{Soutput}
output
\end{Soutput}
\end{Schunk}

\section{t-검정}
\begin{Schunk}
\begin{Soutput}
output
\end{Soutput}
\end{Schunk}

\section{}
\begin{Schunk}
\begin{Soutput}
output
\end{Soutput}
\end{Schunk}




%  File tutorial-dev/Parts/part1-ch02.tex
%  Part of the iHELP project at http://ihelp.r-forge.r-project.org
%
%  Copyright (C) 2013- The iHELP Working Group 
%                                in the Korean R Translation Team
%
%  This program is free software; you can redistribute it and/or modify
%  it under the terms of the GNU General Public License as published by
%  the Free Software Foundation; either version 2 of the License, or
%  (at your option) any later version.
%
%  This program is distributed in the hope that it will be useful,
%  but WITHOUT ANY WARRANTY; without even the implied warranty of
%  MERCHANTABILITY or FITNESS FOR A PARTICULAR PURPOSE.  See the
%  GNU General Public License for more details.
%
%  A copy of the GNU General Public License is available at
%  http://www.r-project.org/Licenses/
%

\chapter{분석 후 개발과 관련하여}
\section{클래스와 메소드 그리고 패키지 제작}
\begin{enumerate}
\item 패키지를 만들고 싶어요. (흠.. Generalized Linear Model 프레임워크 흉내내서 똑같이 만들어보기 실습자료로 제공해주기)
\end{enumerate}
\begin{Schunk}
\begin{Soutput}
output
\end{Soutput}
\end{Schunk}



%%%%%%%%%%%%%%%%%%%%%%%%%%%%%%%%%%%%%%%%%%%%%%%%%%%%%%%%%%%%%%%%%%%%%%%%
%
% Miscellinous
%
%%%%%%%%%%%%%%%%%%%%%%%%%%%%%%%%%%%%%%%%%%%%%%%%%%%%%%%%%%%%%%%%%%%%%%%%


%%%%%%%%%%%%%%%%%%%%%%%%%%%%%%%%%%%%%%%%%%%%%%%%%%%%%%%%%%%%%%%%%%%%%%%%
%
%
%
%%%%%%%%%%%%%%%%%%%%%%%%%%%%%%%%%%%%%%%%%%%%%%%%%%%%%%%%%%%%%%%%%%%%%%%%

\chapter{간단한 GUI 제작 해보기}

\begin{enumerate}
	\item 다른 언어로 인터페이싱 하는 방법마로, 그냥 \texttt{R}에서 주어지는 패키지를 이용해서 간단한 GUI 환경만들기
	\item 아마도... R Commander를 확장하는 방법을 예로 들면 좋을 것 같음
	\item 원리도 간단히 설명해주면 더욱 좋을 것 같음. 
\end{enumerate}

\begin{Schunk}
\begin{Soutput}
output
\end{Soutput}
\end{Schunk}

%%%%%%%%%%%%%%%%%%%%%%%%%%%%%%%%%%%%%%%%%%%%%%%%%%%%%%%%%%%%%%%%%%%%%%%%
%
%
%
%%%%%%%%%%%%%%%%%%%%%%%%%%%%%%%%%%%%%%%%%%%%%%%%%%%%%%%%%%%%%%%%%%%%%%%%


\chapter{미분류 질문들}

% 콘솔 컨트롤
% ctrl + L을 누르면 콘솔의 내용이 모두 지워짐

이 섹션에 등록된 질문들은 접수만 되고 아직은 답변되지 않은 상태입니다
% http://www.columbia.edu/~cjd11/charles_dimaggio/DIRE/resources/R/rFunctionsList.pdf
% http://www.ats.ucla.edu/stat/r/library/advanced_function_r.htm
% http://www.sr.bham.ac.uk/~ajrs/R/r-function_list.html
% http://www.scidav.org/techno/r_environments

\begin{enumerate}
	
% 아래의 주소로부터 질문 다 만들어내기 	
% http://www.statmethods.net/input/valuelabels.html

	\item (접수: 2013-APR-23)  제가 가진 데이터셋이 있는데, 이 데이터를 어떤 특정한 변수들의 값을 이용하여 분류하려고 합니다.  어떻게 해야하나요?
	
	\textsf{(답변)}  요것은 \texttt{split()} 함수를 이용하도록 알려줄 것. 

	\item (접수: 2013-APR-21) R 패키지를 CRAN에 올리는 방법을 알려주세요 
	
	\textsf{(답변)} 이 질문을 대답할 때는 반드시 CRAN Package Submission Guideline에 대해서 알려줘야 함.  (이거 번역해 놨는데 당췌 어디에 뒀는지 찾을 수가 없음, 2013-04-20 까지 못 찾으면 새로이 번역할 것)
	
	\item (접수: 2013-04-18, Reproducibility=NO) read.xlsx함수를 이용해 xlsx파일에서 데이터프레임형태로 가져옵니다. 이 때 [3,3] 셀에 있는 텍스트가 "3월" 이라고 할 때 temp[3,3] == "3월" 이렇게 비교하려고 하면 제대로 비교가 안되더군요.. 한글 텍스트로 이루어진 변수값를 비교하는 방법이 어떤게 있는지 궁금합니다.
	
	\item 분석을 하고 나면 결과를 그래프나 그림으로 나타내게 되는데 R에서는 그림을 나타내는 창이 하나만 나타나서 동시에 두 개를 보지 못하는 경우가 허다한데, 이의 해결방법은 없나요? (접수: 2013-APR-13, 분류: 그래픽스 관련) 
	
	\textsf{(답변)} \texttt{R}에서는 그래픽 디바이스가 그래픽 생성시 마다 초기화되어 다시 보여줌으로서 그래픽 창이 하나만 계속 보여지는 것입니다.  새로운 그래프를 또다른 장치를 통해 보여주고자 한다면 \texttt{X11()}이라는 명령어를 이용하면 됩니다.  
	이 명령어는 유닉스환경에 설치된 \texttt{R}의 경우에 해당합니다.  
\end{enumerate}


\section{답변되지 않을 수도 있는 질문들}
\begin{enumerate}
	\item (접수: 2013-04-18, Reproducibility=NA) R의 장점이자 단점이라고 생각되는 것 중에 하나가 엄청난 수의 패키지들임. 즉 어떤 분석을 하고자 할 때 그것에 대해 하나의 패키지가 있는 것이 아니라 대체적으로 사용가능한 패키지들이 존재하는데 이들 중 어느 것을 써야할 지 잘 모름. 다른 분석 프로그램의 경우 이러한 문제가 없는데... 결국엔 어떻게 제일 성능이 좋은? 결과가 신뢰할 만한? 좋은 패키지를 선택하는가를 알려주었으면 좋겠씀돠.
	
	\textsf{(답변)} 이것은 경험에 해당되며, 해당분야의 전문가로부터의 조언을 받는 것이 안전합니다.  그렇지 않다면, 직접 베이스를 이용하여 작성하면 됩니다. 
	
\end{enumerate}

% R을 사용해서 web 크롤링을 하는데 속도가 엄청 느리네요 scan() 함수를 쓰면 url로부터 html데이터를 받게 되는데 이것을 substring()함수와 regexprs()함수를 사용하여 적절히 데이터를 분해 하는 작업을 하는데 엄청느리네요. 원체느린거 같은데 혹시 해결방안 아시는분 게시나요?

% R에 있는 ggplot2 패키지를 이용해 얻은 얻은 plot그림을 웹에 보여주는 방법에 대해 아시는 분 계신가요?


\part{알면 도움이 되는}
\chapter{유닉스 명령어}

\chapter{\LaTeX 사용법}

\chapter{Perl 명령어}

\chapter{C 언어} 



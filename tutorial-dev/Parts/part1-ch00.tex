%  File tutorial-dev/Parts/part1-ch02.tex
%  Part of the iHELP project at http://ihelp.r-forge.r-project.org
%
%  Copyright (C) 2013- The iHELP Working Group 
%                                in the Korean R Translation Team
%
%  This program is free software; you can redistribute it and/or modify
%  it under the terms of the GNU General Public License as published by
%  the Free Software Foundation; either version 2 of the License, or
%  (at your option) any later version.
%
%  This program is distributed in the hope that it will be useful,
%  but WITHOUT ANY WARRANTY; without even the implied warranty of
%  MERCHANTABILITY or FITNESS FOR A PARTICULAR PURPOSE.  See the
%  GNU General Public License for more details.
%
%  A copy of the GNU General Public License is available at
%  http://www.r-project.org/Licenses/
%


% 
% 이런 느낌은 특히 최소한 내가 알기로는 경영학과 경제학 분야(거의 확실함) 및 사회학, 심리학 분야가 특히 심하다. 
% 그리고 이들 분야에서는
% R의 다양한 기능(특히 visualization) 또는 packages를 필요로 하지 않을 가능성이 높다. 왜냐하면 분석 방법이 단순하기 때문이다.
% 여기에서 분석 방법이 단순하다는 것은 그 이론적 배경이 쉽다는 것을 의미하는 것이 아니라 절차상의 얘기이다.
% 언급한 분야에서 학술연구에서 실행되는 분석의 양을 보면 대부분 [기술통계 - 상관관계 - 차이분석, 회귀분석, 구조방정식분석] 정도로
% 분석이 마무리 되기 때문이다. 그리고 대부분 표(table)로 결과를 보고하는 형식을 따르고 있다.
% 분석 모형 자체를 다루는 연구가 거의 없는 것이다. 따라서 R의 작업 방식은 그들에게는 불필요할 지도 모른다.

% 그러나 여전히 R의 다양한 source를 필요로 하는 아주 작은 규모의 학술연구 분야는 존재한다. 따라서 여기의 tips에는 이들을 고려하여
% R 사용에 도움이 될 만한 사항들을 정리하고자 한다.
% 개인적으로는 통계학 및 다른 분야의 연구 과정이 어떻게 진행되는지 잘 모르는 것도 한 몫하여 이에 해당하는 부분은 다른 이의 도움이
% 절실하다.

% 덧붙여 나는 윈도우 환경에서의 R tip에 주목하고자 한다(Mac이나 유닉스/리눅스 환경에서의 R 사용에 관한 tip도 다른 이의 도움이 필요한
% 부분이다). 그 이유는 내가 윈도우에서 SPSS와 SAS를 숙련시켜 왔기 때문에 대부분의 통계분석 초심자들도 윈도우 환경에서 R 을 접하게
% 될 것이라는 막연한 억측 때문이다. 시간이 지나면서 여러 사람들을 만나보게 꼭 그렇지만은 않다는 것을 알게 되었지만 최소한 경영학과
% 경제학 분야에 속한 사람들은 그럴 것이라는 확신 때문이다. 이것은 다른 분야의 사람들을 홀대하는 것이 아니라 나의 능력의 한계 때문이다.
% 실제로 나의 경우에도 최근에 C 언어, CentOS, 그리고 MySQL도 공부하기 시작했으니 오해는 하지 말기 바란다(누군가는 포기하라고도 했다.
% 왜냐하면 지금 그것들을 공부하기에는 분량이 너무 많고 난이도가 낮지 않기 때문이다). 그런데 결국에는 내 전공분야에서의 R 사용은
% 그 분량이 많지 않고 리눅스/유닉스 기반의 R 사용에 보다 많은 시간과 지면을 할애하게 될 것이라는 생각이 든다.

% R은 최근의 빅데이터라는 화두와 함께 큰 주목을 받고 있다. 하지만 대부분의 사람들, 특히 사회과학 전공자들은 이 개념적 정의도 제대로
% 알지 못하며 들어보지 못한  전문용어(SQL, Hadoop, 그리고 Mapreduce 등)에 혹하고 있는 실정이다.
% 나의 경우 R을 구글링(googling)으로 공부하였다. 왜냐하면 내가 재학하던 대학에서는 R 교육 프로그램이 없었기 때문이다. 처음 접하게 된
% 것은 미시경제분석 수업이었다. 그 수업에서 교수님이 수업을 진행하시는데 R을 사용한 것이다(SAS도 조금 사용했었으나 SPSS 는 잡동사니
% 취급을 하셨던 것으로 기억한다). 누군가에게 강의를 받는 식의 교육은 그것이 전부였다. 이후의 학습은 모두 맨땅에 해딩과 계란으로
% 바위깨기 식이었다. 이 R tips 과정은 이러한 어려움을 덜어주기 위한 방안이 될 수도 있을 것이다.
% 그러기를 한참 후 어느 정도 익숙해지고 나니, 또 연구자로서의 건방이 살아나 R 교육이 어떻게 이루어지고 있는 지가 궁금해졌다.
% 이 부분은 나중에 다시 써 볼라오,,,


\documentclass[../tutorial.tex]{subfiles}
\begin{document}
	

통계소프트웨어 \texttt{R}을 사용하는데 이 문서가 도움이 되길 바랍니다. 
아래와 같은 내용에 중점을 두고자 하였습니다. 

\begin{itemize}
\item 원리중심의 문제해결을 설명하고자, 다양한 패키지들로부터 제공되는 함수들에 대한 사용법에 대한 설명은 가급적이면 피하려고 하였습니다.
따라서, 이 문서내에서는 특정 패키지가 해결할 수 있는 경우를 제외하고는 BASE (기본) 시스템을 이용하여 문제를 해결할 수 있도록 정리하려고 노력합니다(다른 문서를 통해 R에서 제공하는 여러 패키지를 활용하는 사례들을 소개할 것입니다).

\item R이라는 언어의 특징을 살린 설명을 하고자 하였습니다.  
예를들면, R의 가장 큰 특징이라고 할 수 있는 벡터라는 개념을 활용하여 원리 중심의 설명을 제공합니다. 

\item 통계와 전산에 관련된 전문용어는 쉽게 풀어쓰거나, 용어에 대한 이해를 돕기 위한 참고 자료를 제공합니다. 
\end{itemize}

본 문서와 관련한 프로젝트의 시작일은 2013년 4월 10일이며, iHELP Working Group 관리자에 의하여 수시로 갱신되고 있습니다.
따라서, 이 문서는 어떤 특정한 기간을 두고 완성되지 않으며, 오로지 업데이트된 버전만이 존재합니다.

이 문서는 2005년 이래로 R의 사용에 대한 본 문서의 지은이 개인의 경험과 R Documentation 만을 토대로 하여 작성되었으므로 아직 경험하지 못하여 다루지 못하는 부분이 있습니다.
이 문서의 지은이는 통계학에 대한 
%  B.Sc. (Statistics), M.Sc. (Statistics), 그리고 PhD. Biostatistics Candidate 의 
배경지식을 가지고 있으므로 문서의 내용이 다소 통계 및 수학 분야에 치중하였을 수 있습니다.
지은이는 이 문서가 보다 다양한 계층의 분들께서 R을 이용하여 하고자 하시는 일에 밑거름이 될 수 있는 도움이 되고자 합니다.
따라서, 이 문서를 읽고 있는 독자가 문서 내용에 대한 추가, 수정, 및 제안이 있다면 \href{mailto:gnustats@gmail.com}{gnustats@gmail.com} 또는 \href{mailto:ihelp-urquestion@lists.r-forge.r-project.org}{ihelp-urquestion@lists.r-forge.r-project.org} 의 주소로 이메일을 보내주신다면 감사하겠습니다. 

또한, 아래에 기재된 분들 (가나다 순)의 관심과 제안들이 없었다면 이 문서가 발전될 수 없었기에 그 감사의 말씀을 올리고 싶습니다. 
\begin{itemize}
\item \href{jhshin@assist.ac.kr}{신종화 교수님} (서울종합과학대학원 사회학과)
\item \href{mailto:buillee@hanmail.net}{이부일 박사님} (충남대학교 정보통계학과)
\item \href{mailto:muoe78@gmail.com}{정우준 박사님} (홍익대학교 경영학과)
\item \href{mailto:christ00@hanmail.net}{이성윤님}
\end{itemize}

\paragraph{이 문서를 읽는 방법:} 
\begin{itemize}
\item ``완전초보에요''라는 챕터와 ``기초프로그래밍과 운영체제'' 챕터는 이 문서를 읽기 전에 반드시 먼저 읽으셔야 합니다.  
\end{itemize}

\end{document}

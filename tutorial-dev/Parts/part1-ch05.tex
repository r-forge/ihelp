%  File tutorial-dev/Parts/part1-ch02.tex
%  Part of the iHELP project at http://ihelp.r-forge.r-project.org
%
%  Copyright (C) 2013- The iHELP Working Group 
%                                in the Korean R Translation Team
%
%  This program is free software; you can redistribute it and/or modify
%  it under the terms of the GNU General Public License as published by
%  the Free Software Foundation; either version 2 of the License, or
%  (at your option) any later version.
%
%  This program is distributed in the hope that it will be useful,
%  but WITHOUT ANY WARRANTY; without even the implied warranty of
%  MERCHANTABILITY or FITNESS FOR A PARTICULAR PURPOSE.  See the
%  GNU General Public License for more details.
%
%  A copy of the GNU General Public License is available at
%  http://www.r-project.org/Licenses/
%


%%%%%%%%%%%%%%%%%%%%%%%%%%%%%%%%%%%%%%%%%%%%%%%%%%%%%%%%%%%%%%%%%%%%%%%%
%
%
% CHAPTER: INPUT AND OUTPUT 으로 변경이 가능한 섹션임 
%
%
%%%%%%%%%%%%%%%%%%%%%%%%%%%%%%%%%%%%%%%%%%%%%%%%%%%%%%%%%%%%%%%%%%%%%%%%

\section{데이터 파일 입출력}

데이터 입력과 출력은 R을 이용한 분석에서의 첫번째 단계와 마지막 단계라고도 할 수 있습니다. 
데이터가 입력된 형식은 매우 다양하지만,  일반적으로 R을 이용하여 데이터의 입출력을 하는데 있어 안전한 방법은 \texttt{.csv}이라는 파일확장자를 가진 파일을 이용하는 것입니다.
따라서, 가급적이면 다른종류의 파일확장자를 .csv로 먼저 변경한 뒤에 사용하는 것이 좋습니다. 

\begin{Schunk}
\begin{Soutput}
mydata <- read.table(file="./filename.csv", header=TRUE, sep=",")
\end{Soutput}
\end{Schunk}

여기에서 filename.csv 은 파일명입니다.

\paragraph{입력과 관련된 문제해결법}

\texttt{read.table()} 함수를 이용하여 데이터를 불러오는데 있어서 많이 발생되는 오류는 ``데이터 파일을 작업디렉토리로부터 찾을 수 없다'' 또는 ``데이터 파일이 존재한 파일경로가 올바르지 않다'' 라는 것입니다. 

파일을 입력받을 때 R은 일반적으로 첫번째 인자에 주어진 파일명과 현재 작업디렉토리의 파일경로를 함께 묶어 절대경로를 생성한 뒤, 이 절대경로를 이용하여 파일명을 찾습니다. 
이러한 원리때문에 운영체제가 영어가 아닌 컴퓨터의 경우, 이 절대경로를 올바르게 생성하지 못할 경우가 있습니다. 
또한, 파일경로명에 띄어쓰기가 있는 경우 및 특수문자가 포함된 경우에 이러한 문제가 발생할 경우가 있습니다. 
따라서, 사용자는 간혹 문법에서 틀린 점도 없고, 불러오고자 하는 데이터 파일도 올바른 파일경로에 위치하고 있음에도 불구하고, 데이터를 찾을 수 없다는 에러 메시지를 보게 되는 경우가 있습니다. 
이러한 경우에 보다 안전한 방법으로 \texttt{read.table()} 사용하고자 한다면 아래와 같이 \texttt{file.choose()} 함수 또는 \texttt{file.path()} 함수를 이용하시길 바랍니다.

\begin{Schunk}
\begin{Soutput}
mydata <- read.table(file.choose(), header=TRUE, sep=",")
\end{Soutput}
\end{Schunk}

\texttt{file.choose()}는 탐색기를 띄워 사용자가 원하고자 하는 파일을 찾을 수 있도록 도와줍니다. 

\begin{Schunk}
\begin{Soutput}
mydata <- read.table(file.path(), header=TRUE, sep=",")
mydata1 <- read.table(file=url(site_address), header=TRUE, sep=",")
\end{Soutput}
\end{Schunk}

\texttt{file.path()}는 절대경로를 보다 안전하게 R이 이해할 수 있도록 도와줍니다. 


% http://www.statmethods.net/input/importingdata.html

\begin{Schunk}
\begin{Soutput}
age <- scan()
32 33 39 28 20 20
\end{Soutput}
\end{Schunk}


\subsection{출력}

\paragraph{저장하기}
\subparagraph{.RData}
\subparagraph{.CSV}
\subparagraph{.HTML}
\begin{Schunk}
\begin{Soutput}
write(t(mydata), file="./where/should/be/saved", ncolumns)
\end{Soutput}
\end{Schunk}

\paragraph{데이터셋 또는 변수에 주석첨가하기}





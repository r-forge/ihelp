%  File tutorial-dev/Parts/part1-ch02.tex
%  Part of the iHELP project at http://ihelp.r-forge.r-project.org
%
%  Copyright (C) 2013- The iHELP Working Group 
%                                in the Korean R Translation Team
%
%  This program is free software; you can redistribute it and/or modify
%  it under the terms of the GNU General Public License as published by
%  the Free Software Foundation; either version 2 of the License, or
%  (at your option) any later version.
%
%  This program is distributed in the hope that it will be useful,
%  but WITHOUT ANY WARRANTY; without even the implied warranty of
%  MERCHANTABILITY or FITNESS FOR A PARTICULAR PURPOSE.  See the
%  GNU General Public License for more details.
%
%  A copy of the GNU General Public License is available at
%  http://www.r-project.org/Licenses/
%


%%%%%%%%%%%%%%%%%%%%%%%%%%%%%%%%%%%%%%%%%%%%%%%%%%%%%%%%%%%%%%%%%%%%%%%%
%
%
%
%
%%%%%%%%%%%%%%%%%%%%%%%%%%%%%%%%%%%%%%%%%%%%%%%%%%%%%%%%%%%%%%%%%%%%%%%%

\documentclass[../tutorial.tex]{subfiles}
\begin{document}


\part{통계모형의 이해와 적용}

\chapter{통계모형의 선택 및 적용}

각 섹션은 다음과 같은 방법으로 이루어져야 합니다. 
\begin{itemize}
\item 아래의 모형이 어느 경우에 사용되어야 하는가?
\item 모형을 사용하는데 있어서 요구되는 가정들은 무엇인가?
\item 모형의 계수에 대한 추정치는 어떻게 구하는가?
\item 모형의 진단
\item 추정치들에 대한 해석
\item 사용법들 
\item 모형에 대한 한계점
\end{itemize}

\section{분산분석 (Analysis of Variance-Covariance)}
\begin{Schunk}
\begin{Soutput}
output
\end{Soutput}
\end{Schunk}

\section{상관분석 (Correlation Analysis) }
\begin{Schunk}
\begin{Soutput}
output
\end{Soutput}
\end{Schunk}

\section{회귀분석 (Regression Analysis) }
\begin{Schunk}
\begin{Soutput}
output
\end{Soutput}
\end{Schunk}

\section{주성분분석 (Principle Component Analysis)}
\begin{Schunk}
\begin{Soutput}
output
\end{Soutput}
\end{Schunk}

\section{판별분석 (Discriminant Analysis) }
\begin{Schunk}
\begin{Soutput}
output
\end{Soutput}
\end{Schunk}

\section{군집분석 (Cluster Analysis) }
\begin{Schunk}
\begin{Soutput}
output
\end{Soutput}
\end{Schunk}

\section{시계열분석 (Time-Series Analysis) }
\begin{Schunk}
\begin{Soutput}
output
\end{Soutput}
\end{Schunk}

\section{일반선형모델 (GLM)}
\begin{Schunk}
\begin{Soutput}
output
\end{Soutput}
\end{Schunk}

\section{의사결정 나무(Decision Tree)}
\begin{Schunk}
\begin{Soutput}
output
\end{Soutput}
\end{Schunk}

\section{Longitudinal data analysis}
\begin{Schunk}
\begin{Soutput}
output
\end{Soutput}
\end{Schunk}

\section{생존분석 (Survivial analysis)}
\begin{Schunk}
\begin{Soutput}
output
\end{Soutput}
\end{Schunk}

\section{Mixture and latent class analysis}
\begin{Schunk}
\begin{Soutput}
output
\end{Soutput}
\end{Schunk}

\section{신경망 분석}
\begin{Schunk}
\begin{Soutput}
output
\end{Soutput}
\end{Schunk}

\section{기계학습 (Machine Learning)}
\begin{Schunk}
\begin{Soutput}
output
\end{Soutput}
\end{Schunk}

\section{메타 분석 (Meta Analysis)}
\begin{Schunk}
\begin{Soutput}
output
\end{Soutput}
\end{Schunk}


%%%%%%%%%%%%%%%%%%%%%%%%%%%%%%%%%%%%%%%%%%%%%%%%%%%%%%%%%%%%%%%%%%%%%%%%%%%%%%%%%%%
%
%
%
%%%%%%%%%%%%%%%%%%%%%%%%%%%%%%%%%%%%%%%%%%%%%%%%%%%%%%%%%%%%%%%%%%%%%%%%%%%%%%%%%%%

\section{패키지 관리}
\begin{enumerate}
\item 	이와 반대로 현재 연결된 라이브러리를 떼어낼 수도 있습니다. 

	\begin{Schunk}
	\begin{Soutput}
	> detach(package:pkg_name)	
	\end{Soutput}
	\end{Schunk}

	% 웹에 > detach(package:pkg_name) 과 5. 패키지를 설치 (분류: 사용자 환경)이 한 줄 띄어져야 함.

	\item 패키지를 설치 (분류: 사용자 환경)  
	
	\textsf{(답변)} 설치되는 패키지의 \textbf{설치위치}와 \textbf{의존성}에 대해서 반드시 알아야 합니다. 
	
	\begin{Schunk}
	\begin{Soutput}
	> install.packages("패키지명", dependencies=TRUE, )
	\end{Soutput}
	\end{Schunk}
% 을 쓰지 못하는 사례도 허다하다. package를 설치할 때 메뉴에서 [패키지 -> 패키지 설치하기]를 선택하고 난 뒤
% [mirror]를 선택한 뒤 패키지 리스트에서 하나씩 어디선가 본 패키지 이름을 어렵게 어렵게 찾아 더블클릭하는 절차를 따르는 것이다.

	\item 설치된 패키지의 목록을 확인하는 방법을 알고 싶습니다.
\end{enumerate}


\end{document}


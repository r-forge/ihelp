\batchmode
\makeatletter
\def\input@path{{/home/jhshin/Downloads/iHELP/ihelp/branch-krug/shin.jonghwa//}}
\makeatother
\documentclass[,noae]{report}
\usepackage[T1]{fontenc}
\usepackage[utf8]{inputenc}
\setcounter{secnumdepth}{3}
\setcounter{tocdepth}{3}

\makeatletter
%%%%%%%%%%%%%%%%%%%%%%%%%%%%%% Textclass specific LaTeX commands.
\usepackage{noweb}

%%%%%%%%%%%%%%%%%%%%%%%%%%%%%% User specified LaTeX commands.
\usepackage{kotex}

\makeatother


\usepackage{Sweave}
\begin{document}

\title{r-forge.r-project.org subversion 사용법}


\author{신종화}

\maketitle

\part{iHELP 프로젝트}


\chapter{계정취득}


\chapter{subversion 사용법}


\section{Ubuntu}


\subsection{subversion 프로그램 설치}
\begin{itemize}
\item sudo apt-get install subversion
\end{itemize}

\section{iHELP repository 접속방법}
\begin{enumerate}
\item 사용자는 본인의 컴퓨터에 iHELP repository를 다운로드 받을 디렉토리를 만든다.

\begin{enumerate}
\item 예: mkdir /home/사용자이름/Downloads/iHELP
\end{enumerate}
\item 위에서 만든 디렉토리로 이동한다.
\item \textquotedbl{}svn checkout svn+ssh://사용자계정@r-forge.r-project.org/svnroot/ihelp/\textquotedbl{}을
입력한다. 

\begin{enumerate}
\item ihelp 프로젝트의 홈페이지에 접속방법이 나와있다.
\end{enumerate}
\end{enumerate}

\section{iHELP repository 업로드 방법}
\begin{enumerate}
\item 이 파일의 위치는 ihelp/branch-krug/shin.jonghwa/일 것이다.
\item 이 파일의 이름은 ihelp.svn.readme.texi 일 것이다.
\item 파일(예: ihelp.svn.readme.texi)을 디렉토리에 추가해야 한다.

\begin{enumerate}
\item svn add ``ihelp.svn.readme.texi''
\end{enumerate}
\item 파일을 원격지에 있는 ihelp 저장소로 올린다.

\begin{enumerate}
\item svn commit ``ihelp.svn.readme.texi''\end{enumerate}
\end{enumerate}

\end{document}

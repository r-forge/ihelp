\documentclass{article}

\title{First Tex}
\author{iHELP Working Group}

% 아 안된다... tex에서 한글사용 어렵다.... 특히 윈도우즈라서 그런가.., 으으...
% 그래도 난 윈도우즈에서 이걸로다가 한글 해낼거다!! ㅋㅋㅋㅋ 미췬다~~~ 헤롱헤롱~~~ ㅔ매냐픠;자허ㅢ;ㅏ어

\begin{document}
	Welcome!
	eererfdfdf	이 문서는 첫번째 테스트 페이지 입니다.
	
% texniccenter에서 빌드는 되는데 왜 pdf형성이 안될까.. 이제 한글 에러 안뜨는데.. 으으..

% R은 처음부터 기존의 통계팩키지와는 다른 모습에 약간 두렵기까지 하다..
% 기존의 분석은 일반적으로 [프로그램 실행 -> 데이터 불러오기 -> 분석(메뉴클릭:SPSS 또는 명령어입력:SAS) -> 실행]의 절차를 밟아왔다.
% 즉 할 거 다하고 나중에 결과를 보는 식이다.
% 그러나 R은 다르다. 일단 여기부터 생소하다.
% 뭔가 이상하다.. 하는 거 같은 느낌도 없다. 데이터를 불러오기 하면 바로 데이터시트를 볼 수 있는 것도 아니다.
% 물론 나중에는 스크립트를 이용하여 비슷한 방법으로 분석을 실행할 수도 있음을 알게 되었지만 이것도 친숙한 편은 아니다.
% 이런 느낌은 특히 최소한 내가 알기로는 경영학과 경제학 분야(거의 확실함) 및 사회학, 심리학 분야가 특히 심하다. 그리고 이들 분야에서는
% R의 다양한 기능(특히 visualization) 또는 packages를 필요로 하지 않을 가능성이 높다. 왜냐하면 분석 방법이 단순하기 때문이다.
% 여기에서 분석 방법이 단순하다는 것은 그 이론적 배경이 쉽다는 것을 의미하는 것이 아니라 절차상의 얘기이다.
% 언급한 분야에서 학술연구에서 실행되는 분석의 양을 보면 대부분 [기술통계 - 상관관계 - 차이분석, 회귀분석, 구조방정식분석] 정도로
% 분석이 마무리 되기 때문이다. 그리고 대부분 표(table)로 결과를 보고하는 형식을 따르고 있다.
% 분석 모형 자체를 다루는 연구가 거의 없는 것이다. 따라서 R의 작업 방식은 그들에게는 불필요할 지도 모른다.

% 그러나 여전히 R의 다양한 source를 필요로 하는 아주 작은 규모의 학술연구 분야는 존재한다. 따라서 여기의 tips에는 이들을 고려하여
% R 사용에 도움이 될 만한 사항들을 정리하고자 한다.
% 개인적으로는 통계학 및 다른 분야의 연구 과정이 어떻게 진행되는지 잘 모르는 것도 한 몫하여 이에 해당하는 부분은 다른 이의 도움이
% 절실하다.

% 덧붙여 나는 윈도우 환경에서의 R tip에 주목하고자 한다(Mac이나 유닉스/리눅스 환경에서의 R 사용에 관한 tip도 다른 이의 도움이 필요한
% 부분이다). 그 이유는 내가 윈도우에서 SPSS와 SAS를 숙련시켜 왔기 때문에 대부분의 통계분석 초심자들도 윈도우 환경에서 R 을 접하게
% 될 것이라는 막연한 억측 때문이다. 시간이 지나면서 여러 사람들을 만나보게 꼭 그렇지만은 않다는 것을 알게 되었지만 최소한 경영학과
% 경제학 분야에 속한 사람들은 그럴 것이라는 확신 때문이다. 이것은 다른 분야의 사람들을 홀대하는 것이 아니라 나의 능력의 한계 때문이다.
% 실제로 나의 경우에도 최근에 C 언어, CentOS, 그리고 MySQL도 공부하기 시작했으니 오해는 하지 말기 바란다(누군가는 포기하라고도 했다.
% 왜냐하면 지금 그것들을 공부하기에는 분량이 너무 많고 난이도가 낮지 않기 때문이다). 그런데 결국에는 내 전공분야에서의 R 사용은
% 그 분량이 많지 않고 리눅스/유닉스 기반의 R 사용에 보다 많은 시간과 지면을 할애하게 될 것이라는 생각이 든다.

% R은 최근의 빅데이터라는 화두와 함께 큰 주목을 받고 있다. 하지만 대부분의 사람들, 특히 사회과학 전공자들은 이 개념적 정의도 제대로
% 알지 못하며 들어보지 못한  전문용어(SQL, Hadoop, 그리고 Mapreduce 등)에 혹하고 있는 실정이다.
% 나의 경우 R을 구글링(googling)으로 공부하였다. 왜냐하면 내가 재학하던 대학에서는 R 교육 프로그램이 없었기 때문이다. 처음 접하게 된
% 것은 미시경제분석 수업이었다. 그 수업에서 교수님이 수업을 진행하시는데 R을 사용한 것이다(SAS도 조금 사용했었으나 SPSS 는 잡동사니
% 취급을 하셨던 것으로 기억한다). 누군가에게 강의를 받는 식의 교육은 그것이 전부였다. 이후의 학습은 모두 맨땅에 해딩과 계란으로
% 바위깨기 식이었다. 이 R tips 과정은 이러한 어려움을 덜어주기 위한 방안이 될 수도 있을 것이다.
% 그러기를 한참 후 어느 정도 익숙해지고 나니, 또 연구자로서의 건방이 살아나 R 교육이 어떻게 이루어지고 있는 지가 궁금해졌다.
% 이 부분은 나중에 다시 써 볼라오,,,

% 또한 위의 내용과 관련하여,
% > packages.install()
% 을 쓰지 못하는 사례도 허다하다. package를 설치할 때 메뉴에서 [패키지 -> 패키지 설치하기]를 선택하고 난 뒤
% [mirror]를 선택한 뒤 패키지 리스트에서 하나씩 어디선가 본 패키지 이름을 어렵게 어렵게 찾아 더블클릭하는 절차를 따르는 것이다.

% 보다 더 엄청난 것은 여러 참고용 문서들에서 운영체제에 따른 프롬프트들을 알려주지 않은 이유로 프롬프트인지 뭔지를 구분하지 못하는
% 경우도 있다.

% 데이터 불러오기
% 보통 사람들은 R 콘솔에서 직접 입력한 데이터를 제외하고 그 외의 데이터는 *.txt, *.xlsx 이든 *.sav이든 모두 외부데이터로 생각하기
% 일쑤이다.
% 즉 foreign 팩키지를 사용해야 하는 경우를 제대로 구분하지 못한다. 그러니,
% > library(foreign)
% 을 선언할 줄 모르는 것은 당연하다.

% 그리고 인코딩이란 무엇인지를 모르는 사람은 보다 맞다.

% R에서 편집한 데이터를 엑셀로 내보내는 방법[새로운 변수를 생성해서 기존 데이터에 더하려고 하는데
% (csv로 R에서 불러서 편집한 자료)]
% 

% interacrive한 작업 방식을 취하는 R은 객체 관리에 혼란은 느끼는 경우가 많다. 특히 동시에 작업을 한다면 이루 말할 수 없을 정도이다. % 따라서 작업 중에 살아있는 객체들이 무엇이 있는지 파악하고 있는 것이 중요하다. 따라서 프로세스 차트 등을 먼저 작성해 놓은 것이 좋다


% 통계컴퓨팅을 하는 사람들의 분류도 미리 해 보는 것이 좋겠다.
% 유저에 따라 필요도 달라질 것이기 때문...
% 예를 들어, 데이터 마이닝은 CRM을 위한 기법이지만 CRM은 경영학에서 다루는 개념이고 데이터마이닝은 통계학 쪽이다.
% CRM을 연구하는 경영학자는 연구 논문에서 흔히 사용되는 분석기법이면 연구를 하는데 지장이 없으나 통계학자들은 
% 데이터마이닝 기법이 더 필요한 것..
	
\end{document}

\documentclass{article}



\usepackage{verbatim}
\usepackage{Sweave}
\usepackage{amsmath, amssymb, amsthm}
\usepackage{kotex}

\title{First Tex}
\author{iHELP Working Group}

\begin{document}


	Welcome!
	eererfdfdf	이 문서는 첫번째 테스트 페이지 입니다.
	
% texniccenter에서 빌드는 되는데 왜 pdf형성이 안될까.. 이제 한글 에러 안뜨는데.. 으으..
%% (답변) 문서에 꼭  \usepackage{kotex}를넣어야 함.


% R은 처음부터 기존의 통계팩키지와는 다른 모습에 약간 두렵기까지 하다..
% 기존의 분석은 일반적으로 [프로그램 실행 -> 데이터 불러오기 -> 분석(메뉴클릭:SPSS 또는 명령어입력:SAS) -> 실행]의 절차를 밟아왔다.
% 즉 할 거 다하고 나중에 결과를 보는 식이다.
% 그러나 R은 다르다. 일단 여기부터 생소하다.
% 뭔가 이상하다.. 하는 거 같은 느낌도 없다. 데이터를 불러오기 하면 바로 데이터시트를 볼 수 있는 것도 아니다.
% 물론 나중에는 스크립트를 이용하여 비슷한 방법으로 분석을 실행할 수도 있음을 알게 되었지만 이것도 친숙한 편은 아니다.
% 이런 느낌은 특히 최소한 내가 알기로는 경영학과 경제학 분야(거의 확실함) 및 사회학, 심리학 분야가 특히 심하다. 그리고 이들 분야에서는
% R의 다양한 기능(특히 visualization) 또는 packages를 필요로 하지 않을 가능성이 높다. 왜냐하면 분석 방법이 단순하기 때문이다.
% 여기에서 분석 방법이 단순하다는 것은 그 이론적 배경이 쉽다는 것을 의미하는 것이 아니라 절차상의 얘기이다.
% 언급한 분야에서 학술연구에서 실행되는 분석의 양을 보면 대부분 [기술통계 - 상관관계 - 차이분석, 회귀분석, 구조방정식분석] 정도로
% 분석이 마무리 되기 때문이다. 그리고 대부분 표(table)로 결과를 보고하는 형식을 따르고 있다.
% 분석 모형 자체를 다루는 연구가 거의 없는 것이다. 따라서 R의 작업 방식은 그들에게는 불필요할 지도 모른다.

% 그러나 여전히 R의 다양한 source를 필요로 하는 아주 작은 규모의 학술연구 분야는 존재한다. 따라서 여기의 tips에는 이들을 고려하여
% R 사용에 도움이 될 만한 사항들을 정리하고자 한다.
% 개인적으로는 통계학 및 다른 분야의 연구 과정이 어떻게 진행되는지 잘 모르는 것도 한 몫하여 이에 해당하는 부분은 다른 이의 도움이
% 절실하다.

% 덧붙여 나는 윈도우 환경에서의 R tip에 주목하고자 한다(Mac이나 유닉스/리눅스 환경에서의 R 사용에 관한 tip도 다른 이의 도움이 필요한
% 부분이다). 그 이유는 내가 윈도우에서 SPSS와 SAS를 숙련시켜 왔기 때문에 대부분의 통계분석 초심자들도 윈도우 환경에서 R 을 접하게
% 될 것이라는 막연한 억측 때문이다. 시간이 지나면서 여러 사람들을 만나보게 꼭 그렇지만은 않다는 것을 알게 되었지만 최소한 경영학과
% 경제학 분야에 속한 사람들은 그럴 것이라는 확신 때문이다. 이것은 다른 분야의 사람들을 홀대하는 것이 아니라 나의 능력의 한계 때문이다.
% 실제로 나의 경우에도 최근에 C 언어, CentOS, 그리고 MySQL도 공부하기 시작했으니 오해는 하지 말기 바란다(누군가는 포기하라고도 했다.
% 왜냐하면 지금 그것들을 공부하기에는 분량이 너무 많고 난이도가 낮지 않기 때문이다). 그런데 결국에는 내 전공분야에서의 R 사용은
% 그 분량이 많지 않고 리눅스/유닉스 기반의 R 사용에 보다 많은 시간과 지면을 할애하게 될 것이라는 생각이 든다.

% R은 최근의 빅데이터라는 화두와 함께 큰 주목을 받고 있다. 하지만 대부분의 사람들, 특히 사회과학 전공자들은 이 개념적 정의도 제대로
% 알지 못하며 들어보지 못한  전문용어(SQL, Hadoop, 그리고 Mapreduce 등)에 혹하고 있는 실정이다.
% 나의 경우 R을 구글링(googling)으로 공부하였다. 왜냐하면 내가 재학하던 대학에서는 R 교육 프로그램이 없었기 때문이다. 처음 접하게 된
% 것은 미시경제분석 수업이었다. 그 수업에서 교수님이 수업을 진행하시는데 R을 사용한 것이다(SAS도 조금 사용했었으나 SPSS 는 잡동사니
% 취급을 하셨던 것으로 기억한다). 누군가에게 강의를 받는 식의 교육은 그것이 전부였다. 이후의 학습은 모두 맨땅에 해딩과 계란으로
% 바위깨기 식이었다. 이 R tips 과정은 이러한 어려움을 덜어주기 위한 방안이 될 수도 있을 것이다.
% 그러기를 한참 후 어느 정도 익숙해지고 나니, 또 연구자로서의 건방이 살아나 R 교육이 어떻게 이루어지고 있는 지가 궁금해졌다.
% 이 부분은 나중에 다시 써 볼라오,,,

% 또한 위의 내용과 관련하여,
% > packages.install()
% 을 쓰지 못하는 사례도 허다하다. package를 설치할 때 메뉴에서 [패키지 -> 패키지 설치하기]를 선택하고 난 뒤
% [mirror]를 선택한 뒤 패키지 리스트에서 하나씩 어디선가 본 패키지 이름을 어렵게 어렵게 찾아 더블클릭하는 절차를 따르는 것이다.

% 보다 더 엄청난 것은 여러 참고용 문서들에서 운영체제에 따른 프롬프트들을 알려주지 않은 이유로 프롬프트인지 뭔지를 구분하지 못하는
% 경우도 있다.

% 데이터 불러오기
% 보통 사람들은 R 콘솔에서 직접 입력한 데이터를 제외하고 그 외의 데이터는 *.txt, *.xlsx 이든 *.sav이든 모두 외부데이터로 생각하기
% 일쑤이다.
% 즉 foreign 팩키지를 사용해야 하는 경우를 제대로 구분하지 못한다. 그러니,
% > library(foreign)
% 을 선언할 줄 모르는 것은 당연하다.

% 그리고 인코딩이란 무엇인지를 모르는 사람은 보다 맞다.

% R에서 편집한 데이터를 엑셀로 내보내는 방법[새로운 변수를 생성해서 기존 데이터에 더하려고 하는데
% (csv로 R에서 불러서 편집한 자료)]
% 

% interacrive한 작업 방식을 취하는 R은 객체 관리에 혼란은 느끼는 경우가 많다. 특히 동시에 작업을 한다면 이루 말할 수 없을 정도이다. % 따라서 작업 중에 살아있는 객체들이 무엇이 있는지 파악하고 있는 것이 중요하다. 따라서 프로세스 차트 등을 먼저 작성해 놓은 것이 좋다


% 통계컴퓨팅을 하는 사람들의 분류도 미리 해 보는 것이 좋겠다.
% 유저에 따라 필요도 달라질 것이기 때문...
% 예를 들어, 데이터 마이닝은 CRM을 위한 기법이지만 CRM은 경영학에서 다루는 개념이고 데이터마이닝은 통계학 쪽이다.
% CRM을 연구하는 경영학자는 연구 논문에서 흔히 사용되는 분석기법이면 연구를 하는데 지장이 없으나 통계학자들은 
% 데이터마이닝 기법이 더 필요한 것..
	
	
% Eugene Jung님, 의견 물어봐 주셔서 감사드립니다. 저의 전반적인 의견은 다소 R의 한글화가 과연 산학연 차원의 과제인지에 대한 "매우 강한 의구심"이 있습니다. 제가 말 주변이 좋지 못해 좋게 표현은 하지 못하지만, 제 생각이 이렇다고만 참고를 해 주신다면 더할 나위 없이 감사하는 바램으로 저의 의견을 남겨봅니다. 저는 토론을 함이지 정우준님께 어떤 의도 및 감정이 있는 것은 아니니 부디 편히 받아주시길 다시 한 번 부탁드립니다. 제가 현재 알고 싶은 것은 정우준님께서 말씀하신 "R의 한글화"가 무엇을 의미하는 것인지 정확히 알고 싶습니다. R 시스템 자체를 한국어로 사용할 수 있게 개발하는 것을 의미하는 것을 말씀하시나요? 아니면, R을 보다 쉽게 활용할 수 있도록 메시지 및 문서 관련 한국어 서비스를 제공하는 것인가요? 한글화에 대한 잘못된 개념의 시작은 추후에 일의 중복 및 표준화에 엄청난 부작용을 낳습니다. 전자의 경우에 일반적으로 한국어로 보임은 보여지는 메시지와 데이터를 읽어들이는 인코딩를 한국어로 국한한다는 이미지를 가질 수 있습니다. 이는 국제표준에 적합하지 않으며, 협력관계를 추후에 도모하기 어렵습니다. 현재 R은 UTF-8 기반으로 일반 리눅스에서 사용되는 방식으로 언어에 대한 입출력을 관리합니다. 이에 대한 자세한 관련사항들은 http://ihelp.r-forge.r-project.org/lang.html 를 참고 부탁드립니다. 후자의 경우가 전자의 경우보다는 조금 더 합리적인것과 같이 보입니다. 메시지와 문서를 따로 분리하여 한글작업을 하는 것은 두가지로 생각되어 질 수있습니다. 원래의 영문소스를 건드리지 않고 단순히 언어팩을 따로 지원하는 것은 철학적으로는 오픈소스에 대하여 기존에 작업한 기존 작업자들의 노고에 대한 일종의 감사의 표현입니다. 실제적인면에서는 협력작업에 대한 표준을 건드리지 않았기 때문에 일의 중복을 막는 효과도 있습니다. 만약, 정우준님께서 R의 한글화 작업이라고 말씀하시는 프로젝트에서 메시지가 첫번째 태스크라면 이미 85%이상이 되어 있습니다. 만약, 문서가 또다른 태스크라면 이는 R을 정말 잘 알고 있는 사람이 작업을 해야할 것입니다.

% R 언어는 다른 프로그래밍 언어와 어떤 일을 수행함에 있어서 도구에 지나지 않는다는 것은 크게 다른 점이 없습니다. 따라서, 전산분야에서 새로운 언어라고 생각하고 배우면 정말 C처럼 배울 수도 있습니다. 그러나, 다른점이라고 한다면 이를 잘 활용하기 위해서는 데이터를 보는 통계적 지식이 있으면 좋다고 할 수 있을 것입니다. 그 이유는 기본적으로 데이터는 정보가 아니기 때문입니다. 통계는 랜덤한 특징을 가진 데이터로부터 어떤 의미를 포함한 정보를 시스템적인 요소를 찾는 수학적 표현을 통해 얻는 기술이기 때문입니다. 더 나아가 이런 정보를 어떻게 합리적으로 객관적으로 도출하는데 어떻게 시스템적인 프레임워크를 가지고 설계가 가능할까 하는데 사용하는것이 R이기 때문입니다. 이러한 면에서 대용량 자료처리를 위한 강점을 가진 SAS의 행벡터 처리방식보다 R은 통계적 철학에 바탕을 둔 열벡터 처리방식을 취합니다. 또한, list, factor, data.frame등과 같이 다양한 데이터형을 제공하는데는 설계자체에 통계적 배경이 녹아 있기 때문입니다. 그러나, 행렬의 수행방식등은 Matlab과 동일합니다. 이들은 모두 S 언어의 특징에 기반을 둡니다. 따라서, 한국어 문서작업에는 S언어를 알고 계시거나 파이썬과 같이 리스트 형식의 언어를 아시는 분들이 R의 기술적 요소에 대한 설명을 추가해 주시는 것은 매우 도움이 되며, 통계를 아시는 분들께서 도움을 주신다면 더할 나위 없이 좋을 것입니다. 따라서, 한국어 문서작업을 추진하신다면 문서를 읽는 독자가 어떤 목적으로 아는 것이 더 중요합니다.

% 독자가 과연 R 자체의 기술적 개발을 위해서 R을 필요로 하는 것인지 아니면 분석에 필요한 패키지를 위주로 R을 사용하는 것에 대한 구분은 프로젝트 설계에 중요한 촛점이 됩니다. 그 이유는 패키지라는 어떤 특정한 통계적 방법과 모델을 구현한 것들이거나, 어떤 특정 데이터 프로세스를 처리하는 다양한 방법을 제공하는 하나의 라이브러리이기 때문입니다. 실제로는 통계적 방법면적에서 본다면 SAS 사용자 매뉴얼 혹은 SPSS의 매뉴얼, 더 나아가 대학교 통계학 교재만큼 좋은 것이 없습니다. 그 이유는 구현된 알고리즘들은 모두 동일하기 때문입니다. 굳이 R이 좋다고 해야 할 이유는 오로지 비용적 측면에서 밖에 없습니다. 결국은 R을 사용하는 분석자가 통계적 지식이 얼마나 있는가에 따라 모두 달라지게 되는 것입니다. 이는 우리가 어떤 프로젝트를 수행할때 프로젝트 자체보다도 프로젝트 수행원이 누구인지를 보는것과 동일하다고 생각하시면 더욱 좋을 것 같습니다. 일반적으로 R은 복잡한 데이터 프로세싱 및 시스템처리보다 분석적인 부분에서만 강점을 가집니다. 만약, 비정형 데이터 혹은 대용량의 서베이 데이터등을 처리하려면 Perl과 SQL로 통계분석에 요구되어지는 데이터 형식을 갖춘뒤, 이 데이터만을 가지고 R을 사용하는 것입니다. R은 만능이 아닙니다. 따라서, 한글 문서를 제공을 하는데 있어서는 크게 누가 어떤 목적으로 사용되어지는데 도움이 될까가 첫번째로 정해져야 하며, 이는 R 매뉴얼로는 매우 부족합니다. 이러한 경우에는 오히려 R을 이용하여 통계적 방법을 올바르게 수행하고 해석하는 케이스형 하우투를 제공하는 것이 더 도움이 될 것같습니다. 만약, 새로운 언어를 개발하는 것이 아니라면 R이라는 언어를 이용하여 이를 확장해 나가는 것입니다. 이런 경우에는 R의서 Extending R System 문서의 번역에 가중을 두는 것이 좋을 수도 있습니다. 그러나, 위에서 언급한 바와 같이 R도 다른 언어와 같이 알고리즘 수행을 위한 수단입니다. C 전문가가 C로 R보다 더 효율적인 알고리즘을 짤 수 있음에도 불구하고 R을 쓰는것은 아마도 바람직하지 않을 것입니다. 대체적으로 수치알고리즘등은 C/Fortran으로 씌여진 이전 라이브러리를 R로 가져오기 때문입니다.

% 따라서, 저의 전반적인 생각은 산학연 프로젝트 요청을 통한 작업이기 보다는 현재의 오픈소스 생태계 흐름대로 두는 것이 바람직하다고 봅니다. 그러나, 이러한 활동을 위하여 작업환경에 대한 기부방식을 통한 지원을 받는 것은 좋다고 생각합니다. 또한, 각분야에서 고유하게 발전되어온 통계교육에 좀 더 비중을 줄 수 있는 내용들이 다루어 진다면 더할 나위 없이 좋을 것같습니다. 
	
	
	
	
\end{document}

안내

이번의 프로젝트는 크게 다음의 세 가지 단계에 따라 이루어지게 됩니다.

데이터 다루기

분석하기

분석결과 표 그리기 및 그래프 그리기

1.  데이터 다루기

R에서 데이터는 크게 두 가지 방법으로 입력될 수 있습니다. 하나는 콘솔에서 집적 입력하는 것이고 다른 하나는 R 이외의 다른 프로그램을 통해 작성된 데이터를 R로 불러오는 것입니다.

1.1 콘솔에서 데이터 입력
1.2 외부 데이터 파일 가져오기
1.3 데이터 조작

1.1 R 콘솔에서 데이터 입력

가장 기초적인 방법으로 R 콘솔에서 데이터를 직접 입력하는 방법부터 익혀 보도록 하겠습니다. 우리가 일반적으로 접하게 되는 데이터는 숫자형 및 문자형 데이터이겠지만 R에서는 숫자형 데이터라도 실수인지 허수인지를 구분하고 이런 데이터들로 구성된 데이터세트를 스칼라, 벡터, 매트릭스, 배열, 리스트, 그리고 데이터프레임의 여섯 가지 구조로 구분합니다. 그리고 문자형 데이터도 어느 정도의 규칙을 지켜야 합니다.

스칼라의 경우는 제외하고 나머지 다섯 가지 형태의 데이터세트를 다루는 방법을 익혀 보도록 하겠습니다.

먼저, 여러분들은 각각 R 콘솔에서 숫자로만 구성된 데이터세트를 벡터, 매트릭스, 배열, 리스트, 그리고 데이터프레임의 형태로 만들어 주시기 바랍니다. 여기에서 크게 두 가지 방법이 있을 수 있습니다. 하나는 먼저 벡터부터 생성하여 데이터프레임까지 순차적으로 만드는 것이고 다른 하나는 각 데이터구조를 만드는 함수를 이용하여 만들수도 있습니다. 가급적 두 가지 방법으로 모두 만들어 보시기 바랍니다. 입력할 데이터의 개수는 총 100개(너무 많다고 생가되시면 40개) 정도로 하시면 됩니다. 임의로 만드셔도 되고 아니면 성적, 직원과 관련된 정보 또는 주변에서 쉽게 볼 수 있는 어떤 것 - 주변을 한 번 살펴 보시면 아주 많이 있습니다 - 을 이용하시면 됩니다.

이 작업을 하시면서 자신이 작성한 코드와 결과를 텍스트를 복사하시거나 화면캡쳐하여 제게 이메일로 보내주시고 이 작업을 실행하는 도중 발생하는 질문이나 에러는 ihelp-urquestion 메일링 시스템으로 보내주십시오. R cookbook으로 해결하실 수도 있으나 메일링 시스템을 통하여 해결해 드릴 수도 있습니다. 그리고 이러한 과정은 iHelp 웹페이지에서 정리된 형태로 설명되고 나타내어 질 것입니다. 아무리 사소한 것이라도 메일링 시스템을 자주 사용하시는 것을 권고합니다.

그럼 이만..

정우준 올림
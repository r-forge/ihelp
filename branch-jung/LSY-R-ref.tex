\documentclass[landscape,twocolumn,letterpaper]{article}

\setlength{\oddsidemargin}{0in}		% default=0in
\setlength{\textwidth}{9in}		% default=9in

\setlength{\columnsep}{0.5in}		% default=10pt
\setlength{\columnseprule}{1pt}		% default=0pt (no line)

\setlength{\textheight}{5.85in}		% default=5.15in
\setlength{\topmargin}{-0.15in}		% default=0.20in
\setlength{\headsep}{0.25in}		% default=0.35in

\setlength{\parskip}{1.2ex}
\setlength{\parindent}{0mm}

\usepackage{amsmath, amssymb, amsthm}
\usepackage{geometry}
\usepackage{kotex}

\usepackage{draftwatermark}
\SetWatermarkFontSize{10cm}
\SetWatermarkScale{5}
\SetWatermarkText{Ver. 0.3}

\title{R Reference Card}
\author{Sae Yeun Lee \\ CRM Consultant \\ AMA.COMM \\ Seoul, South Korea \\
mandooms@gmail.com \and
Chel Hee Lee \\ gnustats@gmail.com }
\begin{document}
\maketitle

By Tom Short, EPRI PEAC, tshort@epri-peac.com 2004-11-07. Granted to the public
domain. See www.Rpad.org for the source and the latest version.
We appreciate for sharing this wonderful R reference card.
This is a simple translation of Tom's original reference card into Korean.

이 문서는 Tom Short (tshort@epri-peac.com) 에 의해서 최초로 작성되어 공개된 문서입니다.
원본소스는 www.Rpad.org 에서 찾을수 있습니다.
이문서는 단순히 Tom의 원본을 번역한 것입니다.

\section{Getting help}
대부분의 R 함수는 온라인 도움말이 있습니다.

\textbf{help(topic)} 주제와 관련된 문서를 찾습니다.

\textbf{?topic} 위와 같음.

\textbf{help.search("topic")} 도움말 시스템에서 찾습니다.

\textbf{apropos("topic")} 주제와 관련있는 오브젝트를 찾아냅니다.

\textbf{help.start()} HTML버전의 도움말을 시작합니다.

\textbf{str(a)} R 오브젝트의 내부 구조를 보여줍니다.

\textbf{summary(a)} 일반적으로는 통계량의 요약을 보여주지만, 오브젝트의 특성에 따라 다르게 작동합니다.

\textbf{ls()} 검색경로의 오브젝트를 보여줍니다;

\textbf{ls.str()} 검색경로의 오브젝트에 대한 내부 구조를 보여줍니다.

\textbf{dir()} 현재 사용중인 디렉터리를 보여줍니다.

\textbf{methods(a)} 오브젝트 a에 대한 S3방법을 보여줍니다.

\textbf{methods(class=class(a))} class a에 대한 사용한 가능한 방법들을 보여줍니다.


\section{Input and output}

\textbf{load()} 저장해놓은 데이터셋을 불러옵니다.

\textbf{data(x) }특정 데이터셋을 불러옵니다.

\textbf{library(x)} 추가된 패키지를 불러옵니다.

\textbf{read.tanle(file)} 테이블 형태나 데이터 프레임 형태로 된 파일을 읽습니다;
	기본으로 설정된 구분자(separator)는 공백입니다;
	header=TRUE 는 가장 첫번째 행을 열의 이름으로 인식합니다;
	as.is=TRUE는 문자벡터가 요인변수로 변환되는 것을 방지합니다;
	comment.char=" "를 사용하면, "\#"이 코멘트로 인식되는 것을 막아줍니다;
	skip=n 은 데이터에서 n번째 행을 제외하고 불러옵니다;
	도움말을 참조하면 행이름이나 NA처리방법, 기타 다른 옵션에 대해 알 수 있습니다.

\textbf{read.csv("filename", header=TRUE) id.} 하지만 기본설정은 쉼표로 분리된 파일입니다.

\textbf{read.delim("filename", header=TRUE) id.} 하지만 기본설정은 탭으로 분리된 파일입니다.

\textbf{read.fwf(file, widths, header=FALSE, sep="", as.is=FALSE)}
	테이블의 가로길이가 정해져있는 파일을 불러옵니다; 가로길이는 정수이어야 하고,
	고정된 값으로 주어져야 합니다.

\textbf{save(file, ...)} 특정 오브젝트 (...) 를 XDR 플랫폼의 독립적인 binary 형식으로 저장합니다.

\textbf{save.image(file) }모든 오브젝트를 저장합니다.

\textbf{cat(..., file="", sep=" ") }요소들을 문자형식으로 바꾸어 출력합니다; sep는  요소들을 구분하는
옵션입니다.

\textbf{print(a, ...)} 요소들을 출력합니다; 오브젝트의 형식이 다르면 출력방법도 달라집니다.

\textbf{format(x,...)} 오브젝트를 출력하기에 적절한 형태로 보여줍니다.

\textbf{write.table(x, file="", row.names=TRUE, col.names=TRUE, sep=" ")}
	x를 데이터프레임 형태로 변환한 뒤 출력합니다;

\section{Data creation}

\textbf{c(...)} 벡터 형식의 결합된 인자들을 생성하는 함수입니다;
	recursive=True 옵션을 사용하면 모든 요소를 하나의 벡터로 from:to 수열을 만듭니다; ":" 기호가 우선권을
갖습니다; 1:4 + 1은 "2, 3, 4, 5"가 됩니다.

\textbf{seq(from, to)} by= 로 지정해준 크기만큼 증가하는 수열을 생성합니다;
	length= 를 사용해서 길이를 지정할 수 있습니다.

\textbf{seq(along=x) 1, 2, ..., length(along)} 의 수열을 생성합니다; loops를 사용할 때 유용합니다.

\textbf{rep(x, times)} x를 times만큼 반복합니다; each= 옵션은 각각의 요소들을 해당 값만큼 반복합니다;

\textbf{rep(c(1,2,3),2)} 은 1 2 3 1 2 3이 되고, rep(c(1,2,3), each=2)는 1 1 2 2 3 3이
됩니다.

\textbf{data.frame(...)} 이름이 있거나 없는 데이터 프레임을 만듭니다;
	data.frame(v=1:4, ch=c("a", "B", "c", "d"), n=10);
	길이가 맞지 않는 경우에는 긴 쪽에 맞춰서 반복됩니다.

\textbf{list(...)} 이름이 있거나 없는 리스트를 만듭니다;
	list(a=c(1,2), b="hi", c=3i);

\textbf{array(x, dim)} 데이터 x를 가지고 배열을 생성합니다;
	세부적인 차원은 dim=c(3,4,2) 등과 같이 설정할 수 있습니다;
	x의 길이가 충분하지 않을 때는 x가 반복해서 들어갑니다.

\textbf{matrix(x, nrow=, ncol=)} 행렬; x가 반복되는 행렬을 생성합니다.

\textbf{factor(x, levels=)} 벡터x를 요인변수로 바꾸어 줍니다.

\textbf{gl(n, k, length=n*k, labels=1:n)} 정해진 패턴에 따라 레벨을 만듭니다;
	n는 레벨의 단위이고, k는 반복될 횟수입니다.

\textbf{expand.grid()} 벡터나 요인의 모든 조합을 데이터 프레임으로 보여줍니다.

\textbf{rbind(...)}  행렬, 데이터 프레임, 혹은 다른 형태의 데이터를 행으로 결합합니다.

\textbf{cbind(...)} 위와 같습니다. 열로 결합합니다.


\section{Slicing and extracting data}

Indexing vectors

\textbf{x[n]} n번째 관측치

\textbf{x[-n]} n번째 관측치를 제외한 나머지

\textbf{x[1:n]} 처음부터 n번째까지의 관측치

\textbf{x[-(1:n)]} n+1번째 관측치부터 끝까지

\textbf{x[c(1, 4, 2)]} 특정 관측치

\textbf{x["name"]} "name"이란 이름을 가지고 있는 관측치

\textbf{x[x>3]} 3보다 큰 모든 관측치

\textbf{x[x>3 \& x<5]} 3보다 크고 5보다 작은 모든 관측치

%\textbf{x[x %in% c("a", "and", "the")] 주어진 set에 해당하는 관측치

\section{Indexing lists}

\textbf{x[n]} n번째 요소를 리스트로 보여줍니다.

\textbf{x[[n]]} 리스트의 n번째 요소를 보여줍니다.

\textbf{x[["name"]]} "name"이란 이름을 가지고 있는 요소를 보여줍니다.

\textbf{x\$name} 위와 같습니다.

\section{Indexing matrics}
\textbf{x[i,j]} i번째 행, j번째 열을 보여줍니다.

\textbf{x[i, ]} i번째 행을 보여줍니다.

\textbf{x[ ,j]} j번째 열을 보여줍니다.

\textbf{x[ ,c(1,3)]} 1열과 3열을 보여줍니다.

\textbf{x["name"]} 이름이 "name" 인 행을 보여줍니다.

\textbf{Indexing data frames (matrix indexing plus the following)}

\textbf{x[["name"]]} 이름이 "name"인 열을 보여줍니다.
\textbf{x\$name} 위와 같습니다.



\section{Variable conversion}

\textbf{as.array(x), as.data.frame(x), as.numeric(x), as.logical(x),
as.complex(x), as.character(x), ...}
타입을 변환합니다; 완전한 리스트의 경우에는 methods(as)를 사용합니다.

\section{Variable information}

\textbf{is.na(x), is.null(x), is.array(x), is.data.frame(x),
is.numeric(x),is.complex(x), is.character(x), ...}
타입을 확인할 때 사용합니다; 완전한 리스트의 경우에는 methods(is)를 사용합니다.

\textbf{length(x)} x의 길이를 확인합니다.

\textbf{dim(x)} 오브젝트의 차원을 설정합니다.

\textbf{dimnames(x)} 오브젝트의 차원의 이름을 설정합니다.

\textbf{nrow(x)} 행의 개수; NROW(x) 는 똑같지만 1행으로 구성된 행렬에 적용합니다.

\textbf{ncol(x)}, NCOL(x) 위와 같음. 열에 대해서 적용합니다.

\textbf{class(x)} x의 클래스를 설정하거나 확인합니다; class(x)<-"myclass"

\textbf{unclass(x)} x의 클래스를 삭제합니다.

\textbf{attr(x, which)} x의 어떤 속성을 설정합니다.

\textbf{attributes(obj) }obj의 속성의 리스트를 설정합니다.


\section{Data selection and manipulation}

\textbf{which.max(x)} x의 요소중에서 가장 큰 값을 불러옵니다.

\textbf{which.min(x) }x의 요소중에서 가장 작은 값을 불러옵니다.

\textbf{rev(x) }x를 거꾸로 배열합니다.

\textbf{sort(x) }오름차순에 따라서 x를 정렬합니다; 내림차순을 사용하고 싶으면 rev(sort(x))를 사용합니다.

\textbf{cut(x, breaks)} x를 간격(요인)에 따라 분리합니다; breaks는 분리되는 간격이나 지점의 개수를 의미합니다.

\textbf{match(x, y)} y안에 있는 x의 인자들과 길이가 같은 벡터를 반환합니다.

\textbf{which(x == a)}

\textbf{choose(n, l)} 조합(combination) 계산을 합니다. n!/[(n-k)!k!]

\textbf{na.omit(x)} 결측치(NA)를 표시하지 않습니다. (행렬이나 데이터프레임의 경우에는 결측치에 해당하는 행 전체를 표시하지
않습니다.)

\textbf{na.fail(x)} x가 하나라고 결측치를 가지고 있으면, 에러메세지를 나타냅니다.

\textbf{unique(x)} x가 벡터나 데이터 프레임의 형태이면, 반복되는 값을 제외한 형태로 보여줍니다.

\textbf{table(x)} x의 관측값의 개수를 테이블로 만들어줍니다.

\textbf{subset(x, ...)}

\textbf{sample(x, size)}

\textbf{prop.table(x, margin=)}


\section{Math}

\textbf{sin, cos, tan, asin, acos, atan, atan2, log, log10, exp}

\textbf{max(x)} x의 최대값을 구합니다.

\textbf{min(x)} x의 최소값을 구합니다.

\textbf{range(x)} x의 범위를 구합니다. c(min(x), max(x))

\textbf{sum(x)} x의 구성요소의 합을 구합니다.

\textbf{diff(x)} 벡터x의 관측치 사이의 차이를 계산합니다.

\textbf{prod(x)} x의 구성요소의 곱을 구합니다.

\textbf{mean(x)} x의 평균을 구합니다.

\textbf{median(x)} x의 중간값을 구합니다.

\textbf{quantile(x, probs=)} 주어진 확률에 따른 sample quantile을 구합니다.. (기본설정은 0, .25,
.5, .75, 1)

\textbf{weighted.mean(x, w)} 가중치w를 적용한 x의 평균을 구합니다.

\textbf{rank(x)} x의 랭크를 구합니다.

\textbf{var(x)} or cov(x) (n-1에서의) x의 분산을 구합니다. 만약에 x가 행렬이나 데이터 프레임이라면 분산-공분산
행렬을 계산합니다.

\textbf{sd(x)} x의 표준편차를 구합니다.

\textbf{cor(x)} x가 행렬이나 데이터 프레임일 경우, 상관계수 행렬을 보여줍니다.

\textbf{var(x, y) }or cov(x, y) x와 y의 공분산 혹은 x, y가 행렬이나 데이터 프레임인 경우에는 x의 열과 y의
열의 공분산을 계산합니다.

\textbf{cor(x, y)} x와 y사이의 선형 상관관계를 구하거나 행렬이나 데이터 프레임의 경우에는 상관관계행렬을 보여줍니다.

\textbf{round(x, n) }소수점 n번째자리까지 반올림을 합니다.

\textbf{log(x, base)} base를 밑으로 하는 로그 계산을 합니다.

\textbf{scale(x)}


\section{Matrices}

\textbf{t(x)} 전치행렬을 구합니다.

\textbf{diag(x)} 대각원소를 구합니다.

%*% 행렬의 곱셈을 합니다.

\textbf{solve(a, b)} x에 대한 a%*%x=b 를 구합니다.

\textbf{solve(a)} a의 역행렬을 구합니다.

\textbf{rowsum(x)} 오브젝트와 마찬가지로 행렬의 행에 대한 합을 구합니다; rowSums(x)는 좀 더 빠른 계산을 할 수
있습니다.

\textbf{colsum(x)}, colSums(x) 위와 같습니다..

\textbf{rowMeans(x)} row means의 빠른 버전입니다.

\textbf{colMeans(x)} 위와 같습니다.





\section{Advanced data processing}

\textbf{apply(X, INDEX, FUN=)}

\section{Strings}

\textbf{paste(...)} 벡터를 문자형으로 변환 후 연결합니다;

\textbf{sep= }는 구분자를 의미합니다(기본은 공백으로 되어있습니다.);

\textbf{collapse=}

\textbf{substr(x, start, stop)} substrings





\section{Dates and Times}

\textbf{Date} 클래스는 시간을 제외한 날짜정보를 가지고 있습니다.

\textbf{POSIXct}는 날짜와 지역마다의 시간대를 포함한 시간정보를 가지고 있습니다. 비교문(예를 들어 >), seq(),
difftime() 등이 유용하게 쓰입니다.
Date는 또한 +와 -가 가능합니다.

\textbf{?DateTimeClasses} 명령어를 사용하면 좀 더 많은 정보를 확인할 수 있습니다.
chron 패키지를 참고하세요.
as.Date(s) 그리고 as.POSIXct(s)
각각의 클래스로 변환합니다;
format(dt) 는 문자열로 바꿔줍니다.
기본적인 형식은 "2001-02-21"입니다.
These accept a second argument to specify a format for conversion.
몇 가지 형식은 아래와 같습니다.

%a, %A 요일의 축약된 이름과 전체이름을 보여줍니다.
%b, %B 월의 축약된 이름과 전체이름을 보여줍니다.
%d 이번 달의 몇 번째 날인지 보여줍니다. (01-31)
%H 시를 보여줍니다. (00-23형태)
%I 시를 보여줍니다. (01-12형태)
%j 1년 중 몇 번째 날인지 보여줍니다. (001-366)
%m 몇 월인지 보여줍니다. (01-12)
%M 몇 분인지 보여줍니다. (00-59)
%P 오전/오후를 알려줍니다.
%S 초를 알려줍니다. (00-61)


\section{최적화, 모델링}

\textbf{optim(par, fn, method = c("Nelder-Mead", "BFGS", "CG", "L-BFGS-B",
"SANN")} 일반적인
목적의 최적화 함수;par는 초기값, fn은 최적화를 위한 함수 (normally minimize)

\textbf{nlm(f,p)} Newton-type 알고리즘을 이용해서 함수 f를 최소화시켜줍니다. P는 초기값입니다.

\textbf{lm(formula)} 선형모델을 적합시킨다; 종속변수의 기본적인 형태는 termA + termB + ... 입니다; I(x*y)
+ I(x?2)이런 형식을 사용하면 비선형 모델을 만들 수 있습니다.

\textbf{glm(formula,family=)} fit generalized linear models, specified by giving
a symbolic description of the linear predictor and a description of
the error distribution; family is a description of the error distribution
and link function to be used in the model; ?family를 참고하세요.

\textbf{nls(formula)} 비선형모델에 대한 추정을 합니다.

\textbf{approx(x,y=)} 주어진 자료를 선형으로 interpolate 합니다.  x can be an xy plotting

\section{structure}

\textbf{spline(x,y=)} cubic spline interpolation

\textbf{loess(formula)} fit a polynomial surface using local fitting
Many of the formula-based modeling functions have several common arguments:
data= the data frame for the formula variables, subset= a subset of
variables used in the fit, na.action= action for missing values: "na.fail",
"na.omit", or a function. 다음은 모델을 적합시킬 때 자주 사용되는 일반적인 함수들입니다:

\textbf{predict(fit,...)} 적합모델에 대한 예측을 합니다

\textbf{df.residual(fit)} 잔차에 대한 자유도를 구합니다

\textbf{coef(fit) }추정계수를 반환합니다(표준오차와 함께 나오는 경우도 있습니다)

\textbf{residuals(fit)} 잔차를 반환합니다

\textbf{deviance(fit)} 편차를 반환합니다

\textbf{fitted(fit)} fitted value를 반환합니다

\textbf{logLik(fit)} computes the logarithm of the likelihood and the number of

parameters

\textbf{AIC(fit)} AIC를 계산합니다

\section{Statistics}

\textbf{aov(formula)} 분산분석을 합니다

\textbf{anova(fit,...) }하나 혹은 그 이상의 모델에 대한 분산분석 테이블을 만듭니다

\textbf{density(x)} x에 대한 kernel density 를 추정합니다

\textbf{binom.test(), pairwise.t.test(), power.t.test(), prop.test(),
t.test(), ... help.search("test")} 참고하세요.

분포

\textbf{rnorm(n, mean=0, sd=1)} 가우시안분포(정규)

\textbf{rexp(n, rate=1)} 지수분포

\textbf{rgamma(n, shape, scale=1)} 감마분포

\textbf{rpois(n, lambda)} 포아송분포

\textbf{rweibull(n, shape, scale=1)} 웨이블분포

\textbf{rcauchy(n, location=0, scale=1)} 코시분포

\textbf{rbeta(n, shape1, shape2)} 베타분포

\textbf{rt(n, df)} ‘Student’ (t)

\textbf{rf(n, df1, df2)} Fisher?Snedecor (F) (c2)

\textbf{rchisq(n, df)} Pearson

\textbf{rbinom(n, size, prob) }이항분포

\textbf{rgeom(n, prob) }기하분포

\textbf{rhyper(nn, m, n, k)} 초기하분포

\textbf{rlogis(n, location=0, scale=1)} 로지스틱분포

\textbf{rlnorm(n, meanlog=0, sdlog=1)} 로그-정규분포

\textbf{rnbinom(n, size, prob)} 음이항분포

\textbf{runif(n, min=0, max=1)} 균일분포

\textbf{rwilcox(nn, m, n), rsignrank(nn, n)} Wilcoxon 통계량

이 함수들은 모두 r을 d, p, q 등으로 바꿔서 사용할 수 있는데, 각각 확률밀도(dfunc(x, ...)),
누적확률밀도(pfunc(x,...)), 분위수(qfunc(p, ...), with 0 < p < 1)를 나타냅니다

\section{Programming}

\textbf{function( arglist )} expr 함수 정의

\textbf{return(value)}

\textbf{if(cond)} expr

\textbf{if(cond)} cons.expr else alt.expr

\textbf{for(var in seq)} expr

\textbf{while(cond)} expr

\textbf{repeat} expr

\textbf{break}

\textbf{next}

{} 를 이용하여 묶어줘야 합니다

\textbf{ifelse(test, yes, no)} a value with the same shape as test filled
with elements from either yes or no

\textbf{do.call(funname, args)} executes a function call from the name of
the function and a list of arguments to be passed to it


\section{참여}
잘 못된 내용을 찾으셨거나, 보완, 수정 그리고 더 좋은 레퍼런스 카드를 만들기 위해서 함께 하실 분은 연락주세요.

\end{document}

